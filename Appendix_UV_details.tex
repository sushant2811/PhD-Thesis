% !TEX root = More_PhD_Thesis.tex
% !TEX encoding = UTF-8 Unicode
% !TEX spellcheck = en_US
\cleardoublepage
\chapter{UV extrapolation details}
\label{Appendix:UV_extra_details}

  \section{Deriving the UV cutoff}

  We present here a detailed derivation of the effective UV cutoff $\Lameff$
  as a function of the HO parameters (basis size $N$ and frequency $\Omega$)
  that we report in Subsec.~\ref{subsec:UV_cutoff_duality}.

  Consider the three-dimensional isotropic harmonic oscillator described
  by the Hamiltonian (in natural units with $\hbar=c=1$)
  %
  \begin{equation}
   H_\mathrm{HO} = \frac{p^2}{2\mu} + \frac{\mu\Omega^2r^2}{2} \,,
  \label{eq:H-HO}
  \end{equation}
  %
  where $\mu$ is the reduced mass and $\Omega$ denotes the oscillator
  frequency.  The eigenstates $\ket{n\ell m}$ of $H_\mathrm{HO}$ are
  degenerate in the quantum number $m$,
  %
  \begin{equation}
   H_\mathrm{HO}\ket{n\ell m} = E_{n\ell m}\ket{n\ell m}
  \label{eq:H-nlm}
  \end{equation}
  %
  with
  %
  \begin{equation}
   E_{n\ell m} = \left(2n+\ell+\frac32\right)\hw \,.
  \label{eq:E-nlm}
  \end{equation}
  %
  We use a slightly modified version of the conventions and notation
  from Ref.~\cite{Caprio:2012rv}.  The full three-dimensional wavefunction in
  configuration space is
  %
  \begin{equation}
   \psi_{n\ell m}(\vecr) = \braket{\vecr}{n\ell m}
   = \frac{u_{n\ell}(b;r)}{r}\YY_{\ell m}(\hat{\vecr})
  \label{eq:psiR}
  \end{equation}
  %
  with the reduced radial wavefunction
  %
  \begin{equation}
   u_{n\ell}(b;r) = N_{n\ell}(b)\times(r/b)^{\ell+1} \ee^{-(r/b)^2/2}
   L_n^{\ell+\nicefrac12}\big((r/b)^2\big) \,,
  \label{eq:uR}
  \end{equation}
  %
  where
  %
  \begin{equation}
   N_{n\ell}(b) = \sqrt{\frac{2n!}{b\,\Gamma(n+\ell+\nicefrac32)}} \,,
  \label{eq:N}
  \end{equation}
  %
  and
  %
  \begin{equation}
   b = (\mu\Omega)^{-1/2}
  \label{eq:b}
  \end{equation}
  %
  is the oscillator length.  The Fourier transform of Eq.~\eqref{eq:psiR} is
  %
  \begin{equation}
   \tilde{\psi}_{n\ell m}(\veck) = (2\pi)^{-\nicefrac32} \int\ddr\,
   \ee^{-\ii\veck\cdot\vecr} \psi_{n\ell m}(\vecr) \,.
  \end{equation}
  %
  It can be written as
  %
  \begin{equation}
   \tilde{\psi}_{n\ell m}(\veck)
   = (-\ii)^\ell\frac{\wt{u}_{n\ell}(b;k)}{k}\YY_{\ell m}(\hat{\veck}) \,,
  \label{eq:psiK}
  \end{equation}
  %
  such that $\wt{u}_{n\ell}(b;k)$ is the Fourier--Bessel transform of
  $u_{n\ell}(b;r)$, \ie,
  %
  \begin{equation}
   \wt{u}_{n\ell}(b;k) = \sqrt{\frac{2}{\pi}}\int_0^\infty\dd r'\,
   kr' j_\ell(kr')\,u_{n\ell}(b;r') \,.
  \label{eq:uK-FB}
  \end{equation}
  %
  This gives
  %
  \begin{equation}
   \wt{u}_{n\ell}(b;k) = (-1)^n\wt{N}_{n\ell}(b)\times(kb)^{\ell+1}
   \ee^{-(kb)^2/2} L_n^{\ell+\nicefrac12}\big((kb)^2\big)
  \label{eq:uK}
  \end{equation}
  %
  with
  %
  \begin{equation}
   \wt{N}_{n\ell}(b) = \sqrt{\frac{2n!\,b}{\Gamma(n+\ell+\nicefrac32)}} \,.
  \label{eq:N-tilde}
  \end{equation}

  \subsection{Smallest eigenvalue of $r^2$}

  In the derivation of $\Lameff$ below, we directly consider
  subspaces with an arbitrary (but fixed) angular momentum $\ell$, but
  quote S-wave ($\ell=0$) results explicitly for the sake of
  illustration.  Denoting the square root of the smallest eigenvalue
  of $r^2$ in the truncated oscillator subspace with angular momentum
  $\ell$ by $\rho$,\footnote{Strictly, we should write $\rho_\ell$ here,
  but we omit the additional subscript for notational simplicity.} the
  localized momentum-space eigenfunction for a hard-wall (Dirichlet)
  boundary condition in momentum space is
  %
  \begin{equation}
   \wt{\psi}_{\rho,\ell}(p) =
   \begin{cases}
    p\rho\,j_\ell(p\rho) \,, & 0 \leq p \leq x_\ell/\rho \,, \\
    0 \,, & p > x_\ell/\rho \,,
   \end{cases}
  \label{eq:psi-rho-ell}
  \end{equation}
  %
  where $x_\ell$ denotes the smallest positive zero of the spherical
  Bessel function $j_\ell$.  For S-waves, one simply has
  $\wt{\psi}_{\rho,\ell}(p)=\sin(p\rho)$ and $x_0=\pi$.
  The eigenfunction can be expanded in terms of oscillator functions as
  %
  \begin{equation}
   \wt{\psi}_\rho(p)
   = \sum_{k=0}^\infty \wt{c}_k(\rho) \wt{u}_{k}(p) \,,
  \label{eq:psi-rho-exp}
  \end{equation}
  %
  without basis truncation so far.  We have used the short-hand
  notation
  %
  \begin{equation}
   \wt{u}_n(p) \equiv \wt{u}_{n\ell}(1;p) \,.
  \label{eq:u-tilde}
  \end{equation}
  %
  In particular, we set the oscillator length $b$ to unity for the time
  being.  Exactly as in Subsec.~\ref{subsec:tale_of_tails} and in
  Ref.~\cite{More:2013rma}, the eigenvalue problem
  %
  \begin{equation}
   \left[r^2-\rho^2\right]\wt{\psi}_\rho(p) = 0
  \label{eq:r2rho2-psi}
  \end{equation}
  %
  becomes a set of coupled linear equations.  For S-waves, one can use
  the fact that the three-dimensional oscillator wavefunctions are
  directly related to the (odd) one-dimensional oscillator states and
  write
  %
  \begin{equation}
   r^2 = a^\dagger a + \frac{1}{2}
   + \frac{1}{2}\left[a^2+(a^\dagger)^2\right] \,,
  \label{eq:r2}
  \end{equation}
  %
  where $a$ and $a^\dagger$ are ladder operators, to obtain (after
  shifting some indices)
  %
  \begin{multline}
   \left[r^2-\rho^2\right]\wt{\psi}_\rho(p) = 0
   \iff \sum_{k=0}^\infty\left[(2k+3/2-\rho^2)\wt{c}_k(\rho)
   - \frac12\sqrt{2k+1}\sqrt{2k+3}\,\wt{c}_{k+1}(\rho)\right. \\
   \left.-\,\frac12\sqrt{2k}\sqrt{2k+1}\,\wt{c}_{k-1}(\rho)\right]\wt{u}_k(p)
   = 0 \mathspace (\ell = 0) \,.
  \label{eq:r2rho2-psi-exp}
  \end{multline}
  %
  More generally, a direct evaluation yields (\cf the analogous results
  for $p^2$ given in Subsec.~\ref{subsec:tale_of_tails})
  %
  \begin{equation}
   \mbraket{k\ell m}{r^2}{j\ell m} = (2k+\ell+3/2)\delta_k^j
   + \sqrt{k+1}\sqrt{k+\ell+3/2}\,\delta_k^{j+1}
   + \sqrt{k}\sqrt{k+\ell+1/2}\,\delta_k^{j-1} \,,
  \label{eq:r2-gen}
  \end{equation}
  %
  and thus we get
  %
  \begin{multline}
   \left[r^2-\rho^2\right]\wt{\psi}_\rho(p) = 0
   \iff \sum_{k=0}^\infty\left[(2k+\ell+3/2-\rho^2)\wt{c}_k(\rho)
   - \sqrt{k+1}\sqrt{k+\ell+3/2}\,\wt{c}_{k+1}(\rho)\right. \\
   \left.-\sqrt{k}\sqrt{k+\ell+1/2}\,\wt{c}_{k-1}(\rho)\right]
   \wt{u}_k(p) = 0
  \label{eq:r2rho2-psi-exp-ell}
  \end{multline}
  %
  for arbitrary angular momentum $\ell$.

  If the basis---and thus the sum in Eq.~\eqref{eq:r2rho2-psi-exp-ell}---is now
  truncated at some maximum $k \equiv n$, the last
  equation of the coupled set reads
  %
  \begin{equation}
   (2n+\ell+3/2-\rho^2)\,\wt{c}_{n}(\rho)
   - \sqrt{n}\sqrt{n+\ell+1/2}\,\wt{c}_{n-1}(\rho) = 0 \,.
  \label{eq:quant-cond-ell}
  \end{equation}
  %
  Further following Ref.~\cite{Furnstahl:2013vda}, we introduce the
  Fourier--Bessel transform of $\wt{\psi}_\rho(p)$ as
  %
  \begin{equation}
   \wt{\psi}_\rho(p)
   = \sqrt{\frac{2}{\pi}}\int_0^\infty \dd r\,\psi_\rho(r)\,pr\,j_\ell(pr) \,,
  \label{eq:psi-tilde-psi}
  \end{equation}
  %
  and use
  %
  \begin{equation}
   pr\,j_\ell(pr) = \sqrt{\frac{\pi}{2}}\sum_{n=0}^\infty
   \wt{u}_{n\ell}(b;p)u_{n\ell}(b;r) \mathtext{for arbitrary $b$}
  \label{eq:j-uKuR}
  \end{equation}
  %
  to infer
  %
  \begin{equation}
   \wt{c}_n(\rho) = \int_0^\infty\dd r\,\psi_\rho(r) u_n(r)
  \label{eq:c-tilde-1}
  \end{equation}
  %
  from Eq.~\eqref{eq:psi-rho-exp}.  To proceed, we use the asymptotic
  approximation~\cite{Furnstahl:2013vda,Deano2013}
  %
  \begin{multline}
   u_{n\ell}(b;r) \approx
   \frac{2^{1-n}}{\pi^{1/4}} \sqrt{\frac{(2n+2\ell+1)!}{b\,(n+\ell)!n!}}
   (4n-2\ell+3)^{-\frac{\ell+1}{2}} \\
   \times\sqrt{4n+2\ell+3}\,(r/b)\,j_\ell\left(\sqrt{4n+2\ell+3}\,(r/b)\right)
   \,,
  \label{eq:uR-asympt}
  \end{multline}
  %
  valid for $n\gg 1$.  Defining
  %
  \begin{equation}
   \beta_\ell = \sqrt{4n+2\ell+3}
  \label{eq:beta-ell}
  \end{equation}
  %
  and still setting $b=1$ at this point, we get
  %
  \begin{equation}
  \begin{split}
   \wt{c}_n(\rho) &\approx
   \frac{2^{1-n}}{\pi^{1/4}} \sqrt{\frac{(2n+2\ell+1)!}{(n+\ell)!n!}}
   \, \beta_\ell^{-\ell-1}\int_0^\infty\dd r\,\psi_\rho(r)\,
   \beta_\ell r \, j_\ell(\beta_\ell r) \\
   &= \frac{2^{1-n}}{\pi^{1/4}} \sqrt{\frac{(2n+2\ell+1)!}{(n+\ell)!n!}}
   \, \beta_\ell^{-\ell-1} \sqrt{\frac\pi2}\,\wt{\psi}_\rho(\beta_\ell) \\
   &= \frac{\pi^{1/4}}{2^{n-1/2}}\sqrt{\frac{(2n+2\ell+1)!}{n!(n+\ell)!}}
   \, \beta_\ell^{-\ell} \rho\,j_\ell(\beta_\ell\rho)
   \;.
  \end{split}
  \label{eq:c-tilde-2}
  \end{equation}
  %
  The intermediate and final steps here follow from
  Eqs.~\eqref{eq:psi-tilde-psi} and~\eqref{eq:psi-rho-ell},
  respectively, and we have the constraint $\rho<x_\ell/\beta_\ell$.
  Inserting Eq.~\eqref{eq:c-tilde-2} into the quantization
  condition~\eqref{eq:quant-cond-ell} gives an equation that is formally
  exactly the same as given in Subsec.~\ref{subsec:tale_of_tails} for the
  IR case.\footnote{Some relative minus signs---compare, for example,
  Eq.~\eqref{eq:quant-cond-ell} to Eq.~\eqref{quantize} in
  Subsec.~\ref{subsec:tale_of_tails}---have dropped out along the way.
  Note also that Subsec.~\ref{subsec:tale_of_tails} uses a slightly
  different convention for the momentum-space oscillator wavefunctions
  that does not involve the phase $(-1)^n$ in our Eq.~\eqref{eq:uK}.}

  \subsection{Cutoff identification}

  If we make the ansatz
  %
  \begin{equation}
  \rho = \frac{x_\ell}{\sqrt{4n+2\ell+3+2\Delta}} \,,
  \label{eq:rho-ansatz}
  \end{equation}
  %
  we get $\Delta=2$ in the limit $n \gg 1$ and $n\gg\ell$, independent
  of $\ell$.  As we discuss in Appendix~\ref{subsec:appendix_subleading},
  it is possible to
  derive subleading corrections to this result, which then depend on the
  angular momentum $\ell$, but turn out to be numerically insignificant
  for all present practical applications.

  With $\Nmax=2n+\ell$, and restoring the oscillator length $b$ by
  dimensional analysis, our result can also be written as
  %
  \begin{equation}
   \rho = \frac{x_\ell b}{\sqrt2}\left(\Nmax+\frac32+2\right)^{\!-1/2} \,.
  \label{eq:rho-Nmax}
  \end{equation}
  %
  This implies that the UV cutoff $\Lameff$ corresponding to the basis
  truncation at $\Nmax$ is not given by the naive estimate
  %
  \begin{equation}
   \Lambda_0 = \sqrt{2(\Nmax+3/2)}/b \,,
  \label{eq:Lambda-0-app}
  \end{equation}
  %
  which follows from $k = \sqrt{2\mu E}$ and Eq.~\eqref{eq:E-nlm},
  but rather by
  %
  \begin{equation}
   \Lambda_2 = \frac{x_\ell}{\rho} = \sqrt{2(\Nmax+3/2+2)}/b \,,
  \label{eq:Lambda-2-app}
  \end{equation}
  %
  completely dual to the configuration-space box size $L_2$ given in
  Eq.~\eqref{eq:L2_def}.

  \subsection{Subleading corrections to $L_2$ and $\Lambda_2$}
  \label{subsec:appendix_subleading}

  It is possible to derive subleading corrections to the result $\Delta=2$ that
  was derived in the previous subsection.  Because of the duality of
  configuration-space and momentum-space oscillator wavefunctions, the results
  derived in the following apply directly also to the effective box size $L_2$
  used to calculate IR corrections.

  For the smallest eigenvalue $\rho^2$ of the operator $r^2$ in the (truncated)
  oscillator basis we now wish to make the general ansatz
  %
  \begin{equation}
   \rho = \frac{x_\ell}{\sqrt{4n+2\ell+3+2
   \left(\Delta_0+\dfrac{\Delta_1}{n}+\dfrac{\Delta_2}{n^2}+\cdots\right)}} \,.
  \label{eq:rho-ansatz-gen}
  \end{equation}
  %
  In principle, there is an infinite sum of terms with increasing inverse
  powers
  of $n$ in Eq.~\eqref{eq:rho-ansatz-gen}, but we only give explicit results
  here up to $\OO(1/n^2)$.

  In Sec.~\ref{sec:Lambda-2}, the result $\Delta=\Delta_0=2$ was found by
  inserting \eqref{eq:c-tilde-2} into the quantization
  condition~\eqref{eq:quant-cond-ell} and then considering the limits $n\gg1$
  and $n\gg\ell$.  In practice, this is done by inserting the ansatz for
  $\rho=\rho(n)$ into $\tilde{c}_n(\rho)\sim \rho\,j_\ell(\beta_\ell\rho)$ and
  keeping only the leading term in an asymptotic expansion around $n=\infty$.

  To obtain the desired subleading corrections, it is however not sufficient to
  simply keep higher-order terms in this asymptotic expansion.  Instead, one
  first has to go back a few steps and also keep higher-order corrections to
  the leading asymptotic approximation for the oscillator wavefunctions given
  in Eq.~\eqref{eq:uR-asympt}.  Note that this approximation follows from using
  Eq.~(15) of Ref.~\cite{Deano2013}, which states that the generalized
  Laguerre polynomials have the asymptotic expansion
  %
  %\begin{widetext}
  \begin{multline}
   L_n^\alpha(z) = \frac{\Gamma(n+\alpha+1)}{n!}\ee^{z/2}
   \sum\limits_{m=0}^\infty \left(\frac{z}{2}\right)^m P_m(\alpha+1,z)
   (\kappa z)^{-\frac{m+\alpha}{2}} J_{m+\alpha}(2\sqrt{\kappa z})
  \label{eq:Laguerre-expansion}
  \end{multline}
  %\end{widetext}
  %
  with
  %
  \begin{subequations}\begin{align}
   \kappa &= n + \frac{\alpha+1}{2} \\ &= n+\frac34 \mathtext{for} \alpha=1/2
  \label{eq:kappa}
  \end{align}\end{subequations}
  %
  and
  %
  \begin{equation}
   P_0(c,z) = 1 \mathtext{,} P_1(c,z) = z/6 \mathtext{,} \cdots \,.
  \label{eq:P0-P1}
  \end{equation}
  %
  Using this in Eq.~\eqref{eq:uR} and keeping only the first ($m=0$) term gives
  Eq.~\eqref{eq:uR-asympt}.  More generally, one finds that for large $n$ the
  oscillator wavefunctions $u_{n\ell}(r)$ can be expressed as a sum
  %
  \begin{equation}
   u_{n\ell}(r) = u_{n\ell}^{(0)}(r) + u_{n\ell}^{(1)}(r) + \cdots \,,
  \label{eq:u-sum}
  \end{equation}
  %
  where the individual terms involve (spherical) Bessel functions of
  increasing order.  Recalling Eq.~\eqref{eq:c-tilde-1}, it then follows that
  also
  %
  \begin{equation}
   \tilde{c}_n(\rho) = \tilde{c}_n^{(0)}(\rho)+\tilde{c}_n^{(1)}(\rho)
   +\cdots \,.
  \label{eq:c-tilde-sum}
  \end{equation}
  %
  We already know that
  %
  \begin{equation}
   \tilde{c}_n^{(0)}(\rho) = C_\ell (n)\,\beta_\ell^{-\ell}
   \times\rho\,j_\ell(\beta_\ell\rho)
  \label{eq:c-tilde-j0}
  \end{equation}
  %
  with
  %
  \begin{equation}
   C_\ell(n) = \frac{\pi^{1/4}}{2^{n-1/2}}\sqrt{\frac{(2n+2\ell+1)!}
   {n!(n+\ell)!}} \,.
  \label{eq:C}
  \end{equation}
  %
  The key step in deriving Eq.~\eqref{eq:c-tilde-j0} was to express
  $\tilde{c}_n^{(0)}(\rho)$ in terms of $\tilde{\psi}_\rho$ by using the
  Fourier--Bessel transform, which could be done since asymptotically
  $u_{n\ell}^{(0)}(r)$ is simply proportional to $j_\ell(\beta_\ell\rho)$.
  More generally, for the individual terms in the expansion~\eqref{eq:u-sum}
  we have
  %
  \begin{equation}
   u_{n\ell}^{(k)}(r) = \frac{2^{1-n}}{\pi^{1/4}}
   \sqrt{\frac{(2n+2\ell+1)!}{(n+\ell)!n!}} \beta_\ell^{-(\ell+k)}
   P_k(\ell+3/2,r^2) r^{k+1} j_{\ell+k}(\beta_\ell r) \,.
 \end{equation}
  %
  This means that to obtain a generalization of Eq.~\eqref{eq:c-tilde-2},
  we have to calculate expressions of the form
  %
  \begin{equation}
   \tilde{c}_n^{(k)}(\rho) \sim \beta_\ell^{-(\ell+k)}
   \int_0^\infty\dd r\,\psi_\rho(r)\,  P_k(\ell+3/2,r^2)
   r^{k+1} j_{\ell+k}(\beta_\ell r) \,.
  \label{eq:c-tilde-k-1}
\end{equation}
  %
  To evaluate these integrals, it is more convenient to work with
  Riccati--Bessel functions,
  %
  \begin{equation}
   \hat\jmath_\nu(z) = zj_\nu(z) \,,
  \label{eq:j-hat-j}
  \end{equation}
  %
  in terms of which we have
  %
  \begin{equation}
   \tilde{c}_n^{(k)}(\rho) \sim \beta_\ell^{-(\ell+k+1)}
   \int_0^\infty\dd r\,\psi_\rho(r)\,  P_k(\ell+3/2,r^2)
   r^k \hat\jmath_{\ell+k}(\beta_\ell r) \,.
   \label{eq:c-tilde-k-1a}
  \end{equation}
  %
  For the Riccati--Bessel functions one has the derivative
  relation~\cite{abramowitz1964}
  %
  \begin{equation}
   \frac{\partial \hat\jmath_\nu(z)}{\partial z}
   = \frac{\nu+1}{z} \hat\jmath_\nu(z) - \hat\jmath_{\nu+1}(z) \,,
  \label{eq:j-hat-derivative}
  \end{equation}
  %
  from which it follows straightforwardly that
  %
  \begin{equation}
   \hat\jmath_{\nu+1}(\beta r) = \frac{1}{r}
   \left[\frac{\nu+1}\beta-\frac\dd{\dd\beta}\right]\hat\jmath_\nu(\beta r) \,.
  \label{eq:j-hat-beta}
  \end{equation}

  Using this relation $k$ times in Eq.~\eqref{eq:c-tilde-k-1}, we can eliminate
  the prefactor $r^k$ in favor of a differential operator with respect to a
  variable $\beta$,
  %
  \begin{multline}
   \tilde{c}_n^{(k)}(\rho) \sim \beta_\ell^{-(\ell+k+1)}
   \int_0^\infty\dd r\,\psi_\rho(r)\,  P_k(\ell+3/2,r^2)\,
   \left(\frac{\ell+k}{\beta}-\frac{\dd}{\dd\beta}\right) \\
   \times
   \left(\frac{\ell+k-1}{\beta}-\frac{\dd}{\dd\beta}\right)
   \cdots\left(\frac{\ell}{\beta}-\frac{\dd}{\dd\beta}\right)
   \hat\jmath_{\ell}(\beta r)\,\Big|_{\beta=\beta_\ell} \,.
  \label{eq:c-tilde-k-1b}
  \end{multline}
  %
  At this point, we have also conveniently reduced the order of
  the Riccati--Bessel functions so that we have the same function for each
  $\tilde{c}_n^{(k)}(\rho)$; all remaining additional $r$-dependence comes from
  the $P_k(\ell+3/2,r^2)$, which are polynomials in $r^2$.  This can also be
  eliminated by noting that
  %
  \begin{equation}
   r^2 \hat\jmath_\ell(\beta r) = \left(-\frac{\dd^2}{\dd\beta^2}
   +\frac{\ell(\ell+1)}{\beta^2}\right)\hat\jmath_\ell(\beta r) \,,
  \label{eq:r2-j-hat}
  \end{equation}
  %
  which follows immediately from the differential equations that defines
  the Riccati--Bessel functions and is formally just the free radial
  Schrödinger equation if one interchanges the variables $r$ and $\beta$.
  Altogether, we have found that we can write
  %
  \begin{equation}
   \tilde{c}_n^{(k)}(\rho) \sim \beta_\ell^{-(\ell+k+1)}\,
   \int_0^\infty\dd r\,\psi_\rho(r)\,\mathcal{D}_{\beta,\ell}^{(k)}\,
   \hat\jmath_{\ell}(\beta r)\,\Big|_{\beta=\beta_\ell} \,,
  \label{eq:c-tilde-k-1c}
  \end{equation}
  %
  where $\mathcal{D}_{\beta,\ell}^{(k)}$ is some differential operator (with
  respect to $\beta$) which can be pulled out of the integral.  The precise
  form of this operator can be obtained from the equations above, but it is
  actually not important here.  At this point we can proceed exactly as in
  Eq.~\eqref{eq:c-tilde-2} and write, restoring the full prefactor,
  %
  \begin{equation}
  \begin{split}
   \tilde{c}_n^{(k)}(\rho) &=
   \frac{2^{1-n}}{\pi^{1/4}} \sqrt{\frac{(2n+2\ell+1)!}{(n+\ell)!n!}}
   \,\beta_\ell^{-(\ell+k+1)} \,
   \mathcal{D}_{\beta,\ell}^{(k)} \int_0^\infty\dd r\,\psi_\rho(r)\,
   \hat\jmath_{\ell}(\beta r)\,\Big|_{\beta=\beta_\ell} \\[0.8em]
   &= \frac{2^{1-n}}{\pi^{1/4}} \sqrt{\frac{(2n+2\ell+1)!}{(n+\ell)!n!}}
   \,\beta_\ell^{-(\ell+k+1)} \,
   \sqrt{\frac\pi2}\,
   \mathcal{D}_{\beta,\ell}^{(k)}\,\tilde{\psi}_\rho(\beta)\,
   \Big|_{\beta=\beta_\ell} \\[0.8em]
   &= C_\ell(n)\,\beta_\ell^{-(\ell+k+1)}
   \times \mathcal{D}_{\beta,\ell}^{(k)}\,\hat\jmath(\beta\rho)
   \Big|_{\beta=\beta_\ell} \\[0.8em]
   &= C_\ell(n)\,\beta_\ell^{-\ell-k} \times P_k(\ell+3/2,\rho^2)\,
   \rho^{k+1} j_{\ell+k}(\beta_\ell\rho) \,.
  \end{split}
  \label{eq:c-tilde-k-2}
  \end{equation}
  %
  We have used here that $\tilde{\psi}_\rho(\beta) = \beta\rho\,
  j_\ell(\beta\rho) = \hat\jmath(\beta\rho)$ for $\beta \leq x_\ell/\rho$,
  and that we can ultimately apply the operator
  $\mathcal{D}_{\beta,\ell}^{(k)}$ to get back the original expression as
  in Eq.~\eqref{eq:c-tilde-k-1}, only with $r$ replaced by
  $\rho$.  The coefficients $C_\ell(n)$ have been defined in Eq.~\eqref{eq:C}.

  With these general expressions for all terms in the expansion of
  $\tilde{c}_n(\rho)$, we can now write the quantization
  condition~\eqref{eq:quant-cond-ell} as
  %
  \begin{equation}
   (2n+\ell+3/2-\rho^2)\times\sum_{k=0}^{\kmax} \tilde{c}_n^{(k)}(\rho)
    - \sqrt{n}\sqrt{n+\ell+1/2}
   \sum_{k=0}^{\kmax}\tilde{c}_{n-1}^{(k)}(\rho) = 0 \,.
  \label{eq:quant-cond-ell-gen}
\end{equation}
  %
  The appropriate truncation index $\kmax$ in this equation depends on both
  $\ell$ and the desired order for the subleading corrections.  To solve for
  these, we insert an ansatz of the form~\eqref{eq:rho-ansatz-gen} into
  Eq.~\eqref{eq:quant-cond-ell-gen} and solve for the coefficients $\Delta_0$,
  $\Delta_1$, etc. by performing an asymptotic expansion around $n=\infty$.  To
  do this consistently, it is important to keep all terms that can
  contribute to the
  maximum order we are interested in.  In general, there are cancellations
  between the polynomial prefactors $P_k(\ell+3/2,\rho^2)\times\rho^{k+1}$ and
  the spherical Bessel functions $j_{\ell+k}(\beta_\ell\rho)$ since the latter
  contribute inverse powers of $\beta_\ell\rho$, which become more prominent
  with increasing $\ell$.  At least for $\ell=0$ and $\ell=1$ we find
  that $\kmax=2$
  is sufficient to get the corrections up to and including $\OO(1/n^2)$.  The
  results, obtained with computer algebra software (Wolfram Mathematica), are
  %
  \begin{align}
   \ell &= 0:  & \Delta_1 &= \frac{3-2\pi^2}{48} \;,
   & \Delta_2 &= \frac{-7(3-2\pi^2)}{192} \,, \\
   \ell &= 1:  & \Delta_1 &= \frac{1}{48}(3-2\pi^2) \;,
   & \Delta_2 &= \frac{3(5+2x_1^2)}{64} \,.
   \label{eq:lasteq}
  \end{align}
  %
  One always has $\Delta_0=2$, independent of $\ell$.

  %%%%%%%%%%%%%%%%%%%%%%%%%%%%%%%%%%%%%%%%%%%%%%%%%%%%%%%%%%%%%%%%%%%%%%%%%%%
  \begin{table}[htbp]
  \centering
  %
  \begingroup
  \renewcommand\arraystretch{1.25}
  %
  \label{tab:rho-ell-0}
  %
  \begin{tabular}{r|
   >{\centering\arraybackslash}p{5.1em}|
   >{\centering\arraybackslash}p{5.1em}|
   >{\centering\arraybackslash}p{5.1em}||
   >{\centering\arraybackslash}p{5.1em}
  }
    \hline\hline
    $n$
    & $\rho$, $\OO(1/n^0)$ & $\rho$, $\OO(1/n^1)$ & $\rho$, $\OO(1/n^2)$
    & $\rho$, exact \\
    \hline
    1 & 0.94723 & 0.97876 & 0.92548 & 0.95857 \\
    2 & 0.81116 & 0.82075 & 0.81234 & 0.81629 \\
    3 & 0.72073 & 0.72518 & 0.72258 & 0.72355 \\
    4 & 0.65507 & 0.65756 & 0.65647 & 0.65681 \\
    5 & 0.60460 & 0.60617 & 0.60562 & 0.60576 \\
    6 & 0.56425 & 0.56531 & 0.56500 & 0.56507 \\
    7 & 0.53103 & 0.53178 & 0.53159 & 0.53163 \\
    8 & 0.50306 & 0.50362 & 0.50350 & 0.50352 \\
    9 & 0.47909 & 0.47952 & 0.47944 & 0.47945 \\
   10 & 0.45825 & 0.45859 & 0.45853 & 0.45854 \\
   11 & 0.43991 & 0.44018 & 0.44014 & 0.44015 \\
   12 & 0.42361 & 0.42384 & 0.42380 & 0.42381 \\
   \hline\hline
  \end{tabular}
  %
  \endgroup
  %
  \caption{Comparison of the smallest distance scale $\rho$ at different orders
  in the $1/n$ expansion to the exact answer for several values of $n$.  S-wave
  results ($\ell = 0$).}
  \end{table}
  %%%%%%%%%%%%%%%%%%%%%%%%%%%%%%%%%%%%%%%%%%%%%%%%%%%%%%%%%%%%%%%%%%%%%%%%%%%%%
  \begin{table}[htbp]
  \centering
  %
  \begingroup
  \renewcommand\arraystretch{1.25}
  %
  \label{tab:rho-ell-1}
  %
  \begin{tabular}{r|
   >{\centering\arraybackslash}p{5.1em}|
   >{\centering\arraybackslash}p{5.1em}|
   >{\centering\arraybackslash}p{5.1em}||
   >{\centering\arraybackslash}p{5.1em}
  }
    \hline\hline
    $n$
    & $\rho$, $\OO(1/n^0)$ & $\rho$, $\OO(1/n^1)$ & $\rho$, $\OO(1/n^2)$
    & $\rho$, exact \\
    \hline
    1 & 1.2462 & 1.3481 & 1.1464 & 1.2764 \\
    2 & 1.0898 & 1.1214 & 1.0860 & 1.1047 \\
    3 & 0.98054 & 0.99560 & 0.98424 & 0.98920 \\
    4 & 0.89868 & 0.90730 & 0.90242 & 0.90423 \\
    5 & 0.83441 & 0.83990 & 0.83741 & 0.83821 \\
    6 & 0.78220 & 0.78596 & 0.78455 & 0.78495 \\
    7 & 0.73871 & 0.74142 & 0.74055 & 0.74077 \\
    8 & 0.70175 & 0.70378 & 0.70321 & 0.70334 \\
    9 & 0.66984 & 0.67141 & 0.67101 & 0.67110 \\
    10 & 0.64192 & 0.64316 & 0.64288 & 0.64293 \\
    11 & 0.61722 & 0.61822 & 0.61802 & 0.61805 \\
    12 & 0.59517 & 0.59599 & 0.59584 & 0.59586 \\
   \hline\hline
  \end{tabular}
  %
  \endgroup
  %
  \caption{Comparison of the smallest distance scale $\rho$ at different orders
  in the $1/n$ expansion to the exact answer for several values of $n$.  P-wave
  results ($\ell = 1$).}
  \end{table}
  %%%%%%%%%%%%%%%%%%%%%%%%%%%%%%%%%%%%%%%%%%%%%%%%%%%%%%%%%%%%%%%%%%%%%%%%%%%%%

  In Tables~\ref{tab:rho-ell-0} and~\ref{tab:rho-ell-1} we show (for $\ell=0$
  and $\ell=1$, respectively) how subsequent inclusion of the correction terms
  makes the values for $\rho$ as defined in Eq.~\eqref{eq:rho-ansatz-gen}
  converge to the exact results, which have been calculated numerically.


  \section{P\"{o}schl-Teller states and form factors}
