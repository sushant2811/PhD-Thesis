% !TEX root = More_PhD_Thesis.tex
\cleardoublepage
\chapter{Extrapolation}

	As we have seen in the introduction, the Harmonic Oscillator (HO) basis is
	routinely used in Low-energy Nuclear Physics (LENP) calculations.  We also saw
	that
	the size of Hamiltonian matrix that we need to diagonalize grows factorially
	with the number of nucleons, severely restricting the number of terms that
	can be kept in the basis expansion.  The maximum number of terms in the basis
	expansion is often denoted by $\Nmax$, so the single particle nuclear wave
	function is given by
  \beq
	  \psi_{N_{\rm max}}^{\Omega}(r) = \sum_{\alpha = 0}^{N_{\rm max}} c_{\alpha}
	  \phi_{\alpha}^{\Omega}(r) \;.
	\label{eq:single_particle_truncation}
	\eeq
	$\phi_{\alpha}^{\Omega}(r)$ in Eq.~\eqref{eq:single_particle_truncation} are
	the HO wave functions; $\Omega$ is the frequency of the HO \footnote{
  In LENP, the oscillator frequency is denoted by $\Omega$ rather than $\omega$.
	}.
	Thus the energy obtained in the HO basis -- $E(\Nmax, \Omega)$ -- is a
	function of $\Nmax$ and $\Omega$.  This is illustrated in
	Fig.~\ref{fig:H6_function_Omega}.
	%
	\begin{figure}[h]
			\centering
			\includegraphics[width=0.5\textwidth]
			{Extrapolation/He6_Eb_vs_hw_kvnn10_srg_lam2p0_combined_Nmax6_Kval1_L1_0b.pdf}
			\caption{Ground state energy for $^6$He as a function of $\Nmax$ and
			  $\Omega$.  Figure taken from \cite{Furnstahl2012}. }
			\label{fig:H6_function_Omega}
	\end{figure}
	%
  We see that as we go to higher $\Nmax$, the curves get flatter with respect
	to $\Omega$, or in other words the dependence on $\Omega$ drops out.

	The goal is to extrapolate to $\Nmax = \infty$ from a finite $\Nmax$.
	The most widely used extrapolation scheme employs an exponential in $\Nmax$
	form
	\beq
	  E(\Nmax) = \Einf + a e^{-c \Nmax}\;,
		\label{eq:exp_Nmax_extrapolation}
	\eeq
	where $a$ and $c$ are determined separately for each $\hw$ (with the
	option of constraining the fit to get the same asymptotic $\Einf$ value).
	Figure~\ref{fig:exp_Nmax_extrapolation_6He} shows estimate for the ground
	state energy for $^6$He obtained using the extrapolation form of
	Eq.~\eqref{eq:exp_Nmax_extrapolation}.
	%
	\begin{figure}[h]
		\centering
		\includegraphics[width=0.5\textwidth]
		{Extrapolation/6He_gs_range2_142.pdf}
		\caption{The estimate for exact $^6$He ground state energy using
		  Eq.~\eqref{eq:exp_Nmax_extrapolation}.  Extrapolated answer from the
			constrained fit and the experimental binding energy are indicated by
			horizontal lines.  Figure from \cite{Maris2009}.}
		\label{fig:exp_Nmax_extrapolation_6He}
	\end{figure}
  %

	The exponential in $\Nmax$ extrapolation is widely used in literature and
	seems to work quite well
	\cite{Hagen:2007hi,Bogner:2007rx,Forssen:2008qp,Maris:2008ax,Roth:2009cw}.
	There are however many open questions about this extrapolation scheme such as
	the answer for $\Einf$ depends on the oscillator frequency $\Omega$ and it is
	not clear which the best choice for $\Omega$.  The terms $a$ and $c$ in
	Eq.~\eqref{eq:exp_Nmax_extrapolation} are fit to data.  There is no way to
	extract these terms for one nucleus and use it to predict something else.
	Moreover, the physical motivation for exponential in $\Nmax$ extrapolation is
	slim at best.  It has been claimed that for larger nuclei $\Nmax$ is a
	logarithmic measure of the number of states \cite{Bogner:2007rx}.

	An alternative approach to extrapolations is motivated by effective field
	theory (EFT) and based instead on explicitly considering the infrared (IR)
	and ultraviolet (UV) cutoffs imposed by a finite oscillator
	basis~\cite{Coon:2012ab}.  The truncation in the oscillator basis introduces
	a maximum length scale (or an IR cutoff) and also a maximum momentum scale
	(or an UV cutoff).  These length and momentum scales can be motivated by
	the classical turning points denoted by $L_0$ and $\Lambda_0$ respectively.
	We have
	\begin{align}
		L_0 = \sqrt{2 (\Nmax + 3/2)} b \;, \nonumber \\
		\Lambda_0 = \sqrt{2 (\Nmax + 3/2)} \hbar/b \;.
		\label{eq: L0_Lam0_cutoff}
	\end{align}
	$b$ is the oscillator length given by
	$\displaystyle b = \sqrt{\hbar/ m \Omega}$.
	%
	The errors due the finite IR (UV) cutoff are called the IR (UV) errors.
	To draw a lattice analogy, IR errors stem from the finite box size,
	and the UV errors are a result of the finite lattice spacing (or the
	graininess of the lattice).

	This approach of thinking of the HO truncation in terms of IR and UV cutoffs
	has led to lot of development in past three years.
  We can choose the oscillator parameters such that one of the cutoffs in
	Eqs.~\eqref{eq: L0_Lam0_cutoff} is large, making the errors due to that cutoff
	small and focus on the errors due to the other cutoff.
	The first attempt and
	test for a theoretically motivated IR correction was made in
	\cite{Furnstahl2012}.  These corrections were made theoretically sound
	in \cite{More:2013rma, Furnstahl:2013vda}.  In \cite{Konig:2014hma} we looked
	at the UV correction for the deuteron.
	Our papers \cite{More:2013rma, Furnstahl:2013vda, Konig:2014hma} have led to
	physically motivated extrapolation schemes and will form the basis of the next
	two sections.  Insights from our work has also led to development of
	extrapolation schemes (both in IR and UV) for the many-body case by other
	groups.  We will touch upon these developments in
	Sec.~\ref{sec:related_development_IR}.

	\section[Infrared story]{Infrared story
		\footnote{Based on \cite{More:2013rma} and \cite{Furnstahl:2013vda}}}

	\section{Ultraviolet story}

	\section{Related development and Moving forward}

	\label{sec:related_development_IR}
