% !TEX root = More_PhD_Thesis.tex
\cleardoublepage
\chapter{Extrapolation}

	As we have seen in the introduction, the Harmonic Oscillator (HO) basis is
	routinely used in Low-energy Nuclear Physics (LENP) calculations.  We also saw
	that
	the size of Hamiltonian matrix that we need to diagonalize grows factorially
	with the number of nucleons, severely restricting the number of terms that
	can be kept in the basis expansion.  The maximum number of terms in the basis
	expansion is often denoted by $\Nmax$, so the single particle nuclear wave
	function is given by
	\beq
	\psi_{N_{\rm max}}^{\Omega}(r) = \sum_{\alpha = 0}^{N_{\rm max}} c_{\alpha}
	\phi_{\alpha}^{\Omega}(r) \;.
	\label{eq:single_particle_truncation}
	\eeq
	$\phi_{\alpha}^{\Omega}(r)$ in Eq.~\eqref{eq:single_particle_truncation} are
	the HO wave functions; $\Omega$ is the frequency of the HO \footnote{
	In LENP, the oscillator frequency is denoted by $\Omega$ rather than $\omega$.
	}.
	Thus the energy obtained in the HO basis -- $E(\Nmax, \Omega)$ -- is a
	function of $\Nmax$ and $\Omega$.  This is illustrated in
	Fig.~\ref{fig:H6_function_Omega}.
	%
	\begin{figure}[h]
		\centering
		\includegraphics[width=0.5\textwidth]
		{Extrapolation/He6_Eb_vs_hw_kvnn10_srg_lam2p0_combined_Nmax6_Kval1_L1_0b.pdf}
		\caption{Ground state energy for $^6$He as a function of $\Nmax$ and
		  $\Omega$.  Figure taken from \cite{Furnstahl2012}. }
		\label{fig:H6_function_Omega}
	\end{figure}
	%
	We see that as we go to higher $\Nmax$, the curves get flatter with respect
	to $\Omega$, or in other words the dependence on $\Omega$ drops out.

	The goal is to extrapolate to $\Nmax = \infty$ from a finite $\Nmax$.
	The most widely used extrapolation scheme employs an exponential in $\Nmax$
	form
	\beq
	E(\Nmax) = \Einf + a e^{-c \Nmax}\;,
	\label{eq:exp_Nmax_extrapolation}
	\eeq
	where $a$ and $c$ are determined separately for each $\hw$ (with the
	option of constraining the fit to get the same asymptotic $\Einf$ value).
	Figure~\ref{fig:exp_Nmax_extrapolation_6He} shows estimate for the ground
	state energy for $^6$He obtained using the extrapolation form of
	Eq.~\eqref{eq:exp_Nmax_extrapolation}.
	%
	\begin{figure}[h]
	\centering
	\includegraphics[width=0.5\textwidth]
	{Extrapolation/6He_gs_range2_142.pdf}
	\caption{The estimate for exact $^6$He ground state energy using
	  Eq.~\eqref{eq:exp_Nmax_extrapolation}.  Extrapolated answer from the
		constrained fit and the experimental binding energy are indicated by
		horizontal lines.  Figure from \cite{Maris2009}.}
	\label{fig:exp_Nmax_extrapolation_6He}
	\end{figure}
	%

	The exponential in $\Nmax$ extrapolation is widely used in literature and
	seems to work quite well
	\cite{Hagen:2007hi,Bogner:2007rx,Forssen:2008qp,Maris:2008ax,Roth:2009cw}.
	There are however many open questions about this extrapolation scheme such as
	the answer for $\Einf$ depends on the oscillator frequency $\Omega$ and it is
	not clear which the best choice for $\Omega$.  The terms $a$ and $c$ in
	Eq.~\eqref{eq:exp_Nmax_extrapolation} are fit to data.  There is no way to
	extract these terms for one nucleus and use it to predict something else.
	Moreover, the physical motivation for exponential in $\Nmax$ extrapolation is
	slim at best.  It has been claimed that for larger nuclei $\Nmax$ is a
	logarithmic measure of the number of states \cite{Bogner:2007rx}.

	An alternative approach to extrapolations is motivated by effective field
	theory (EFT) and based instead on explicitly considering the infrared (IR)
	and ultraviolet (UV) cutoffs imposed by a finite oscillator
	basis~\cite{Coon:2012ab}.  The truncation in the oscillator basis introduces
	a maximum length scale (or an IR cutoff) and also a maximum momentum scale
	(or an UV cutoff).  These length and momentum scales can be motivated by
	the classical turning points denoted by $L_0$ and $\Lambda_0$ respectively.
	We have
	\begin{align}
	L_0 = \sqrt{2 (\Nmax + 3/2)} b \;, \nonumber \\
	\Lambda_0 = \sqrt{2 (\Nmax + 3/2)} \hbar/b \;.
	\label{eq: L0_Lam0_cutoff}
	\end{align}
	$b$ is the oscillator length given by
	$\displaystyle b = \sqrt{\hbar/ m \Omega}$.
	%
	The errors due the finite IR (UV) cutoff are called the IR (UV) errors.
	To draw a lattice analogy, IR errors stem from the finite box size,
	and the UV errors are a result of the finite lattice spacing (or the
	graininess of the lattice).  Ideally, we would like the box size to be as
	large as possible and the lattice spacing to be as small as possible.
	Because of finite computational power, this is not always possible though
	and therefore we need reliable extrapolation schemes in both IR as well as UV.

	This approach of thinking of the HO truncation in terms of IR and UV cutoffs
	has led to lot of development in past three years.
	We can choose the oscillator parameters such that one of the cutoffs in
	Eqs.~\eqref{eq: L0_Lam0_cutoff} is large, making the errors due to that cutoff
	small and focus on the errors due to the other cutoff.
	The first attempt and
	test for a theoretically motivated IR correction was made in
	\cite{Furnstahl2012}.  These corrections were made theoretically sound
	in \cite{More:2013rma, Furnstahl:2013vda}.  In \cite{Konig:2014hma} we looked
	at the UV correction for the deuteron.
	Our papers \cite{More:2013rma, Furnstahl:2013vda, Konig:2014hma} have led to
	physically motivated extrapolation schemes and will form the basis of the next
	two sections.  Insights from our work has also led to development of
	extrapolation schemes (both in IR and UV) for the many-body case by other
	groups.  We will touch upon these developments in
	Sec.~\ref{sec:related_development_IR}.

	\section[Infrared story]{Infrared story
	\footnote{Based on \cite{More:2013rma} and \cite{Furnstahl:2013vda}}}

	As mentioned in the introduction, there was a lack of well motivated
	extrapolation schemes in LENP and this is where our work comes in.
	We started with the two-body case because it is more tractable
	mathematically.  Note that extrapolation is usually not necessary for the
	two-body problem, because the convergence is reached before we run out of
	computational power.  This allows us to test our extrapolation formulas.
	Once we establish that the approach works for the two-body case, we can hope
	to extend the approach to few- and many-body case.

	As mentioned earlier, IR cutoff effectively puts system in a finite box.
	We need to find appropriate box length such that
	\beq
	E(\Nmax) = E(L_{\rm box})\;.
	\label{eq:Nmax_Lbox_correspondence}
	\eeq
	Note that our original problem was to find $\Einf \equiv E(\Nmax = \infty)$
	given $E$ at a finite $\Nmax$.  Once we make the correspondence in
	Eq.~\eqref{eq:Nmax_Lbox_correspondence} and express energy as a function of
	box length, we can use various techniques (discussed later) to estimate
	$E (L_{\rm box} = \infty)$ which equals $E(\Nmax = \infty)$.

	\subsection{Tale of tails}

	The box size is usually bigger than the range of the potential.  Thus
	imposing the IR cutoff modifies the asymptotic part (or tail) of the
	bound-state wave function.  Our early work focused on trying to estimate the
	appropriate box size by matching the tails of wave functions in the truncated
	HO basis to the tails of wave functions in boxes of different lengths.

	Our strategy was to use a range of model potentials for which the
	Schr\"odinger equation can be solved
	analytically or to any desired precision numerically to broadly test
	and illustrate various features, and then turn to the deuteron for a
	real-world example.  In particular we considered:
	%
	\bea
	V_{\rm sw}(r) &=& -V_0\, \theta(R-r)   \qquad \mbox{[square well]}
	\;,
	\label{eq:Vsw}
	\\
	V_{\rm exp}(r) &=& -V_0\, e^{-(r/R)}  \qquad\ \ \mbox{[exponential]}
	\;,
	\\
	V_{\rm g}(r) &=& -V_0\, e^{-(r/R)^2}  \qquad\ \mbox{[Gaussian]}
	\;,
	\label{eq:Vg}
	\\
	V_{\rm q}(r) &=& -V_0\, e^{-(r/R)^4} \qquad\ \mbox{[quartic]}
	\;,
	\label{eq:Vq}
	\eea
  %
	where for each of the models we work in units with $\hbar = 1$, reduced mass
	$\mu=1$, and $R=1$, but consider different values for $V_0$.  For the
	realistic potential we use the Entem-Machleidt 500\,MeV chiral EFT
	N$^3$LO potential~\cite{Entem:2003ft} and unitarily evolve it with the
	similarity renormalization group (SRG).  These potentials provide a
	diverse set of tests for universal properties.  Because we can go to
	very high \hw\ and $\Nmax$ for the two-particle bound states (and
	therefore large $\LamUV$), it is possible to always ensure that UV
	corrections are negligible.

	We start with empirical considerations before presenting an
  analytical understanding.  An example of how this correspondence between
	the HO truncation and a hard wall at specific length plays out is
	presented in Fig.~\ref{fig:sq_well_tail_matching}.
	%
	\begin{figure}[h]
		\centering
		\includegraphics[width=0.6 \textwidth]
		{Extrapolation/sqwell_V4_Nmax4_hw6L0b.pdf}
		\caption{(a) The exact radial wave
    function (dashed) for a square well Eq.~\eqref{eq:Vsw} with depth
	  $V_0=4$ (and $\hbar = \mu = R = 1$) is compared to the wave function
		obtained from an HO basis truncated at $\Nmax = 4$ with $\hw=6$ (solid).
		The spatial extent of the wave function obtained from the HO basis
		truncation is dictated by the square of HO wave function for the highest
		radial quantum number (dot-dashed).
		(b) The wave functions obtained from imposing a Dirichlet
	  boundary condition at $L_0$, $\LA$ and $L_2$ are compared to the wave
		function in truncated HO basis. }
		\label{fig:sq_well_tail_matching}
	\end{figure}
  %
	In the top panel, the exact
	ground-state radial wave function (dashed) for the square well in
	Eq.~\eqref{eq:Vsw} is compared to the solution in an oscillator basis
	truncated at $\Nmax = 4$ determined by
	diagonalization (solid).  The truncated basis cuts off the tail of the
	exact wave function because the individual basis wave functions have a
	radial extent that depends on \hw\ (from the Gaussian part) and on the
	largest power of $r$ (from the polynomial part).  The latter is given
	by $\Nmax = 2n + l$.  With $\Nmax = 4$ and $l=0$, this means that $n=2$ gives
	the largest power.

	The cutoff will then be determined by the $n=2$ oscillator wave
	function, $u_{n=2}^{\rm HO}(r)$, whose square (which is the relevant
	quantity) is also plotted in the top panel (dot-dashed).  It is
	evident that the tail of the wave function in the truncated basis is
	fixed by this squared wave function.  Our premise is that the HO truncation
	is well modeled by a hard-wall (Dirichlet) boundary condition at $r=L$.
	If so, the question remains how best to \emph{quantitatively} determine $L$
	given	$\Nmax$ and $\hw$.  In the bottom panel of
	Fig.~\ref{fig:sq_well_tail_matching} we show the wave functions	for several
	possible choices for $L$.
	$L_0$ corresponds to choosing the classical
	turning point (i.e. the half-height point of the tail of
	$[u^{HO}_{n=2}(r)]^2$); it is manifestly too small.  The authors of
	\cite{Furnstahl2012} advocated an improved choice for $L$ given by
	\beq
	\LA = L_0 + 0.54437\, b\, (L_0/b)^{-1/3} \;.
  \label{eq:LA}
	\eeq
	%
	The length $\LA$ in Eq.~\eqref{eq:LA} is obtained by linear extrapolation
	from the slope at the half-height point.
	However, choosing
	\beq
	L = L_2 \equiv \sqrt{2 (\Nmax + 3/2 + 2)}b
	\label{eq:L2_def}
	\eeq
	was found to work the best in almost all examples.

	The most direct illustration of this conclusion comes from the
	bound-state energies.  In the example in Fig.~\ref{fig:sq_well_tail_matching},
	the exact energy (in dimensionless units) is $-1.51$ while the
	result for the basis truncated at $\Nmax=4$ is $-1.33$, which is therefore
	what we	hope to reproduce.  With $L_0$, the energy is $-0.97$, with $\LA$ it
	is $-1.21$, and with $L_2$ it is $-1.29$.  While this is only one
	example of a model problem, we have found that $L_2$ always gives a
	better energy estimate than $\LA$ (and something like
	$L_3 \equiv \sqrt{2 (\Nmax + 3/2 + 3)}b$ is almost always worse).

	Another signature that demonstrates the suitability of $L_2$ is that
	points from many different $\hw$ and $\Nmax$ values all lie on the same
	curve.  Figures.~\ref{fig:spatial_cutoff_vs_HO_gauss} and
	\ref{fig:spatial_cutoff_vs_HO_square_well} show the energies from a
	wide range of HO truncations for $L_0$, $\LA$ and $L_2$ for the
	Gaussian well and the square well potential, respectively.  The
	energies for different $\hbar\Omega$ and $\Nmax$ lie on the same smooth
	and unbroken curve if we use $L_2$ but not with the other choices.  For
	$L=L_0$ and $L=\LA$, one finds that sets of points with different
	$\hbar\Omega$ but same $\Nmax$ fall on smooth, $\Nmax$-dependent curves.  For
	the square well, there are small discontinuities visible even for
	$L=L_2$.  At the square well radius, the wave function's second
	derivative is not smooth, and this is difficult to approximate with a
	finite set of oscillator functions.  This lack of UV convergence is
	likely the origin of the very small discontinuities.
	%
	\begin{figure}[h]
	\centering
	\includegraphics[width=0.6\textwidth]
	  {Extrapolation/spatialcutoffgaussV5b}
	\caption{Ground-state energies versus $L_0$ (top),
		$\LA$ (middle), and $L_2$ (bottom) for a Gaussian potential well
		Eq.~\eqref{eq:Vg} with $V_0=5$ and $R=1$.
		The crosses are the
		energies from HO basis truncation.
		The energies obtained by
		numerically solving the Schr{\"o}dinger equation with a Dirichlet
		boundary condition at $L$ lie on the solid line.
		The horizontal dotted lines mark
		the exact energy $\Einf=-1.27$.}
  \label{fig:spatial_cutoff_vs_HO_gauss}
  \end{figure}
	%
	\begin{figure}[h]
	\centering
	\includegraphics[width=0.6\textwidth]
	  {Extrapolation/spatialcutoffsqwellV4b}
	\caption{Ground-state energies versus $L_0$ (top),
	  $\LA$ (middle), and $L_2$ (bottom) for a square well potential well
	  Eq.~\eqref{eq:Vsw} with $V_0=4$ and $R=1$.
	  The crosses are the
	  energies from HO basis truncation.
	  The energies obtained by
	  numerically solving the Schr{\"o}dinger equation with a Dirichlet
	  boundary condition at $L$ lie on the solid line.
	  The horizontal dotted lines mark
	  the exact energy $\Einf=-1.51$.}
  \label{fig:spatial_cutoff_vs_HO_square_well}
  \end{figure}
	%
	As a further test, we solve the Schr{\"o}dinger equation with a
	vanishing Dirichlet boundary condition (solid lines in
	Figs.~\ref{fig:spatial_cutoff_vs_HO_gauss} and
	\ref{fig:spatial_cutoff_vs_HO_square_well}), and compare to the
	energies obtained from the HO truncations (crosses).  The finite
	oscillator basis energies are well approximated by a Dirichlet
	boundary condition with a mapping from the oscillator $\hbar\Omega$
	and $\Nmax$ to an equivalent length given by $L_2$.  Note that for large
	$\Nmax$, the differences between $L_0$, $\LA$ and $L_2$ may be smaller
	than other uncertainties involved in nuclear calculations, but for
	practical calculations one will want to use small $\Nmax$ results, where
	these considerations are very relevant.

	These results from model calculations are consistent with those from
	realistic potentials applied to the deuteron.  To illustrate this, we
	use the N$^3$LO 500\,MeV potential of Entem and
	Machleidt~\cite{Entem:2003ft}.  We generate results in an HO basis
	with \hw\ ranging from $1$ to $100\,\mbox{MeV}$ and $\Nmax$ from $4$ to
	$100$ (in steps of 4 to avoid HO artifacts for the
	deuteron~\cite{Bogner:2007rx}).  We then restrict the data to where UV
	corrections are negligible (see Section~\ref{Sec:UV_story}).
	%
	\begin{figure}[h]
	\centering
  \includegraphics[width=0.6\textwidth]
	{Extrapolation/spatialcutoffdeutrn}
  \caption{Ground-state energies versus $L_0$ (top),
  	$\LA$ (middle), and $L_2$ (bottom) for the Entem-Machleidt 500\,MeV
  	N$^3$LO potential~\cite{Entem:2003ft}.  The horizontal dotted lines
  	mark the exact energy $\Einf = -2.2246\,\mbox{MeV}$. }
  \label{fig:spatial_cutoff_vs_HO_deuteron}
  \end{figure}
	%
	Figure~\ref{fig:spatial_cutoff_vs_HO_deuteron} shows that the
	criterion of a continuous curve with the smallest spread of points
	clearly favors $L_2$.



	\subsubsection{Analytical derivation of $L_2$}

	In the asymptotic region (this is the region where IR cutoff is imposed), the
	potential is negligible and the only relevant
	part of the Hamiltonian is the kinetic energy or the $p^2$ operator.
	In what follows, we analytically compute the smallest eigenvalue
	$\kappa^2_{\rm min}$ of $p^2$ in a finite oscillator basis and will
	see that $\kappa_{\rm min} = \pi/L_2$. In the remainder of this
	Subsection, we set the oscillator length to one. We focus on $s$-waves
	and thus consider wave functions that are regular at the origin, i.e.
	the radial wave functions are identical to the odd wave functions of
	the one-dimensional harmonic oscillator.

	The localized eigenfunction of the operator $p^2$ with smallest
	eigenvalue $\kappa^2$ is
	%
	\bea
	\label{eigen}
	\psi_{\kappa}(r) = \left\{\begin{array}{ll}
	\sin{\kappa r}\ , & 0 \le r \le {\pi\over\kappa}\\
	0 \ , & r > {\pi\over\kappa}
	\end{array}\right. \; .
	\eea
	%
	We employ the $s$-wave oscillator functions
	%
	\bea
	\varphi_{2n+1}(r) &=&(-1)^n \sqrt{2 n!\over\Gamma(n+3/2)} r
	L^{1\over 2}_n\left(r^2\right)
	e^{-{r^2\over 2}} \nonumber\\
	&=&\left(\pi^{1\over 2} 2^{2n} (2n+1)!\right)^{-1/2}H_{2n+1}(r)
	e^{-{r^2\over 2}} \; , \nonumber
	\eea
	%
	with energy $E=(2n +3/2)\hbar\Omega$. Here, $L_n^{1/2}$ denotes the
	Laguerre polynomial, and it is convenient to rewrite this function in
	terms of the Hermite polynomial $H_n$. We expand the
	eigenfunction~(\ref{eigen}) as
	%
	\beq
	\label{expand}
	\psi_{\kappa}(r) = \sum_{n=0}^\infty c_{2n+1}(\kappa)\varphi_{2n+1}(r) \; .
	\eeq
	%
	Before we turn to the computation of the expansion coefficients
	$c_{2n+1}(\kappa)$, we consider the eigenvalue problem for the
	operator $p^2$.  We have
	%
	\beq
	p^2 = a^\dagger a +{1\over 2} -{1\over 2}
	\left(a^2 +\left(a^\dagger\right)^2\right) \;,
	\eeq
	%
	where $a$ and $a^\dagger$ denote the annihilation and creation
	operator for the one-dimensional harmonic oscillator, respectively.
	The matrix of $p^2$ is tridiagonal in the oscillator basis.
	For the matrix representation, we order the basis states as
	$(\varphi_1, \varphi_3, \varphi_5, \ldots)$. Thus, the eigenvalue
	problem $p^2-\kappa^2=0$ becomes a set of rows of coupled linear
	equations. In an infinite basis, the eigenvector $(c_1(\kappa),
	c_3(\kappa), c_5(\kappa), \ldots )$ identically satisfies every row of
	the eigenvalue problem for any value of $\kappa$. In a finite basis
	$(\varphi_1, \varphi_3, \varphi_5,\ldots \varphi_{2n+1})$, however,
	the last row of the eigenvalue problem
	%
	%\bea
	%\label{quant}
	%\lefteqn{\left(2n+3/2 -\kappa^2\right) c_{2n+1}(\kappa) =}\nonumber\\
	% &&{1\over
	%  2}\sqrt{2n}\sqrt{2 n+1}c_{2n-1}(\kappa) \ ,
	%\eea
	%
	\beq
	\label{quant}
	\left(2n+3/2 -\kappa^2\right) c_{2n+1}(\kappa) =
	 {1\over
	  2}\sqrt{2n}\sqrt{2 n+1} \, c_{2n-1}(\kappa) \; ,
	\eeq
	%
	can only be fulfilled for certain values of $\kappa$, and this is the
	quantization condition. To solve this eigenvalue problem we need
	expressions for the expansion coefficients $c_{2n+1}(\kappa)$
	for $n\gg 1$. Those can be derived analytically as follows.


	We rewrite the eigenfunction~(\ref{eigen}) as a Fourier transform
	%
	\beq
	\psi_{\kappa} (r) = \sqrt{2\over \pi} \int\limits_0^\infty dk
	\tilde{\psi}_{\kappa}(k) \sin kr \; ,
	\eeq
	%
	and expand the sine function in terms of oscillator functions as
	%
	\beq
	\sin kr = \sqrt{\pi\over 2} \sum_{n=0}^\infty (-1)^n \varphi_{2n+1}(r)
	\varphi_{2n+1}(k) \; .
	\eeq
	%
	Thus, the expansion coefficients in Eq.~(\ref{expand}) are given in
	terms of the Fourier transform $\tilde{\psi}_\kappa(k)$ as
	%
	\beq
	\label{integ}
	c_{2n+1}(\kappa) = (-1)^n\int\limits_0^\infty dk\, \tilde{\psi}_{\kappa}(k)
	\varphi_{2n+1}(k) \; .
	\eeq
	%
	So far, all manipulations have been exact.  We need an expression for
	$c_{2n+1}(\kappa)$ for $n\gg 1$ and use the asymptotic expansion
	%
	\beq
	\label{approxwf}
	\varphi_{2n+1}(k) \approx {(-1)^n\sqrt{2}\over \pi^{1/4}}
	{(2n-1)!!\over \sqrt{(2n)!}}
	\sin (\sqrt{4n+3}k) \; ,
	\eeq
	%
	which is valid for $|k|\ll \sqrt{2n}$, see~\cite{gradshteyn}.
	%
	Using this approximation, one finds (making use of Fourier transforms)
	%
	\bea
	\label{phi}
	c_{2n+1}(\kappa) &\approx& \pi^{1/4} {(2n-1)!!\over \sqrt{(2n)!}}
	\psi_{\kappa}(\sqrt{4n+3}) \nonumber\\
	&=&\pi^{1/4}{(2n-1)!!\over \sqrt{(2n)!}}
	\sin (\sqrt{4n+3}\kappa) \; ,
	\eea
	%
	with $\kappa \le \pi/\sqrt{4n+3}$ due to Eq.~(\ref{eigen}).

	Let us return to the solution of the quantization
	condition~(\ref{quant}).  We make the ansatz
	%
	\beq
	\kappa = {\pi\over\sqrt{4n+3+2\Delta}} \; ,
	\eeq
	%
	and must assume that $\Delta > 0$.  This ansatz is well motivated, since the
	naive semiclassical estimate $\kappa = \pi/L_0$ yields $\Delta=0$. We
	insert the expansion coefficients~(\ref{phi}) into the quantization
	condition~(\ref{quant}) and consider its leading-order approximation
	for $n\gg 1$ and $n\gg \Delta$. This yields
	%
	\beq
	\Delta = 2
	\eeq
	%
	as the solution. Recalling that a truncation of the basis at
	$\varphi_{2n+1}$ corresponds to the maximum energy
	$E=(2n+3/2)\hbar\Omega$, we see that we must identify
	$\Nmax \equiv N = 2n$. Thus,
	$\kappa_{\rm min} = \pi/L_2$ is the lowest momentum in a finite
	oscillator basis with $n \gg 1$ basis states (and not $1/b$ as stated in
	Ref.~\cite{Coon:2012ab}).  It is clear from its
	very definition that $\pi/L_2$ is also (a very precise approximation of)
	infrared cutoff in a finite	oscillator basis, and that $L_2$ (and not $b$
	as stated in Refs.~\cite{Stetcu:2006ey,Stetcu:2007ms}) is the radial extent
	of the oscillator basis and the analog to the extent of the lattice in the
	lattice computations \cite{Luscher:1985dn}.

	The derivation of our key result $\kappa_{\rm min}=\pi/L_2$ is based
	on the assumption that the number of shells $N$ fulfills $N\gg 1$.
	Table~\ref{tab1} shows a comparison of numerical results for
	$\kappa_{\rm min}$ in different model spaces. We see that
	$\pi/L_2$ is a very good approximation already for $N=2$, with a
	deviation of about 1\%.

	\begin{table}[ht]
	\centering
	\begin{tabular}{|c|c|c|c|}\hline
	$N$ & $\kappa_{\rm min}$ & $\pi/L_2$ & $\pi/L_0$ \\\hline
	   0 & 1.2247 & 1.1874 & 1.8138\\
	   2 & 0.9586 & 0.9472 & 1.1874\\
	   4 & 0.8163 & 0.8112 & 0.9472\\
	   6 & 0.7236 & 0.7207 & 0.8112\\
	   8 & 0.6568 & 0.6551 & 0.7207\\
	  10 & 0.6058 & 0.6046 & 0.6551\\
	  12 & 0.5651 & 0.5642 & 0.6046\\
	  14 & 0.5316 & 0.5310 & 0.5642\\
	  16 & 0.5035 & 0.5031 & 0.5310\\
	  18 & 0.4795 & 0.4791 & 0.5031\\
	  20 & 0.4585 & 0.4582 & 0.4791\\\hline
	\end{tabular}
	\caption{Comparison between the lowest momentum $\kappa_{\rm min}$, $\pi/L_2$,
	 and $\pi/L_0$ for model spaces with up to $N$ oscillator quanta.}
	\label{tab1}
	\end{table}

	Note that this approach can be generalized to other localized
	bases.  The (numerical) computation of the lowest eigenvalue of the momentum
	operator $p^2$ yields the box size $L$ corresponding to the employed Hilbert
	space.

	\subsubsection{EFT-like approach}

	We mentioned that the relevant operator for IR truncation is $p^2$.  To
	get a better understanding of the correspondence between the HO truncation
	and a hard wall at $L_2$, let's compare the spectrum of $p^2$ in the two
	cases.
	%
	\begin{figure}[h]
	\centering
	\includegraphics[width=0.6\textwidth]
	{Extrapolation/momentum_eigenfunctions1.pdf}
	\caption{Eigenfunctions of $p^2$ in the truncated HO basis compared to those
	  in a box of size $L_2$. }
	\label{fig:mom_eigen_fns}
	\end{figure}
	%
	In Fig.~\ref{fig:mom_eigen_fns}, we compare the low-lying eigenfunctions of
	$p^2$ in the truncated HO basis to the eigenfunctions in a box of size $L_2$.
	As we will see later, the asymptotic or near the wall difference between the
	two eigenfunctions are high-momentum effects irrelevant for the
	long-wavelength physics of the bound states.

	Another way to look at this is to compute the number $M(k)$ of ($s$-wave)
	states up to a momentum $k$.  We find
	\bea
	  M(k)&=& {\rm Tr} \left[\Theta\left(\hbar^2k^2-p^2\right)
	   \Theta\left(E-{p^2\over 2m}-{m\over 2}\Omega^2 r^2\right) \right]
	  \nonumber\\
	  &\approx&{1\over 2\pi\hbar} \int\limits_{-\hbar k}^{\hbar k}\! dp
	  \int\limits_0^\infty\! dr \,
	   \Theta\left(\hbar^2k^2-p^2\right)
	   \Theta\left(E-{p^2\over 2m}-{m\over 2}\Omega^2 r^2\right)
	   \; .
	\eea
	%
	Here, we apply the semiclassical approximation and write the trace
	as a phase-space integral.  We assume $\hbar^2k^2/(2m)\le
	E$, perform the integrations and use $E/(\hbar\Omega)= N+3/2$
	\footnote{For the sake of brevity we replace $\Nmax$ by $N$}. This
	yields
	%
	\bea
	  M(k) = {bk\over 2\pi}\sqrt{2N+3-b^2k^2}
	   +{N+3/2\over \pi}\arcsin{bk\over\sqrt{2N+3}}
	  \;,
	  \label{Mstair}
	\eea
	%
	where $b$ is the oscillator length.
	Figure~\ref{fig:staircase} shows a comparison between the quantum
	mechanical staircase function and the semiclassical
	estimate~(\ref{Mstair}) for $N=32$.  For sufficiently small values of
	$kb\ll\sqrt{2N}$, the number of $s$-wave momentum eigenstates grows
	linearly, and inspection of Eq.~(\ref{Mstair}) shows that the slope at
	the origin is $L_0/\pi$ semiclassically.  The linear growth of the
	number of eigenstates of $p^2$ with $k$ clearly demonstrate that --- at
	not too large values of $kb$ --- the spectrum of $p^2$ in the oscillator
	basis is similar to the spectrum of $p^2$ in a spherical
	box.
  %
	\begin{figure}[h]
	\centering
	\includegraphics[width=0.6\textwidth]{Extrapolation/staircase_N32_v2}
	\caption{The staircase function of the $s$ states of
	  the operator $p^2$ in a finite oscillator basis with $N=32$ (black)
	  compared to its semiclassical estimate (smooth red curve). $M(k)$
	  denotes the number of states of the operator $p^2$ with eigenvalues
	  $p^2\le\hbar^2 k^2$.}
	\label{fig:staircase}
	\end{figure}

	As a final example of the correspondence between the HO truncation and the
	hard wall at $L_2$, we look at the ground state wave functions of a square
	well in the two bases in Fig.~\ref{fig:HO_q_and_rspace_sqwell}.
	%
	\begin{figure}[h]
	\centering
	\includegraphics[width=0.65\textwidth]
	{Extrapolation/HO_q_and_rspace_sqwell_V0_4p0_Nmax8_hw18_v2}
	\caption{Ground-state wave functions for a square well potential of depth
	  $V_0=4$ (see Eq.~\eqref{eq:Vsw}; lengths are in units of $R$ and energies
		in units of $1/R^2$ with $\hbar^2/\mu = 1$) from solving the Schr\"odinger
	  equation with a truncated harmonic oscillator basis with $\hbar\Omega = 18$
	  and $N = 8$ (dashed) and with a
	  Dirichlet boundary condition at $r=L_2$ given from Eq.~\eqref{eq:L2_def}
		(solid).  The coordinate-space radial wave functions in a) exhibit a
		difference at $r$ near 1.5, but the Fourier-transformed wave functions in
		b) are in close agreement at low $k$, showing that the differences are
		high-momentum modes.}
	\label{fig:HO_q_and_rspace_sqwell}
	\end{figure}
	%
	The binding momentum in this case is $1.7$ (in units of $1/R$).  The
	Fourier-transformed wave functions differ at much larger momentum and this
	difference is irrelevant for the long-wavelength physics of bound states.
	Thus the use of Dirichlet boundary condition to take into account the HO
	truncation is similar in spirit to the use of contact interactions to
	describe the effect of unknown short-ranged forces on long-wavelength probes.


	\subsection{The $S$-matrix way}

	\subsection{Radii and phase shifts}

	\begin{figure}[h]
	\centering
	\includegraphics[width=0.6\textwidth]
	{Extrapolation/Deuteron_radii_n3lo500_inset_v4}
	\caption{Deuteron radius squared versus $L_0$ (top) and
  	$L_2$ (bottom) for the Entem-Machleidt 500\,MeV N$^3$LO
  	potential~\cite{Entem:2003ft}.  The horizontal dotted lines mark the
  	exact radius squared $r^2_{\infty} = 3.9006~{\rm fm}^2$. The insets
  	show a magnification of data at smaller lengths $L_n$.}
	\label{fig:deuteron_radii}
	\end{figure}
  %
	Similar comments apply to the computation
	of the radius. Figure~\ref{fig:deuteron_radii} shows that the
	numerical results for the squared radius, when plotted as a function
	of $L_2$ (but not as a function of $L_0$), fall on a continuous curve
	with minimal spread.

	\subsection{Higher angular momenta}

	\section{Ultraviolet story}
	\label{Sec:UV_story}

	\section{Related development and Moving forward}

	\label{sec:related_development_IR}
