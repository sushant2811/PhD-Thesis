% !TEX root = More_PhD_Thesis.tex
\cleardoublepage
\chapter{Extrapolation}
\label{chap:Extrapolation}

	As we have seen in the introduction, the harmonic oscillator (HO) basis is
	routinely used in low-energy nuclear physics (LENP) calculations.  We also saw
	that
	the size of Hamiltonian matrix that we need to diagonalize grows factorially
	with the number of nucleons (cf.~Fig.~\ref{fig:matrix_dimension_growth}),
	severely restricting the number of terms that
	can be kept in the basis expansion.
	The single particle nuclear wave
	function with the $\Nmax$ truncation introduced in
	Chapter~\ref{chap:Intro} is given by
	\beq
	\psi_{N_{\rm max}}^{\Omega}(r) = \sum_{\alpha = 0}^{N_{\rm max}} c_{\alpha}
	\varphi_{\alpha}^{\Omega}(r) \;.
	\label{eq:single_particle_truncation}
	\eeq
	$\varphi_{\alpha}^{\Omega}(r)$ in Eq.~\eqref{eq:single_particle_truncation}
	are
	the HO wave functions;
	$\Omega$ is the frequency of the HO \footnote{
	In LENP, the oscillator frequency is often denoted by $\Omega$ rather
	than $\omega$.}.
	For reference, the $S$-wave HO wave function is given by
	\beq
	\varphi_{\alpha}^{\Omega}(r) =  \mathcal{N} \ee^{
		\frac{- \mu \Omega}{2 \hbar} r^2} L_{\alpha}^{1/2}
		(\frac{\mu \Omega}{\hbar} r^2) \;,
	\label{eq:HO_S_wave_written_out}
	\eeq
	where $\mathcal{N}$ is the normalization constant, $\mu$ is the
	reduced mass, and $L_{\alpha}^{1/2}$ denotes the generalized Laguerre
	polynomial.

	The energy obtained in the HO basis---$E(\Nmax, \Omega)$---is a
	function of $\Nmax$ and $\Omega$.  This is illustrated in
	Fig.~\ref{fig:H6_function_Omega}.
	%
	\begin{figure}[h]
		\centering
		\includegraphics[width=0.55\textwidth]
		{Extrapolation/He6_Eb_vs_hw_kvnn10_srg_lam2p0_combined_Nmax6_Kval1_L1_0b.pdf}
		\caption{Ground state energy for $^6$He as a function of $\Nmax$ and
		  $\Omega$.  Figure taken from \cite{Furnstahl2012}. }
		\label{fig:H6_function_Omega}
	\end{figure}
	%
	We see that as we go to higher $\Nmax$, the curves get flatter with respect
	to $\Omega$, or in other words the dependence on $\Omega$ drops out.

	The goal is to extrapolate to $\Nmax = \infty$ from a finite $\Nmax$.
	The most widely used extrapolation scheme employs an exponential in $\Nmax$
	form
	\beq
	E(\Nmax) = \Einf + a e^{-c \Nmax}\;,
	\label{eq:exp_Nmax_extrapolation}
	\eeq
	where $a$ and $c$ are determined separately for each $\hw$ (with the
	option of constraining the fit to get the same asymptotic $\Einf$ value).
	Figure~\ref{fig:exp_Nmax_extrapolation_6He} shows estimate for the ground
	state energy for $^6$He obtained using the extrapolation form of
	Eq.~\eqref{eq:exp_Nmax_extrapolation}.
	%
	\begin{figure}[h]
	\centering
	\includegraphics[width=0.55\textwidth]
	{Extrapolation/6He_gs_range2_142.pdf}
	\caption{The estimate for exact $^6$He ground state energy using
	  Eq.~\eqref{eq:exp_Nmax_extrapolation}.  Extrapolated answer from the
		constrained fit and the experimental binding energy are indicated by
		horizontal lines.  Figure from \cite{Maris2009}.}
	\label{fig:exp_Nmax_extrapolation_6He}
	\end{figure}
	%

	The exponential in $\Nmax$ extrapolation is widely used in literature and
	seems to work quite well
	\cite{Hagen:2007hi,Bogner:2007rx,Forssen:2008qp,Maris:2008ax,Roth:2009cw}.
	There are however many open questions about this extrapolation scheme such as
	the answer for $\Einf$ depends on the oscillator frequency $\Omega$ and it is
	not clear which is the best choice for $\Omega$.  The terms $a$ and $c$ in
	Eq.~\eqref{eq:exp_Nmax_extrapolation} are fit to data.  There is no way to
	extract these terms for one nucleus and use it to predict something else.
	Moreover, the physical motivation for an exponential in $\Nmax$ extrapolation
	is slim at best.  It has been claimed that for larger nuclei $\Nmax$ is a
	logarithmic measure of the number of states \cite{Bogner:2007rx}.
	This would account for the exponential behavior, but there is no
	demonstration that it follows in general or with a specific
	logarithmic dependence.

	An alternative approach to extrapolations is motivated by effective field
	theory (EFT) and based instead on explicitly considering the infrared (IR)
	and ultraviolet (UV) cutoffs imposed by a finite oscillator
	basis~\cite{Coon:2012ab}.  The truncation in the oscillator basis introduces
	a maximum length scale (or an IR cutoff) and also a maximum momentum scale
	(or an UV cutoff).  These length and momentum scales can be motivated by
	the classical turning points denoted by $L_0$ and $\Lambda_0$ respectively.
	We have
	\begin{align}
	L_0 = \sqrt{2 (\Nmax + 3/2)} b \;,  \nonumber\\
	\Lambda_0 = \sqrt{2 (\Nmax + 3/2)} \hbar/b \;.
	\label{eq:L0_Lam0_cutoff}
	\end{align}
	$b$ is the oscillator length given by
	$\displaystyle b = \sqrt{\hbar/ m \Omega}$.
	%
	The errors due the finite IR (UV) cutoff are called the IR (UV) errors.
	To draw a lattice analogy, IR errors stem from the finite box size,
	and the UV errors are a result of the finite lattice spacing (or the
	graininess of the lattice).  Ideally, we would like the box size to be as
	large as possible and the lattice spacing to be as small as possible.
	Because of finite computational power, this is not always possible though
	and therefore we need reliable extrapolation schemes in both IR as well as UV.

	This approach of thinking of the HO truncation in terms of IR and UV cutoffs
	has led to lot of development in the past three years.
	Note that $b$ appears in the numerator for $L_0$ and in the denominator
	for $\Lambda_0$, so it is not possible to make both cutoffs large
	simultaneously.
	We can choose the oscillator parameters such that one of the cutoffs in
	Eqs.~\eqref{eq:L0_Lam0_cutoff} is large, making the errors due to that cutoff
	small and focus on the errors due to the other cutoff.
	The first attempt and
	test for a theoretically motivated IR correction was made in
	\cite{Furnstahl2012}.  These corrections were made theoretically sound
	in \cite{More:2013rma, Furnstahl:2013vda}.  In \cite{Konig:2014hma} we looked
	at the UV correction for the deuteron.
	Our papers \cite{More:2013rma, Furnstahl:2013vda, Konig:2014hma} have led to
	physically motivated extrapolation schemes and will form the basis of the next
	two sections.  Insights from our work have also led to development of
	extrapolation schemes (both in IR and UV) for the many-body case by other
	groups.  We will touch upon these developments in
	Subsec.~\ref{subsec:IR_front}.

	\section[Infrared story]{Infrared story
	\footnote{Based on \cite{More:2013rma} and \cite{Furnstahl:2013vda}}}
	\label{sec:IR_story}

	As mentioned in the introduction, there was a lack of well motivated
	extrapolation schemes in LENP and this is where our work comes in.
	We started with the two-body case because it is more tractable
	mathematically.  Note that extrapolation is usually not necessary for the
	two-body problem, because convergence is reached before we run out of
	computational power.  This allows us to test our extrapolation formulas.
	Once we establish that the approach works for the two-body case, we can hope
	to extend the approach to few- and many-body case.

	As mentioned earlier, IR cutoff effectively puts system in a finite box.
	We need to find appropriate box length such that
	\beq
	E(\Nmax) = E(L_{\rm box})\;.
	\label{eq:Nmax_Lbox_correspondence}
	\eeq
	Note that our original problem was to find $\Einf \equiv E(\Nmax = \infty)$
	given $E$ at a finite $\Nmax$.  Once we make the correspondence in
	Eq.~\eqref{eq:Nmax_Lbox_correspondence} and express energy as a function of
	box length, we can use various techniques (discussed later) to estimate
	$E (L_{\rm box} = \infty)$ which equals $E(\Nmax = \infty)$.

	\subsection{Tale of tails}
	\label{subsec:tale_of_tails}

	The box size is usually bigger than the range of the potential.  Thus
	imposing the IR cutoff modifies the asymptotic part (or tail) of the
	bound-state wave function.  Our early work focused on trying to estimate the
	appropriate box size by matching the tails of wave functions in the truncated
	HO basis to the tails of wave functions in boxes of different lengths.

	Our strategy was to use a range of model potentials for which the
	Schr\"odinger equation can be solved
	analytically or to any desired precision numerically to broadly test
	and illustrate various features, and then turn to the deuteron for a
	real-world example.  In particular we considered:
	%
	\bea
	V_{\rm sw}(r) &=& -V_0\, \theta(R-r)   \qquad \mbox{[square well]}
	\;,
	\label{eq:Vsw}
	\\
	V_{\rm exp}(r) &=& -V_0\, e^{-(r/R)}  \qquad\ \ \mbox{[exponential]}
	\;,
	\\
	V_{\rm g}(r) &=& -V_0\, e^{-(r/R)^2}  \qquad\ \mbox{[Gaussian]}
	\;,
	\label{eq:Vg}
	\\
	V_{\rm q}(r) &=& -V_0\, e^{-(r/R)^4} \qquad\ \mbox{[quartic]}
	\;,
	\label{eq:Vq}
	\eea
  %
	where for each of the models we work in units with $\hbar = 1$, reduced mass
	$\mu=1$, and $R=1$, but consider different values for $V_0$.  For the
	realistic potential we use the Entem-Machleidt 500\,MeV chiral EFT
	N$^3$LO potential~\cite{Entem:2003ft} and unitarily evolve it with the
	similarity renormalization group (SRG).  These potentials provide a
	diverse set of tests for universal properties.  Because we can go to
	very high \hw\ and $\Nmax$ for the two-particle bound states (and
	therefore large $\LamUV$), it is possible to always ensure that UV
	corrections are negligible.

	We start with empirical considerations before presenting an
  analytical understanding.  An example of how this correspondence between
	the HO truncation and a hard wall at specific length plays out is
	presented in Fig.~\ref{fig:sq_well_tail_matching}.
	%
	\begin{figure}[h]
		\centering
		\includegraphics[width=0.6 \textwidth]
		{Extrapolation/sqwell_V4_Nmax4_hw6L0b.pdf}
		\caption{(a) The exact radial wave
    function (dashed) for a square well Eq.~\eqref{eq:Vsw} with depth
	  $V_0=4$ (and $\hbar = \mu = R = 1$) is compared to the wave function
		obtained from an HO basis truncated at $\Nmax = 4$ with $\hw=6$ (solid).
		The spatial extent of the wave function obtained from the HO basis
		truncation is dictated by the square of HO wave function for the highest
		radial quantum number (dot-dashed).
		(b) The wave functions obtained from imposing a Dirichlet
	  boundary condition (bc) at $L_0$, $\LA$ and $L_2$ are compared to the wave
		function in truncated HO basis. }
		\label{fig:sq_well_tail_matching}
	\end{figure}
  %
	In the top panel, the exact
	ground-state radial wave function (dashed) for the square well in
	Eq.~\eqref{eq:Vsw} is compared to the solution in an oscillator basis
	truncated at $\Nmax = 4$ determined by
	diagonalization (solid).  The truncated basis cuts off the tail of the
	exact wave function because the individual basis wave functions have a
	radial extent that depends on \hw\ (from the Gaussian part;
	cf.~Eq.~\eqref{eq:HO_S_wave_written_out}) and on the
	largest power of $r$ (from the polynomial part).  The latter is given
	by $\Nmax = 2n + l$.  With $\Nmax = 4$ and $l=0$, this means that $n=2$ gives
	the largest power.

	The cutoff will then be determined by the $n=2$ oscillator wave
	function, $u_{n=2}^{\rm HO}(r)$, whose square (which is the relevant
	quantity) is also plotted in the top panel (dot-dashed).  It is
	evident that the tail of the wave function in the truncated basis is
	fixed by this squared wave function.  Our premise is that the HO truncation
	is well modeled by a hard-wall (Dirichlet) bc at $r=L$.
	If so, the question remains how best to \emph{quantitatively} determine $L$
	given	$\Nmax$ and $\hw$.  In the bottom panel of
	Fig.~\ref{fig:sq_well_tail_matching} we show the wave functions	for several
	possible choices for $L$.
	$L_0$ corresponds to choosing the classical
	turning point (i.e. the half-height point of the tail of
	$[u^{HO}_{n=2}(r)]^2$); it is manifestly too small.  The authors of
	\cite{Furnstahl2012} advocated an improved choice for $L$ given by
	\beq
	\LA = L_0 + 0.54437\, b\, (L_0/b)^{-1/3} \;.
  \label{eq:LA}
	\eeq
	%
	The length $\LA$ in Eq.~\eqref{eq:LA} is obtained by linear extrapolation
	from the slope at the half-height point.
	However, choosing
	\beq
	L = L_2 \equiv \sqrt{2 (\Nmax + 3/2 + 2)}b
	\label{eq:L2_def}
	\eeq
	was found to work the best in almost all examples.

	The most direct illustration of this conclusion comes from the
	bound-state energies.  In the example in Fig.~\ref{fig:sq_well_tail_matching},
	the exact energy (in dimensionless units) is $-1.51$ while the
	result for the basis truncated at $\Nmax=4$ is $-1.33$, which is therefore
	what we	hope to reproduce.  With $L_0$, the energy is $-0.97$, with $\LA$ it
	is $-1.21$, and with $L_2$ it is $-1.29$.  While this is only one
	example of a model problem, we have found that $L_2$ always gives a
	better energy estimate than $\LA$ (and something like
	$L_3 \equiv \sqrt{2 (\Nmax + 3/2 + 3)}b$ is almost always worse).

	Another signature that demonstrates the suitability of $L_2$ is that
	points from many different $\hw$ and $\Nmax$ values all lie on the same
	curve.  Figures~\ref{fig:spatial_cutoff_vs_HO_gauss} and
	\ref{fig:spatial_cutoff_vs_HO_square_well} show the energies from a
	wide range of HO truncations for $L_0$, $\LA$ and $L_2$ for the
	Gaussian well and the square well potential, respectively.  The
	energies for different $\hbar\Omega$ and $\Nmax$ lie on the same smooth
	and unbroken curve if we use $L_2$ but not with the other choices.  For
	$L=L_0$ and $L=\LA$, one finds that sets of points with different
	$\hbar\Omega$ but same $\Nmax$ fall on smooth, $\Nmax$-dependent curves.  For
	the square well, there are small discontinuities visible even for
	$L=L_2$.  At the square well radius, the wave function's second
	derivative is not smooth, and this is difficult to approximate with a
	finite set of oscillator functions.  This lack of UV convergence is
	likely the origin of the very small discontinuities.
	%
	\begin{figure}[h]
	\centering
	\includegraphics[width=0.6\textwidth]
	  {Extrapolation/spatialcutoffgaussV5b}
	\caption{Ground-state energies versus $L_0$ (top),
		$\LA$ (middle), and $L_2$ (bottom) for a Gaussian potential well
		Eq.~\eqref{eq:Vg} with $V_0=5$ and $R=1$.
		The crosses are the
		energies from HO basis truncation.
		The energies obtained by
		numerically solving the Schr{\"o}dinger equation with a Dirichlet
		bc at $L$ lie on the solid line.
		The horizontal dotted lines mark
		the exact energy $\Einf=-1.27$.}
  \label{fig:spatial_cutoff_vs_HO_gauss}
  \end{figure}
	%
	\begin{figure}[h]
	\centering
	\includegraphics[width=0.6\textwidth]
	  {Extrapolation/spatialcutoffsqwellV4b}
	\caption{Ground-state energies versus $L_0$ (top),
	  $\LA$ (middle), and $L_2$ (bottom) for a square well potential well
	  Eq.~\eqref{eq:Vsw} with $V_0=4$ and $R=1$.
	  The crosses are the
	  energies from HO basis truncation.
	  The energies obtained by
	  numerically solving the Schr{\"o}dinger equation with a Dirichlet
	  bc at $L$ lie on the solid line.
	  The horizontal dotted lines mark
	  the exact energy $\Einf=-1.51$.}
  \label{fig:spatial_cutoff_vs_HO_square_well}
  \end{figure}
	%
	As a further test, we solve the Schr{\"o}dinger equation with a
	vanishing Dirichlet bc (solid lines in
	Figs.~\ref{fig:spatial_cutoff_vs_HO_gauss} and
	\ref{fig:spatial_cutoff_vs_HO_square_well}), and compare to the
	energies obtained from the HO truncations (crosses).  The finite
	oscillator basis energies are well approximated by a Dirichlet
	bc with a mapping from the oscillator $\hbar\Omega$
	and $\Nmax$ to an equivalent length given by $L_2$.  Note that for large
	$\Nmax$, the differences between $L_0$, $\LA$ and $L_2$ may be smaller
	than other uncertainties involved in nuclear calculations, but for
	practical calculations one will want to use small $\Nmax$ results, where
	these considerations are very relevant.

	These results from model calculations are consistent with those from
	realistic potentials applied to the deuteron.  To illustrate this, we
	use the N$^3$LO 500\,MeV potential of Entem and
	Machleidt~\cite{Entem:2003ft}.  We generate results in an HO basis
	with \hw\ ranging from $1$ to $100\,\mbox{MeV}$ and $\Nmax$ from $4$ to
	$100$ (in steps of 4 to avoid HO artifacts for the
	deuteron~\cite{Bogner:2007rx}).  We then restrict the data to where UV
	corrections are negligible (see Section~\ref{sec:UV_story}).
	%
	\begin{figure}[h]
	\centering
  \includegraphics[width=0.6\textwidth]
	{Extrapolation/spatialcutoffdeutrn}
  \caption{Ground-state energies versus $L_0$ (top),
  	$\LA$ (middle), and $L_2$ (bottom) for the Entem-Machleidt 500\,MeV
  	N$^3$LO potential~\cite{Entem:2003ft}.  The horizontal dotted lines
  	mark the exact energy $\Einf = -2.2246\,\mbox{MeV}$. }
  \label{fig:spatial_cutoff_vs_HO_deuteron}
  \end{figure}
	%
	Figure~\ref{fig:spatial_cutoff_vs_HO_deuteron} shows that the
	criterion of a continuous curve with the smallest spread of points
	clearly favors $L_2$.

	\medskip
	\subsubsection{Analytical derivation of $L_2$}

	In the asymptotic region (this is the region where IR cutoff is imposed), the
	potential is negligible and the only relevant
	part of the Hamiltonian is the kinetic energy or the $p^2$ operator.
	In what follows, we analytically compute the smallest eigenvalue
	$\kappa^2_{\rm min}$ of $p^2$ in a finite oscillator basis and will
	see that $\kappa_{\rm min} = \pi/L_2$. In the remainder of this
	Subsection, we set the oscillator length to one. We focus on $s$-waves
	and thus consider wave functions that are regular at the origin, i.e.
	the radial wave functions are identical to the odd wave functions of
	the one-dimensional harmonic oscillator.

	The localized eigenfunction of the operator $p^2$ with smallest
	eigenvalue $\kappa^2$ is
	%
	\bea
	\label{eq:eigen_l0}
	\psi_{\kappa}(r) = \left\{\begin{array}{ll}
	\sin{\kappa r}\ , & 0 \le r \le {\pi\over\kappa}\\
	0 \ , & r > {\pi\over\kappa}
	\end{array}\right. \; .
	\eea
	%
	We employ the $s$-wave oscillator functions
	%
	\bea
	\varphi_{2n+1}(r) &=&(-1)^n \sqrt{2 n!\over\Gamma(n+3/2)} r
	L^{1\over 2}_n\left(r^2\right)
	e^{-{r^2\over 2}} \nonumber\\
	&=&\left(\pi^{1\over 2} 2^{2n} (2n+1)!\right)^{-1/2}H_{2n+1}(r)
	e^{-{r^2\over 2}} \; ,
	%\label{eq:HO_S_wave_written_out}
	\eea
	%
	with energy $E=(2n +3/2)\hbar\Omega$. Here, $L_n^{1/2}$ denotes the
	Laguerre polynomial, and it is convenient to rewrite this function in
	terms of the Hermite polynomial $H_n$. We expand the
	eigenfunction in Eq.~\eqref{eq:eigen_l0} as
	%
	\beq
	\label{expand}
	\psi_{\kappa}(r) = \sum_{n=0}^\infty c_{2n+1}(\kappa)\varphi_{2n+1}(r) \; .
	\eeq
	%
	Before we turn to the computation of the expansion coefficients
	$c_{2n+1}(\kappa)$, we consider the eigenvalue problem for the
	operator $p^2$.  We have
	%
	\beq
	p^2 = a^\dagger a +{1\over 2} -{1\over 2}
	\left(a^2 +\left(a^\dagger\right)^2\right) \;,
	\eeq
	%
	where $a$ and $a^\dagger$ denote the annihilation and creation
	operator for the one-dimensional harmonic oscillator, respectively.
	The matrix of $p^2$ is tridiagonal in the oscillator basis.
	For the matrix representation, we order the basis states as
	$(\varphi_1, \varphi_3, \varphi_5, \ldots)$. Thus, the eigenvalue
	problem $p^2-\kappa^2=0$ becomes a set of rows of coupled linear
	equations. In an infinite basis, the eigenvector $(c_1(\kappa),
	c_3(\kappa), c_5(\kappa), \ldots )$ identically satisfies every row of
	the eigenvalue problem for any value of $\kappa$. In a finite basis
	$(\varphi_1, \varphi_3, \varphi_5,\ldots \varphi_{2n+1})$, however,
	the last row of the eigenvalue problem
	%
	\beq
	\label{quant}
	\left(2n+3/2 -\kappa^2\right) c_{2n+1}(\kappa) =
	 {1\over
	  2}\sqrt{2n}\sqrt{2 n+1} \, c_{2n-1}(\kappa) \; ,
	\eeq
	%
	can only be fulfilled for certain values of $\kappa$, and this is the
	quantization condition. To solve this eigenvalue problem we need
	expressions for the expansion coefficients $c_{2n+1}(\kappa)$
	for $n\gg 1$. Those can be derived analytically as follows.


	We rewrite the eigenfunction in Eq.~\eqref{eq:eigen_l0} as a Fourier transform
	%
	\beq
	\psi_{\kappa} (r) = \sqrt{2\over \pi} \int\limits_0^\infty dk
	\tilde{\psi}_{\kappa}(k) \sin kr \; ,
	\eeq
	%
	and expand the sine function in terms of oscillator functions as
	%
	\beq
	\sin kr = \sqrt{\pi\over 2} \sum_{n=0}^\infty (-1)^n \varphi_{2n+1}(r)
	\varphi_{2n+1}(k) \; .
	\eeq
	%
	Thus, the expansion coefficients in Eq.~\eqref{expand} are given in
	terms of the Fourier transform $\tilde{\psi}_\kappa(k)$ as
	%
	\beq
	\label{integ}
	c_{2n+1}(\kappa) = (-1)^n\int\limits_0^\infty dk\, \tilde{\psi}_{\kappa}(k)
	\varphi_{2n+1}(k) \; .
	\eeq
	%
	So far, all manipulations have been exact.  We need an expression for
	$c_{2n+1}(\kappa)$ for $n\gg 1$ and use the asymptotic expansion
	%
	\beq
	\label{approxwf}
	\varphi_{2n+1}(k) \approx {(-1)^n\sqrt{2}\over \pi^{1/4}}
	{(2n-1)!!\over \sqrt{(2n)!}}
	\sin (\sqrt{4n+3}k) \; ,
	\eeq
	%
	which is valid for $|k|\ll \sqrt{2n}$, see~\cite{gradshteyn}.
	%
	Using this approximation, one finds (making use of Fourier transforms)
	%
	\bea
	\label{phi}
	c_{2n+1}(\kappa) &\approx& \pi^{1/4} {(2n-1)!!\over \sqrt{(2n)!}}
	\psi_{\kappa}(\sqrt{4n+3}) \nonumber\\
	&=&\pi^{1/4}{(2n-1)!!\over \sqrt{(2n)!}}
	\sin (\sqrt{4n+3}\kappa) \; ,
	\eea
	%
	with $\kappa \le \pi/\sqrt{4n+3}$ due to Eq.~\eqref{eq:eigen_l0}.

	Let us return to the solution of the quantization
	condition in Eq.~\eqref{quant}.  We make the ansatz
	%
	\beq
	\kappa = {\pi\over\sqrt{4n+3+2\Delta}} \; ,
	\eeq
	%
	and must assume that $\Delta > 0$.  This ansatz is well motivated, since the
	naive semiclassical estimate $\kappa = \pi/L_0$ yields $\Delta=0$.  We
	insert the expansion coefficients of Eq.~\eqref{phi} into Eq.~\eqref{quant}
	and consider its leading-order approximation
	for $n\gg 1$ and $n\gg \Delta$. This yields
	%
	\beq
	\Delta = 2
	\eeq
	%
	as the solution. Recalling that a truncation of the basis at
	$\varphi_{2n+1}$ corresponds to the maximum energy
	$E=(2n+3/2)\hbar\Omega$, we see that we must identify
	$\Nmax \equiv N = 2n$. Thus,
	$\kappa_{\rm min} = \pi/L_2$ is the lowest momentum in a finite
	oscillator basis with $n \gg 1$ basis states (and not $1/b$ as stated in
	Ref.~\cite{Coon:2012ab}).  It is clear from its
	very definition that $\pi/L_2$ is also (a very precise approximation of)
	the infrared cutoff in a finite	oscillator basis, and that $L_2$ (and not $b$
	as stated in Refs.~\cite{Stetcu:2006ey,Stetcu:2007ms}) is the radial extent
	of the oscillator basis and the analog to the extent of the lattice in the
	lattice computations \cite{Luscher:1985dn}.

	The derivation of our key result $\kappa_{\rm min}=\pi/L_2$ is based
	on the assumption that the number of shells $N$ fulfills $N\gg 1$.
	Table~\ref{tab:L0_L2_k_min_comparison} shows a comparison of
	numerical results for
	$\kappa_{\rm min}$ in different model spaces. We see that
	$\pi/L_2$ is a very good approximation already for $N=2$, with a
	deviation of about 1\%.

	\begin{table}[ht]
	\centering
	\begin{tabular}{|c|c|c|c|}\hline
	$N$ & $\kappa_{\rm min}$ & $\pi/L_2$ & $\pi/L_0$ \\\hline
	   0 & 1.2247 & 1.1874 & 1.8138\\
	   2 & 0.9586 & 0.9472 & 1.1874\\
	   4 & 0.8163 & 0.8112 & 0.9472\\
	   6 & 0.7236 & 0.7207 & 0.8112\\
	   8 & 0.6568 & 0.6551 & 0.7207\\
	  10 & 0.6058 & 0.6046 & 0.6551\\
	  12 & 0.5651 & 0.5642 & 0.6046\\
	  14 & 0.5316 & 0.5310 & 0.5642\\
	  16 & 0.5035 & 0.5031 & 0.5310\\
	  18 & 0.4795 & 0.4791 & 0.5031\\
	  20 & 0.4585 & 0.4582 & 0.4791\\\hline
	\end{tabular}
	\caption{Comparison between the lowest momentum $\kappa_{\rm min}$, $\pi/L_2$,
	 and $\pi/L_0$ for model spaces with up to $N$ oscillator quanta.}
	\label{tab:L0_L2_k_min_comparison}
	\end{table}

	Note that this approach can be generalized to other localized
	bases.  The (numerical) computation of the lowest eigenvalue of the momentum
	operator $p^2$ yields the box size $L$ corresponding to the employed Hilbert
	space.

	\medskip
	\subsubsection{EFT-like approach}

	We mentioned that the relevant operator for IR truncation is $p^2$.  To
	get a better understanding of the correspondence between the HO truncation
	and a hard wall at $L_2$, let's compare the spectrum of $p^2$ in the two
	cases.
	%
	\begin{figure}[h]
	\centering
	\includegraphics[width=0.6\textwidth]
	{Extrapolation/momentum_eigenfunctions1.pdf}
	\caption{Eigenfunctions of $p^2$ in the truncated HO basis compared to those
	  in a box of size $L_2$. }
	\label{fig:mom_eigen_fns}
	\end{figure}
	%
	In Fig.~\ref{fig:mom_eigen_fns}, we compare the low-lying eigenfunctions of
	$p^2$ in the truncated HO basis to the eigenfunctions in a box of size $L_2$.
	As we will see later, the asymptotic or near the wall difference between the
	two eigenfunctions are high-momentum effects irrelevant for the
	long-wavelength physics of the bound states.

	Another way to look at this is to compute the number $M(k)$ of ($s$-wave)
	states up to a momentum $k$.  We find
	\bea
	  M(k)&=& {\rm Tr} \left[\Theta\left(\hbar^2k^2-p^2\right)
	   \Theta\left(E-{p^2\over 2m}-{m\over 2}\Omega^2 r^2\right) \right]
	  \nonumber\\
	  &\approx&{1\over 2\pi\hbar} \int\limits_{-\hbar k}^{\hbar k}\! dp
	  \int\limits_0^\infty\! dr \,
	   \Theta\left(\hbar^2k^2-p^2\right)
	   \Theta\left(E-{p^2\over 2m}-{m\over 2}\Omega^2 r^2\right)
	   \; .
	\eea
	%
	Here, we apply the semiclassical approximation and write the trace
	as a phase-space integral.  We assume $\hbar^2k^2/(2m)\le
	E$, perform the integrations and use $E/(\hbar\Omega)= N+3/2$
	\footnote{For the sake of brevity we replace $\Nmax$ by $N$}. This
	yields
	%
	\bea
	  M(k) = {bk\over 2\pi}\sqrt{2N+3-b^2k^2}
	   +{N+3/2\over \pi}\arcsin{bk\over\sqrt{2N+3}}\;,
	  \label{Mstair}
	\eea
	%
	where $b$ is the oscillator length.
	Figure~\ref{fig:staircase} shows a comparison between the quantum
	mechanical staircase function and the semiclassical
	estimate of Eq.~\eqref{Mstair} for $N=32$.  For sufficiently small values of
	$kb\ll\sqrt{2N}$, the number of $s$-wave momentum eigenstates grows
	linearly, and inspection of Eq.~\eqref{Mstair} shows that the slope at
	the origin is $L_0/\pi$ semiclassically.  The linear growth of the
	number of eigenstates of $p^2$ with $k$ clearly demonstrate that --- at
	not too large values of $kb$ --- the spectrum of $p^2$ in the oscillator
	basis is similar to the spectrum of $p^2$ in a spherical
	box.
  %
	\begin{figure}[h]
	\centering
	\includegraphics[width=0.6\textwidth]{Extrapolation/staircase_N32_v2}
	\caption{The staircase function of the $s$ states of
	  the operator $p^2$ in a finite oscillator basis with $N=32$ (black)
	  compared to its semiclassical estimate (smooth red curve). $M(k)$
	  denotes the number of states of the operator $p^2$ with eigenvalues
	  $p^2\le\hbar^2 k^2$.}
	\label{fig:staircase}
	\end{figure}

	As a final example of the correspondence between the HO truncation and the
	hard wall at $L_2$, we look at the ground state wave functions of a square
	well in the two bases in Fig.~\ref{fig:HO_q_and_rspace_sqwell}.
	%
	\begin{figure}[h]
	\centering
	\includegraphics[width=0.65\textwidth]
	{Extrapolation/HO_q_and_rspace_sqwell_V0_4p0_Nmax8_hw18_v2}
	\caption{Ground-state wave functions for a square well potential of depth
	  $V_0=4$ (see Eq.~\eqref{eq:Vsw}; lengths are in units of $R$ and energies
		in units of $1/R^2$ with $\hbar^2/\mu = 1$) from solving the Schr\"odinger
	  equation with a truncated harmonic oscillator basis with $\hbar\Omega = 18$
	  and $N = 8$ (dashed) and with a
	  Dirichlet bc at $r=L_2$ given from Eq.~\eqref{eq:L2_def}
		(solid).  The coordinate-space radial wave functions in a) exhibit a
		difference at $r$ near 1.5, but the Fourier-transformed wave functions in
		b) are in close agreement at low $k$, showing that the differences are
		high-momentum modes.}
	\label{fig:HO_q_and_rspace_sqwell}
	\end{figure}
	%
	The binding momentum in this case is $1.7$ (in units of $1/R$).  The
	Fourier-transformed wave functions differ at much larger momentum and this
	difference is irrelevant for the long-wavelength physics of bound states.
	Thus the use of Dirichlet bc to take into account the HO
	truncation is similar in spirit to the use of contact interactions to
	describe the effect of unknown short-ranged forces on long-wavelength probes.


	\subsection{Cashing in on the hard wall correspondence}
	\label{subsec:make_cash}

	We have so far focused on establishing how HO truncation is analogous to
	putting the system in a spherical box of a specified radius.  Now let's see
	how this correspondence helps us in getting the exact energy $\Einf$.

	\medskip
	\subsubsection{Linear energy method}

	Our first approximation to the IR correction
	is based on what is known in quantum chemistry
	as the linear energy method~\cite{Djajaputra:2000aa}.  Given a
	hard-wall bc at $r=L$ beyond the range of the
	potential, we write the energy compared to that for $L=\infty$ as
	%
	\beq
	 E_L = E_{\infty}+\Delta E_L
	 \;.
	\eeq
	%
	We seek an estimate for $\Delta E_L$, which is assumed to be small,
	based on an expansion of the wave function in $\Delta E_L$.  Let
	$u_E(r)$ be a radial solution with regular bc at the
	origin and energy $E$.  For convenience in using standard quantum
	scattering formalism below, we choose the normalization corresponding
	to what is called the ``regular solution'' in
	Ref.~\cite{taylor2006scattering}, which means that $u_E(0) = 0$ and
	the slope at the origin is unity for all $E$.  We denote the
	particular solutions $u_{E_L}(r)\equiv u_L(r)$ and $u_{\Einf}(r)
	\equiv u_\infty (r)$. Then there is a smooth expansion of $u_E$ about
	$E=\Einf$ at fixed $r$, so we approximate~\cite{Djajaputra:2000aa}
	%
	\beq
	  u_L(r)\approx u_\infty (r) + \Delta E_L
	  \left.\frac{du_E(r)}{dE}\right|_{E_{\infty}}
	  + \mathcal{O}(\Delta E_L^2) \;,
	  \label{eq:linear_energy_approx}
	\eeq
	%
	for $r\leq L$.
	%
	By evaluating Eq.~\eqref{eq:linear_energy_approx} at $r=L$ with the
	bc $u_L(L)=0$, we find
	%
	\beq
	  \Delta E_L \approx -u_\infty(L) \left(\left.\frac{d u_E(L)}
	  {dE}\right|_{\Einf}\right)^{-1}
	  \;,
	  \label{eq:Delta_EL}
	\eeq
	%
	which is the estimate for the IR correction.

	\begin{figure}[h]
	\centering
	\includegraphics[width=0.6\textwidth]{Extrapolation/linear_approx_v2}
	\caption{Testing the linear energy approximation
	  Eq.~\eqref{eq:linear_energy_approx} for (a) deep ($V_0=10$) and (b)
	  shallow ($V_0=2$) Gaussian potential well Eq.~\eqref{eq:Vg}
		($\hbar = \mu = R=1$).
	  The solid lines are the exact solutions $u_L(r)$ for energies $-3.5$
	  and $-0.020$, respectively, whose zero crossings determine the
	  corresponding values for $L$.}
	\label{fig:linear_energy_approx}
	\end{figure}

	We can check the accuracy of the linear energy
	approximation~(Eq.~\eqref{eq:linear_energy_approx}) by numerically solving
	the Schr\"odinger equation with a specified energy.  This determines
	$L$ as the radius at which the resulting wave function vanishes. Then
	we compare this wave function for $r \leq L$ to the right side of
	Eq.~\eqref{eq:linear_energy_approx}, with the derivative calculated
	numerically.  Figure~\ref{fig:linear_energy_approx} shows
	representative examples for a deep and shallow Gaussian potential.  In
	these examples and other cases, the approximation to the wave function
	is good, particularly in the interior.  The estimates for $\Delta E_L$
	using the right side of Eq.~\eqref{eq:Delta_EL} are within a few to
	ten percent: 0.68 versus 0.70 and 0.050 versus 0.055 for the two
	cases.

	The good approximation to the wave function suggests that for the
	calculation of other observables the linear energy approximation will
	be useful.  For observables most sensitive to the long distance
	(outer) part of the wave function, such as the radius, this has
	already been shown to be true~\cite{Furnstahl:2012qg}.  But the good
	approximation to the wave function at small $r$ means that corrections
	for short-range observables should also be controlled, with the
	dominant contribution in an extrapolation formula coming from the
	normalization.

	Next we derive an expression for the derivative in
	Eq.~\eqref{eq:Delta_EL}.  To start with we assume we have a single
	partial-wave channel.  For general $E < 0$, the asymptotic form of the
	radial wave function for $r$ greater than the range of the potential is
	%
	\beq
	  u_E(r)\overset{r \gg R}\longrightarrow A_E(e^{-k_E r}+\alpha_E e^{+k_E r})
	  \;,
	  \label{eq:asymptotic_form_uE}
	\eeq
	%
	with $u_\infty(r)\overset{r \gg R}\longrightarrow \Ainf e^{-\kinf r}$
	for $E=\Einf$.  We take the derivative of
	Eq.~\eqref{eq:asymptotic_form_uE} with respect to energy, evaluate at
	$E=\Einf$ using $\alpha_{\Einf}=0$ and $dk_E/dE = -\mu/(\hbar^2 k_E)$,
	to find
	%
	\bea
	    \left.\frac{d u_E(r)}{d E}\right|_{\Einf} &=&
	    \Ainf \left.\frac{d \alpha_E}{d E}\right|_{\Einf} e^{+\kinf r}
	    +
	    \Ainf \frac{\mu}{\hbar^2}\frac{r}{\kinf} e^{-\kinf r}
	    +
	    \left.\frac{d A_E}{d E}\right|_{\Einf} e^{-\kinf r}
	    \;.
	  \label{eq:entire_correction}
	\eea
	%
	We now evaluate at $r=L$ and anticipate that the $e^{+\kinf L}$ term
	dominates:
	%
	\beq
	  \left.\frac{d u_E(L)}{d E}\right|_{\Einf}
	  \approx \Ainf \left.\frac{d\alpha_E}{d E}\right|_{\Einf} e^{+\kinf L}
	   + \mathcal{O}(e^{-\kinf L})
	   \;.
	  \label{eq:substitute_this}
	\eeq
	%
	Substituting Eq.~\eqref{eq:substitute_this} into
	Eq.~\eqref{eq:Delta_EL}, we obtain
	%
	\beq \Delta E_L \approx -\left[ \left.\frac{d \alpha_E}{d
	      E}\right|_{\Einf}\right]^{-1} e^{-2 \kinf L} + \mathcal{O}(e^{-4
	  \kinf L}) \;.
	\eeq
	Note that this result is independent of	the normalization of the
	wave function.

	To calculate the derivative explicitly, we turn to scattering theory,
	following the notation and discussion in
	Ref.~\cite{taylor2006scattering}.  In particular, the asymptotic form
	of the regular scattering wave function $\phi_{l,k}$ for orbital
	angular momentum $l$ and for positive energy $E \equiv \hbar^2 k^2/2\mu$
	is given in terms of the Jost function
	$\Jost_l(k)$~\cite{taylor2006scattering},
	%
	\beq
	  \phi_{l,k}(r) \longrightarrow
	    \frac{i}{2}[\Jost_l(k)\hat{h}_l ^{-}(k r)-\Jost_l(-k)\hat{h}_l^{+}(k r)]
	    \;,
	    \label{eq:phi_asymp}
	\eeq
	%
	where the $\hat{h}_l^{\pm}$ functions (related to Hankel functions)
	behave asymptotically as
	%
	\beq
	\hat{h}_l ^{\pm}(k r)
	    \overset{r\rightarrow\infty}\longrightarrow e^{\pm i (k r - l \pi/2)}
	    \;.
	\eeq
	%
	The ratio of the Jost functions appearing in Eq.~\eqref{eq:phi_asymp}
	gives the partial wave $S$-matrix $s_l(k)$:
	%
	\beq
	   s_l(k) = \frac{\Jost_l(-k)}{\Jost_l(+k)}
	   \;,
	   \label{eq:s_l}
	\eeq
	%
	which is in turn related to the partial-wave scattering amplitude
	$f_l(k)$ by
	%
	\beq
	  f_l(k) = \frac{s_l(k) - 1}{2i k}
	  \;.
	  \label{eq:f_l}
	\eeq
	%
	We will restrict ourselves to $l=0$ for simplicity; the generalization
	to higher $l$ is straightforward and will be considered later.

	To apply Eq.~\eqref{eq:phi_asymp} to negative energies, we
	analytically continue from real to (positive) imaginary $k$.  So,
	%
	\bea
	   \phi_{0,ik_E}(r)
	       &\overset{r \gg R}\longrightarrow&
	         \frac{i}{2}\bigl(
		    \Jost_0(ik_E)e^{k_E r}-\Jost_0(-ik_E) e^{-k_E r} \bigr)
		 \nonumber \\
	       &=&
	       -\frac{i}{2}\Jost_0(-ik_E) \bigl(
	           e^{-k_E r} - \frac{\Jost_0(-ik_E)}{\Jost_0(ik_E)} e^{k_E r}
	         \bigr)
	       	\;,
	  \label{eq:asymptotic_form_phiE}
	\eea
	%
	where $R$ is the range of the potential.  Upon comparing to
	Eq.~\eqref{eq:asymptotic_form_uE} we conclude that
	%
	\beq
	  \alpha_E = -\frac{\Jost_0(ik_E)}{\Jost_0(-ik_E)}
	   = -\frac{1}{s_0(ik_E)} \;.
	  \label{eq:alphaE_and_S}
	\eeq
	%
	Note that Eq.~\eqref{eq:alphaE_and_S} is consistent with the
	bound-state limit of Eq.~\eqref{eq:asymptotic_form_uE}: at a bound
	state where $\Einf = -\hbar^2 \kinf^2/2\mu$ there is a simple pole in the $S$
	matrix, which means $\alpha_E = 0$ as expected (no exponentially
	rising piece).

	From Ref.~\cite{taylor2006scattering} we learn that the residue as a
	function of $E$ of the partial wave amplitude $f_l(E)$ at the
	bound-state pole is $\displaystyle (-1)^{l+1} \ANC^2 \hbar^2/2 \mu$,
	where $\ANC$ is	the asymptotic normalization coefficient (ANC).  The ANC is
	defined by the large-$r$ behavior of the
	\emph{normalized} bound-state wave function:
	%
	\beq
	  u_{\rm norm}(r)\overset{r\gg R}\longrightarrow \ANC e^{-\kinf r}
	  \;.
	  \label{eq:definition_of_ANC}
	\eeq
	%
	Thus, near the bound-state pole (with $E = \hbar^2 k^2/2\mu$),
	%
	\beq
	  f_0(k)  \approx  \frac{- \hbar^2\ANC^2}{2\mu (E-\Einf)}
	      =  \frac{-\ANC^2}{k^2 + \kinf^2} \;.
	\eeq
	%
	or, using Eqs.~\eqref{eq:f_l} and \eqref{eq:alphaE_and_S},
	%
	\beq
	  \alpha_E(k) \approx -\frac{k^2+\kinf^2}{k^2+\kinf^2-2 i k \ANC^2}
	  \;.
	  \label{eq:alphaE_k}
	\eeq
	%
	Now,
	%
	\beq
	  \left.\frac{d \alpha_E}{d E}
	  \right|_{\Einf}=\frac{d\alpha_E/dk\vert_{k=i\kinf}}
	  {dE/dk|_{k=i\kinf}}
	  \;,
	\eeq
	%
	so using Eq.~\eqref{eq:alphaE_k} we find
	%
	\beq
	  \left.\frac{d\alpha_E}{d k}\right|_{k=i\kinf}=\frac{-i}{\ANC^2}
	  \;,
	\eeq
	%
	and therefore
	%
	\beq
	  \left.\frac{d\alpha_E}{dE}\right|_{\Einf}=\frac{-\mu}{\hbar^2 \kinf \ANC^2}
	  \;.
	\eeq
	%
	Putting it all together, we have
	%
	\beq
	  \Delta E_L = \frac{\hbar^2 \kinf \ANC^2}{\mu} e^{-2 \kinf L}
	    + \mathcal{O}(e^{-4 \kinf L})
	    \;.
	    \label{eq:complete_IR_scaling}
	\eeq
	%
	Equation~\eqref{eq:complete_IR_scaling} matches the result
	\beq
	E(L) = \Einf + A e^{-2 \kinf L} + \mathcal{O}(e^{-4 \kinf L})
	\label{eq:exp_L_extrapolation}
	\eeq
	in \cite{Furnstahl2012}, but now we have identified
	$\displaystyle A = \hbar^2\kinf\ANC^2/\mu$.

	In \cite{More:2013rma}, we advocated including second term in
	Eq.~\eqref{eq:entire_correction} for weakly bound states (small $\kinf$
	makes the term $\displaystyle \Ainf \frac{\mu}{\hbar^2}\frac{r}{\kinf}
	e^{-\kinf r}$
	non-negligible).  Including this term was also seen to give better prediction
	for weakly bound states like deuteron.  As we pointed out in
	\cite{Furnstahl:2013vda}, a better way to arrange an expression for
	$\Delta E$ is to have a systematic expansion in powers of $e^{-2 \kinf L}$.
 	Keeping the second term in Eq.~\eqref{eq:entire_correction}, generates terms
	in higher powers of $\mathcal{O}(e^{-2 \kinf L})$.  However, these higher
	order terms also arise from the $\mathcal{O}(\Delta E_L ^2)$ term in
	Eq.~\eqref{eq:linear_energy_approx} and to be consistent
	we need to take in account contributions up to a given order from both the
	sources (i.e., Eqs.~\eqref{eq:linear_energy_approx} and
	\eqref{eq:entire_correction}).  As we will see below, relating $k_L$
	($k_L$ is the binding momentum when we have a hard wall at length $L$)
	directly to the $S$-matrix allows us to transparently obtain systematic
	expansion for $\Delta E$ in powers of $e^{-2 \kinf L}$.

	\medskip
	\subsubsection{The $S$-matrix way}

	In \cite{Furnstahl:2013vda}, we returned to Eq.~\eqref{eq:asymptotic_form_uE}
	and noted that the bc uniquely fixed the coefficient
	$\alpha_E$.  We need $u_E(r = L) = 0$ which fixes
	\beq
	\alpha_E = -e^{-2 k_E L}\;.
	\label{eq:alpha_def}
	\eeq
	To make the $L$ dependence explicit, we modify the notation and let
	$k_L \equiv k_E$.  Comparing Eqs.~\eqref{eq:alpha_def} and
	\eqref{eq:alphaE_and_S}, we have
	\beq
	e^{-2 k_L L} = \left[s_0(i k_L)\right]^{-1} \;.
	\label{eq:basiceq}
	\eeq
	We then use appropriate parametrization for $s_0$ valid in the complex $k$
	region and solve the transcendental
	equation~\eqref{eq:basiceq} for $k_L$ and thereby find $E_L$.

	If the potential has no long-range part that introduces a singularity
	in the complex $k$ plane nearer to the origin than the bound-state
	pole (which is the case, for example, for the deuteron when we assume that
	the longest-ranged interaction is from pion exchange), then the
	continuation of the positive-energy partial-wave S-matrix (i.e., the
	phase shifts) to the pole should be unique.  Because $|k_L| <
	|\kinf|$, $s_0(i k_L)$ and therefore $k_L$ and the energy shift $E_L$
	should be determined solely by observables.

	The leading term in an expansion of $k_L - \kinf$ using
	Eq.~\eqref{eq:basiceq} comes from the bound-state pole, at which $s_0$
	behaves like~\cite{newton2002scattering}
	%
	\beq
	  s_0(k) \approx \frac{-i\ANC^2}{k-i\kinf}
	  \;.
	  \label{eq:purepole}
	\eeq
	%
	Note that $\ANC$ here is
	the asymptotic normalization coefficient (ANC) defined in
	Eq.~\eqref{eq:definition_of_ANC}.
	%
	Substituting Eq.~\eqref{eq:purepole} into Eq.~\eqref{eq:basiceq} yields
	%
	\beq
	  k_L - \kinf \approx -\ANC^2 e^{-2k_L L} \approx -\ANC^2 e^{-2\kinf L}
	  \;.
	  \label{eq:kLatLO}
	\eeq
	%
	This is the leading-order (LO) result for $k_L$ obtained in
	Eq.~\eqref{eq:complete_IR_scaling}.  Note that in
	Eq.~\eqref{eq:complete_IR_scaling}, $\displaystyle \Delta E_L \equiv
	E_L - \Einf = \kinf^2/2 - k_L^2/2$.  We set $\displaystyle \hbar^2/ \mu = 1$.
	The notation $\kinf$ for the exact binding momentum make sense in this
	context, because in the exact case, the hard wall is at $L = \infty$.

	Iterations of the intermediate equation  in \eqref{eq:kLatLO} motivate the
	NLO parameterization of $k_L$ as
	%
	\beq
	  k_L = \kinf + A e^{-2 \kinf L} + (B L + C) e^{-4 \kinf L}  +
		\mathcal{O}(e^{-6 \kinf L})
	  \;,
	  \label{eq:kLexpansion}
	\eeq
	%
	with $A = -\ANC^2$.  In general we can substitute this expansion into
	Eq.~\eqref{eq:basiceq} using an parametrized form of the S-matrix,
	then expand in powers of $e^{-2 \kinf L}$ and equate $e^{-2 \kinf L}$,
	$L e^{-4 \kinf L}$, and $e^{-4 \kinf L}$ terms on both sides of the
	equation.  However, while both $A$ and $B$ are uniquely determined by
	the pole in $s_0(k)$ at $k=i\kinf$, $C$ is only determined unambiguously if
	$s_0(k)$ is consistently parameterized away from the pole.  For
	example, the two parametrizations
	%
	\beq
	\label{eq:S0_1}
	    s_0(i k_L) \approx \frac{\kinf^2 - k_L^2 + 2 k_L \ANC^2}{\kinf^2 - k_L^2}
	\eeq
	%
	and
	%
	\beq
	  s_0(k) \approx \frac{-\ANC^2}{2\kinf}\,\frac{k+i\kinf}{k- i\kinf}
	  \label{eq:s0_Newton}
	\eeq
	%
	yield different results for $C$. The first
	parametrization~(Eq.~\eqref{eq:S0_1}) is based on a particular form for the
	partial-wave scattering amplitude near the
	pole~\cite{taylor2006scattering}, and was employed in
	Ref.~\cite{More:2013rma}.  The second paramerization
	(Eq.~\eqref{eq:s0_Newton}) correctly incorporates that the S-matrix also has
	a zero at $-i\kinf$~\cite{newton2002scattering}.  In neither case,
	however, do we have a sufficiently general parametrization that allows
	us to unambiguously determine $C$.

	For the complete NLO energy correction, we start from the
	general expression for the S-matrix
	%
	\beq
	  s_0(k) = \frac{k\,\cot\delta_0(k) + ik}{k\,\cot\delta_0(k) -ik}
	  \;,
	  \label{eq:s0_exact}
	\eeq
	%
	and use an effective range expansion to substitute for $k\,\cot
	\delta_0(k)$.  In particular, we use an expansion
	around the bound-state pole rather than about zero energy,
	namely~\cite{wu2011scattering,Phillips:1999hh},
	%
	 \beq
	  k\,\cot\delta_0(k) = -\kinf + \frac12 \rho_d(k^2 + \kinf^2)
	     + w_2(k^2+\kinf^2)^2 +  \cdots
	     \;.
	     \label{eq:eff_range_kinf}
	\eeq
	%
	To match the residue at the S-matrix pole as in Eq.~\eqref{eq:purepole},
	we identify
	%
	\beq
	 \rho_d = \frac{1}{\kinf} - \frac{2}{\ANC^2}
	  \;.
	  \label{eq:rhoD_gamma_rel}
	\eeq
	%
	$w_2$ is a low-energy observable like $\ANC$ and $\kinf$.
	Now we substitute Eq.~\eqref{eq:eff_range_kinf} into Eq.~\eqref{eq:s0_exact}
	and use Eq.~\eqref{eq:kLexpansion} to expand both sides of
	Eq.~\eqref{eq:basiceq}, equating terms with equal powers of
	$e^{-2\kinf L}$ and $L$.  The resulting expansion for the binding
	momentum to NLO is
	%
	\begin{align}
    [k_L]_{\rm NLO} &=  \kinf - \ANC^2 e^{-2\kinf L}  - 2  L \ANC^4
		e^{-4\kinf L}
    \nonumber  \\
    &\null - \ANC^2 \left(1 - \frac{\ANC^2}{2\kinf} - \frac{\ANC^4}{4\kinf^2}
		+ 2 \kinf w_2 \ANC^4 \right) e^{-4\kinf L} \;.
    \label{eq:complete_k_correction_NLO}
  \end{align}
	%
	Using $\Delta E_L \equiv E_L - \Einf = \kinf^2/2 - k_L^2/2$,
	the correction for the energy due to finite $L$ is
	%
	\begin{align}
  	[\Delta E_L]_{\rm NLO} &=  \kinf \ANC^2 e^{-2\kinf L}
  	+ 2\kinf L\ANC^4 e^{-4\kinf L}
    \nonumber \\
  	& \null\quad +
    \kinf\ANC^2 \Bigl( 1-\frac{\ANC^2}{\kinf}-\frac{\ANC^4}{4\kinf^2}
    + 2\kinf w_2 \ANC^4  \Bigr)  e^{-4\kinf L} \;.
    \label{eq:complete_E_correction_NLO}
	\end{align}
	%
	In what follows we use LO to refer to the first term in this expansion and
	L-NLO to
	refer to the first two terms (the second term should dominate the full
	NLO expression when $\kinf L$ is large).  We also note that
	higher-order terms in Eq.~\eqref{eq:eff_range_kinf} (e.g., terms
	proportional to $(k^2+\kinf^2)^3$ and higher powers) do not affect the
	binding momentum or energy predictions
	Eqs.~\eqref{eq:complete_k_correction_NLO} and
	\eqref{eq:complete_E_correction_NLO} at NLO.

	As a special case, let us consider the zero-range limit of a
	potential. In this case $\rho_d = w_2 = 0$, $\ANC^2 = 2\kinf$, and
	%
	\beq
	   [s_0(ik_L)]^{-1} = \frac{\kinf-k_L}{\kinf+k_L}\;.
	\eeq
	%
	The expansion for $k_L$ in a form similar to Eq.~\eqref{eq:kLexpansion}
	can be extended to arbitrary order using Eq.~\eqref{eq:basiceq}.

	We note finally that the leading corrections beyond NLO scale as $L^2
	e^{-6\kinf L}$. While we do not pursue a derivation of such high-order
	corrections here, the knowledge of the leading form is useful in some of
	the error analysis we present in Subsec.~\ref{subsec:pudding_proof}.

	\medskip
	\subsubsection{Differential method}

	Because we seek the change in energy with respect to a cutoff, it is
	natural to formulate the problem in the spirit of renormalization
	group methods by seeking a flow equation for the bound-state energy as
	a function of $L$.  Such an approach is already documented in the
	literature, for example in Refs.~\cite{Arteca1984} and
	\cite{Fernandez1981}, and it provides us with an alternative method that
	does not directly reference the S-matrix.  The basic equation is
	%
	\beq
	  \frac{\partial E_L}{\partial L} = -\frac12 \frac{|u'_L(L)|^2}
		{\int_0^L |u_L(r)|^2\, dr}
	  \;.
	  \label{eq:dEdL}
	\eeq
	%
	Here the prime denotes a derivative with respect to $r$.  Given an
	expression for the right-hand side in terms of observables ($\kinf$,
	$\ANC$, and so on) and $L$, we can simply integrate to find the energy
	correction for a bc at $L$
	%
	\beq
	  \Delta E_L \equiv E_L - \Einf = \int_{\Einf}^{E_L}\! dE\,
	     = \int_\infty^L\! \frac{\partial E_L}{\partial L} dL
	     \;.
	\eeq
	%
	To derive Eq.~\eqref{eq:dEdL}, we start with
	%
	\beq
	  \frac{\partial}{\partial L}\left [
	   \int_0^L u_L(r) H u_L(r)\, dr = E_L \int_0^L\! dr\, u_L(r)^2
	  \right]
	  \;,
	\eeq
	%
	which yields (after some cancellations)
	%
	\beq
	    \frac12 \left.\left( \frac{\partial u_L(r)}{\partial r}
	         \frac{\partial u_L(r)}{\partial L}
	         \right)\right|_0^L
	   =
	  \frac{\partial E_L}{\partial L} \int_0^L\! dr\, u_L(r)^2
	         \;.
	         \label{eq:DeltaEL}
	\eeq
	%
	The left-hand side is a surface term from partially integrating the
	kinetic energy in $H$.  The lower limit vanishes because $u_L(0) = 0$
	for any $L$.  Finally, we replace the partial derivative with respect
	to $L$ at the upper limit using
	%
	\beq
	   \frac{\partial u_L(L)}{\partial L} = - \frac{\partial u_L(L)}{\partial r}
	   \;,
	\eeq
	%
	which follows from expanding $u_{L'}(L') = 0$ about $u_{L}(L) = 0$ for
	$L' = L + \Delta L$.

	To apply Eq.~\eqref{eq:dEdL}, we start with $u_L(r)$ in the asymptotic
	region, as given by
	\beq
  	u_L(r) \overset{r \gg R}{\longrightarrow}  \left(e^{-k_L r} - e^{-2k_L L}
		e^{k_L r}\right)
   	\;.
   	\label{eq:uLasymp2}
  \eeq
	The normalization
	constant $\gamma_L$ is chosen so that the integral of $u_L(r)^2$ from
	0 to $L$ is unity; it becomes the ANC $\ANC$ as
	$L\rightarrow\infty$.  Thus
	%
	\beq
	   u_L'(L) = -2 \gamma_L k_L e^{-k_L L}
	   \;.
	   \label{eq:uLprime}
	\eeq
	%
	Now we need to expand $k_L$ and $\gamma_L$ about $\kinf$ and $\ANC$,
	respectively.  The leading term is trivial: $k_L \rightarrow \kinf$
	and $\gamma_L \rightarrow \ANC$, so the only $L$ dependence in
	$u_L'(L)^2$ is in $e^{-2 \kinf L}$ and the integration in
	\eqref{eq:dEdL} is immediate:
	%
	\begin{align}
	\Delta E_L
	     = \int_\infty^L\! \frac{\partial E_L}{\partial L} dL
	     =  -2 \ANC^2 \kinf^2 \int_\infty^L\! e^{-2\kinf L} \, dL
	     =  \kinf \ANC^2 e^{-2 \kinf L} +  \mathcal{O}(e^{-4 \kinf L})
	     \;.
	\end{align}
	%
	This is the same LO result for $\Delta E_L$ found by other methods.

	To go to NLO we need an expression for $\gamma_L$.  In the zero-range
	(zr) limit, $\gamma_L$ is given completely in terms of $k_L$ using the
	normalization condition (because the asymptotic form in
	Eq.~\eqref{eq:uLasymp2} holds
	over the entire range of the integral)
	%
	\begin{align}
	  \gamma_L^2 &= \left[\int_0^L\! dr\, (e^{-k_L r} - e^{-2k_L L}e^{k_L r})^2
		\right]^{-1}
	  =
	  2 k_L(1 + 4 k_L L e^{-2 k_L L}) + \mathcal{O}(e^{-4 k_L L}) \;.
	\end{align}
	%
	We expand $k_L$ everywhere in Eq.~\eqref{eq:dEdL} using
	Eq.~\eqref{eq:uLprime} and our LO result
	%
	\beq
	   k_L = \kinf (1 - 2 e^{-2 \kinf L}) \;.
	\eeq
	%
	Here, we neglected terms that are $\mathcal{O}(e^{-6 \kinf L})$ or smaller.
	We need to expand $e^{-2 k_L L}$ in $u_L'(L)$ to get
	%
	\beq
	   e^{-2 k_L L} = e^{-2 \kinf L} (1 + 4 \kinf L e^{-2 \kinf L})
	    + \mathcal{O}(e^{-6 \kinf L}) \;.
	\eeq
	%
	(Elsewhere it suffices to replace $e^{-2 k_L L}$ by $e^{-2 \kinf L}$ to NLO.)
	So we find that
	%
	\begin{align}
	  \frac{\partial E_L}{\partial L} &= -\frac12 (4 \gamma_L^2 k_L^2
		e^{-2 k_L L})
	  \nonumber \\
	  &\approx -2  [2\kinf (1 - 2 e^{-2 \kinf L})(1 + 4 \kinf L
		e^{-2 \kinf L})]
	  \nonumber \\
	  & \  \null \times [\kinf^2 (1 - 4 e^{-2\kinf L})][e^{-2 \kinf L}
		(1 + 4 \kinf L e^{-2 \kinf L})]
	  \nonumber \\
	  &\approx
	  -4 \kinf^3 e^{-2 \kinf L} -  8 \kinf^3 (4\kinf L - 3)e^{-4 \kinf L}
	  + \mathcal{O}(e^{-6 \kinf L}) \;,
	\end{align}
	%
	and then finally
	%
	\begin{align}
	[\Delta E_L]_{\rm zr, NLO}
	     &= \int_\infty^L\! \frac{\partial E_L}{\partial L} dL
	     \nonumber \\
	     &=
	     2 \kinf^2 e^{-2 \kinf L} +  4 \kinf^2 (2\kinf L - 1)e^{-4 \kinf L}
	     + \mathcal{O}(e^{-6 \kinf L})
	     \;,
	\end{align}
	%
	in agreement with Eq.~\eqref{eq:complete_E_correction_NLO}
	with $\ANC^2 = 2\kinf$ and $w_2 = 0$.
	We can take this procedure to higher order by
	using a more general expansion for $k_L$.

	To extend the differential method
	to higher order for nonzero range, we must parametrize
	$\gamma_L$ to account for the part of the integration within the
	range of the potential; e.g., in terms of the effective range.  However,
	we have not found a clear advantage in doing this
	compared to the straightforward S-matrix method.

	\subsection{The proof is in the pudding}
	\label{subsec:pudding_proof}

	In this subsection we will test the Eq.~\eqref{eq:complete_E_correction_NLO}
	for various test models and for the deuteron.  Note that for the cases
	that we test Eq.~\eqref{eq:complete_E_correction_NLO}, the exact answer
	$\Einf$ is already known (either by exact analytical calculation or by using
	large number of basis states).  So we can compare how good the prediction from
	Eq.~\eqref{eq:complete_E_correction_NLO} is by comparing to exact answer.
	In cases where the exact values for $\Einf$ and $\ANC$ are not known, our
	approach suggests that we can use the exponential in $L$ fit of
	Eq.~\eqref{eq:exp_L_extrapolation} to extract $\Einf$ and $\kinf$.

	Based on the results
	presented in Subsec.~\ref{subsec:tale_of_tails}, we use $L_2$ in all our
	further analyses.
	%
	It is important that we isolate the IR corrections in making these
	tests.  The truncation in the HO basis also introduces an
	ultraviolet error inversely proportional to the ultraviolet cutoff
	$\Lambda_{\rm UV} \approx \sqrt{2 \mu \hbar \Omega (N+3/2)}$.  In the
	results here we use combinations of $\hbar \Omega$ and $N$ values such
	that the UV error in each case can be neglected compared to the IR
	error.

	For each of the model potentials, the radial Schr\"odinger equation is
	accurately solved numerically in coordinate space for the energy,
	which yields $\kinf$, and the wave functions.  The asymptotic
	normalization coefficient $\ANC$ is found by multiplying the wave
	function by $e^{\kinf r}$ and reading off its asymptotic value.  This
	is illustrated in the inset of Fig.~\ref{fig:IR_quartic_inset_ANC},
	which also shows the onset of the plateau that defines the asymptotic
	region in $L_2$ where we expect our correction formulas to hold.  For
	the deuteron, the Hamiltonian is diagonalized in
	momentum space to find $\kinf$, and then an extrapolation to the pole
	is used to find the $s$-wave and $d$-wave ANCs~\cite{Amado:1979zz}.
	In the present subsection we use only the $s$-wave ANC for the deuteron.
	%
	\begin{figure}[h]
	\centering
	\includegraphics[width=0.6\textwidth]
	{Extrapolation/quarticIRfitpredictnANC_v2}
  \caption{Energy versus $L_2$ for a quartic potential
    well Eq.~\eqref{eq:Vq} for a wide range of $N$ and $\hw$
    (circles) ($\hbar = \mu = R=1$).  The solid line is a fit to
		Eq.~\eqref{eq:exp_L_extrapolation}
    with $A$, $\kinf$ and $\Einf$ as fit parameters while the dashed line is
		the prediction from Eq.~\eqref{eq:complete_IR_scaling}.
    The horizontal line is the exact energy, $\Einf=-1.0115$.
    The inset illustrates the calculation of the asymptotic
    normalization coefficient (ANC) from the (normalized) wave
    function.}
		\label{fig:IR_quartic_inset_ANC}
	\end{figure}

	The derivations in Subsec.~\ref{subsec:make_cash} imply that the
	energy corrections should have the same exponential form and
	functional dependence on the radius $L$ at which the wave
	function is zero, independent of the potential and for
	any bound state.
	Here we make some representative tests of a direct fit of
	Eq.~\eqref{eq:exp_L_extrapolation} in comparison to applying
	Eq.~\eqref{eq:complete_IR_scaling}.

	Figure~\ref{fig:IR_quartic_inset_ANC} shows results for a quartic
	potential with a moderate depth.  The fit to
	Eq.~\eqref{eq:exp_L_extrapolation} is very good over a large range in $L_2$
	for which the energy changes by 30\%, and the prediction for $\Einf$
	is accurate to 0.2\%.  However, the fit value of $\kinf$ is 1.61
	compared to the exact value of 1.42.  The dashed curve shows the
	prediction from Eq.~\eqref{eq:complete_IR_scaling} using the exact
	$\kinf$ and $\ANC$.  It is evident that the approximation is very good
	above $L_2 > 2$ but increasingly deviates at smaller $L_2$.

	\begin{figure}[h]
	\centering
	\includegraphics[width=0.6\textwidth]
	{Extrapolation/sq_gaussIRfitspredictn_v2}
	\caption{Energy versus $L_2$ for moderate-depth (a) square
  	well Eq.~\eqref{eq:Vsw} and for (b) Gaussian potential
  	well Eq.~\eqref{eq:Vg} ($\hbar = \mu = R=1$) for a wide range of $N$ and
		$\hw$ (circles).  The solid line is a fit to
		Eq.~\eqref{eq:exp_L_extrapolation} with $A$, $\kinf$ and $\Einf$ as
  	fit parameters while the dashed line is the
  	prediction from Eq.~\eqref{eq:complete_IR_scaling}.  The horizontal dotted
		lines are the exact energies;
   	square well: $\Einf = -1.5088$, Gaussian well: $\Einf = -1.2717$}
	\label{fig:universal_sq_gauss_wells}
	\end{figure}
	%

	In Fig.~\ref{fig:universal_sq_gauss_wells}, examples are shown for
	square well and Gaussian potentials with a moderate depth.  Again we
	find a good fit to an exponential fall-off in $L_2$, but in these
	cases not only are the energies well predicted (again to better than
	0.2\%) but the fit values of $\kinf$ are within 5\% of the exact results.
	%
	\begin{figure}[h]
	\centering
	\includegraphics[width=0.6\textwidth]
	{Extrapolation/deepgaussV10predictnANC_v2}
	\caption{Energy versus $L_2$ for the deeply bound
  	ground state of a Gaussian potential for a wide range of $N$ and
  	$\hw$ (circles) ($\hbar = \mu = R = 1$).  These are compared to the
		prediction of
	  Eq.~\eqref{eq:complete_IR_scaling} (dashed).  The solid line
	  is a fit to Eq.~\eqref{eq:exp_L_extrapolation} with $A$, $\kinf$ and
		$\Einf$ as fit parameters.  The horizontal dotted line
	  is the exact energy, $\Einf=-4.2806$.}
	\label{fig:deep_gauss_ANC_inset}
	\end{figure}
	%
	\begin{figure}[h]
	\centering
	\includegraphics[width=0.6\textwidth]
	{Extrapolation/deepexpV10predictnANC_v2}
	\caption{Energy versus $L_2$ for the deeply bound
	  ground state of an exponential potential well for a wide range of
	  $N$ and $\hw$ (circles) ($\hbar = \mu = R = 1$).  These are compared to the
		predictions of Eq.~\eqref{eq:complete_IR_scaling} (dashed).  The solid line
	  is a fit to Eq.~\eqref{eq:exp_L_extrapolation} with $A$, $\kinf$ and
		$\Einf$ as fit parameters.  The horizontal dotted line
	  is the exact energy, $\Einf=-3.3121$.}
	\label{fig:deep_exp_ANC_inset}
	\end{figure}
	%
	For deeply bound states, Eq.~\eqref{eq:complete_IR_scaling} fails
	for a different reason.  The error in
	Eq.~\eqref{eq:complete_IR_scaling} is proportional to $e^{-4 \kinf L}$,
	so one might expect that the prediction to become increasingly
	accurate as the state becomes more bound.  However, as seen in
	Figs.~\ref{fig:deep_gauss_ANC_inset} and \ref{fig:deep_exp_ANC_inset},
	results for deep Gaussian and exponential potential wells do not match
	this expectation.  In deriving the energy corrections we used
	the asymptotic form of the wave functions.  This is valid only in the
	region $r \gg R$, where $R$ is the range of the potential.  The
	potentials at the smaller values of $L_2$ shown in the figures are not
	negligible.  Indeed, it is evident from the insets in
	Figs.~\ref{fig:deep_gauss_ANC_inset} and \ref{fig:deep_exp_ANC_inset}
	that we are not in the asymptotic region for those values of $L$.
	The lesson is that when applying the IR extrapolation schemes
	discussed in the present paper we need to make sure that the two
	conditions for its applicability are fulfilled.  First, we need $N$
	sufficiently large
	for $L_2$ to be the correct box size
	(see Table~\ref{tab:L0_L2_k_min_comparison}).
	Second we need $L_2$ to be the
	largest length scale in the problem under consideration.

	\begin{figure}[h]
	\centering
	\includegraphics[width=0.6\textwidth]
	{Extrapolation/gaussquartexctdIRpredictn1_v2}
	\caption{Energy versus $L_2$ for the first excited
	  states of deep (a) Gaussian Eq.~\eqref{eq:Vg} and (b) quartic
	  Eq.~\eqref{eq:Vq} potential wells for a wide range of $N$ and $\hw$
	  (circles) ($\hbar = \mu = R=1$).  The solid line is a fit to
		Eq.~\eqref{eq:exp_L_extrapolation}
	  with $A$, $\kinf$ and $\Einf$ as fit parameters while the dashed line is
		the prediction from Eqs.~\eqref{eq:complete_IR_scaling}.
	  The horizontal dotted lines are the exact energies for the first excited
		states; Gaussian well: $\Einf = -1.2147$, quartic well: $\Einf = -1.8236$}
	  \label{fig:IR_excited_gauss_quartic_wells}
	\end{figure}
  %
	The results so far are for the ground state of the potential.  However,
	the derivations in the Subsec.~\ref{subsec:make_cash} should
	also hold for excited states.  This is so because the generalization of
	the results in Subsec.~\ref{subsec:tale_of_tails} shows that
	$(j\pi/L_2)^2$ is a very good approximation to the $j^{\rm th}$
	eigenvalue of the operator $p^2$ for $j \ll N$.  In
	Fig.~\ref{fig:IR_excited_gauss_quartic_wells} representative results
	for excited states from two model potentials are shown.  We find the
	same systematics as with the ground-state results: the exponential fit
	works very well but the extracted $\kinf$ is only correct at about the 10\%
	level.  In assessing the success of	Eq.~\eqref{eq:complete_IR_scaling}, we
	note that these excited states in deep potentials are comparable to the
	ground states in moderate-depth potentials shown in
	Fig.~\ref{fig:universal_sq_gauss_wells}.  The discussion there applies
	here as well, namely that the prediction from
	Eq.~\eqref{eq:complete_IR_scaling} is very good at large $L_2$, but
	increasingly deviates at smaller $L_2$.

	\begin{figure}[h]
	\centering
	\includegraphics[width=0.6\textwidth]
	{Extrapolation/deutrngauss_sq_v2}
	\caption{ (a) Ground-state energy versus $L_2$ for
    model Gaussian potential. (b) Energy versus $L$
    for the square well.  The energies for the square well are from solving
		the	Schr\"odinger equation
    exactly with a Dirichlet bc on wave functions at
    $r=L$. The dashed line is the prediction from
    Eqs.~\eqref{eq:complete_IR_scaling}.  The depths of these model potentials
    are chosen so that the scaled
    energies (with $\hbar = \mu = R=1$) are the same as the deuteron binding
		energy.}
	\label{fig:prediction_deuteron_equivalent}
	\end{figure}
	%
	The case of weakly bound states is of special interest because of the
	correspondence to deuteron which is also a weakly bound shallow state.
	Figure~\ref{fig:prediction_deuteron_equivalent} (a) shows ground-state
	energies for many different $N$ and $\hw$ versus $L_2$ using Gaussian model
	potentials whose parameters are chosen so that the energies are the
	same as the deuteron binding energy (scaled to units with $\hbar=1$,
	$\mu=1$, $R=1$).
	In Fig.~\ref{fig:prediction_deuteron_equivalent} (b)
	we have a `deuteron-like' square well. The energies in this case are
	obtained by	solving the Schr\"odinger equation exactly with a Dirichlet
	bc on wave
	functions at $r=L$.  The prediction from Eq.~\eqref{eq:complete_IR_scaling}
	fails to reproduce the data except at the highest values of $L_2$.
	Eq.~\eqref{eq:complete_IR_scaling} keeps only the leading-order (LO)
	corrections	to $\Delta E$.  For the weakly bound states, the higher-order
	corrections become important.  The corrections to the energy up to the next
	to leading order (NLO) were noted in Eq.~\eqref{eq:complete_E_correction_NLO}.
	Note that the NLO correction involves the low-energy constant $w_2$.  In
	certain cases, this constant $w_2$ can be calculated and $\Delta E$ up to NLO
	can be obtained.  Now we turn to the test of
	Eq.~\eqref{eq:complete_E_correction_NLO}.

	\medskip
	\subsubsection{NLO and systematics of the correction}

	For the square-well potential of Eq.~\eqref{eq:Vsw}, the parameters in
	Eq.~\eqref{eq:complete_E_correction_NLO} can be calculated easily.
	The $S$-wave scattering phase shift for	the square well is
	%
	\beq
	 \delta_0 (k) =
	  \tan^{-1}\left[\sqrt{{k^2 \over k^2 + \eta^2}} \tan(\sqrt{k^2+\eta^2}
	  R)\right] - k R\;,
	  \label{eq:delta0}
	\eeq
	%
	with $\eta = \sqrt{2 V_0}$. 	Analytically continuing the effective
	range expansion by taking $k \rightarrow i k_L$
	in Eqs.~\eqref{eq:eff_range_kinf} and \eqref{eq:delta0}, we obtain
	%
  \begin{multline}
	{i k_L \sqrt{\eta^2 - k_L^2} -k_L^2 \tan( \sqrt{\eta^2 - k_L^2} R)
	\tan(i k_L R) \over i k_L \tan(\sqrt{\eta^2 - k_L^2} R) -
	\sqrt{\eta^2 - k_L^2} \tan(i k_L R) }
	=  \\
	-\kinf
	+ \frac12{\rho_d} (\kinf^2-k_L^2) + w_2 (\kinf^2 - k_L^2)^2
	+ \mathcal{O}\left((\kinf^2-k_L^2)^3\right)	\;.
	\label{eq:get_w2_sq_well}
	\end{multline}
	%
	The branch for the square-root is fixed by the requirement that
	$\tan \delta (i\kinf) = -i$.  Note from Eq.~\eqref{eq:s0_exact} that
	this builds in the requirement that the $S$-matrix has a pole at
	$i \kinf$.  To get $\rho_d$ ($w_2$) we differentiate once (twice) each
	side of Eq.~\eqref{eq:get_w2_sq_well} with
	respect to $k_L$ and then set $k_L = \kinf$.  The $\rho_d$ obtained in
	this way is consistent with Eq.~\eqref{eq:rhoD_gamma_rel} when $\ANC$
	is obtained by the large $r$ behavior of the bound-state wave function
	as defined in Eq.~\eqref{eq:definition_of_ANC}.

	The square well with a Dirichlet bc at $L > R$ can be solved analytically.
	The wave functions inside and outside the square well are
	\beq
  u^<_L(r) = C \sin\kappa_L r \,,
  \qquad
  u^>_L(r) = D(e^{-k_L r} - e^{-2 k_L L} e^{+k_L r})
  \;,
	\eeq
	%
	which builds in the boundary condition $u^>_L(L) = 0$.
	The interior wave number $ \kappa_L = \sqrt{\eta^2 - k_L^2}$ and
	$k_L = \sqrt{2 |E_L|}$.
	Matching the logarithmic derivatives at $r=R$ for $E=\Einf$ yields
	%
	\beq
	  \kappa_\infty \cot{\kappa_\infty R} = -\kinf
	\eeq
	%
	and with the boundary condition at $L$ we get:
	%
	\beq
	 \kappa_L \cot \kappa_L R = - k_L \frac{1 + e^{-2k_L (L-R)}}
	 {1-e^{-2k_L (L-R)}} \;.
	 \label{eq:sq_well_matchingL}
	\eeq
	%
	We expand both sides of Eq.~\eqref{eq:sq_well_matchingL} in powers of
	%
	\beq
	  \Delta k \equiv k_L - \kinf\;.
	  \label{eq:Delta_k_def}
	\eeq
	%
	We write the left-hand side of Eq.~\eqref{eq:sq_well_matchingL} as
	%
	\beq
	\kappa_L \cot \kappa_L R = \kappainf \cot \kappainf R + \mathcal{A}
	(\Delta k)  + \mathcal{B} (\Delta k)^2+\cdots \;,
	\label{eq:LHS_matchingL_expansion}
	\eeq
	%
	and obtain the coefficients $\mathcal{A}$, $\mathcal{B}$ by Taylor
	expanding $\kappa_L	\cot (\kappa_L R)$ around $\kinf$.
	We write $\Delta k$ as
	%
	\beq
	\label{eq:Delta_k_def2} \Delta k = k_{(1)} + k_{(2)} + \cdots \;.
	\eeq
	%
	Here $k_{(1)} \sim e^{-2 \kinf L}$ is the LO correction, $k_{(2)}
	\sim e^{-4\kinf L}$ is the NLO correction and so on, and we truncate the
	expressions consistently to obtain the energy correction for the
	square well to the desired order.  The results of the general S-matrix
	and square-well-only Taylor expansion methods of calculating energy
	corrections are found to match explicitly at LO, \LNLO, and NLO.
	We remind the reader that we use \LNLO~to denote terms proportional
	to $L e^{-4 \kinf L}$.

	\begin{figure}[h]
	\centering
	\includegraphics[width=0.6\textwidth]
	{Extrapolation/sqwell_curvesV4_v2}
	\caption{Bound-state energy for a square well of depth
	  $V_0=4$ (lengths are in units of $R$ and energies in units of
	  $1/R^2$ with $\hbar^2/\mu = 1$) from solving the Schr\"odinger
	  equation with a Dirichlet bc at $r=L$.  The diamonds
	  are exact results for each $L$ while the horizontal dotted line is
	  the energy for $L\rightarrow\infty$, $\Einf = -1.5088$.  The dashed,
	  dot-dashed and solid lines are predictions for the energy using the
	  systematic correction formula
	  Eq.~\eqref{eq:complete_E_correction_NLO} at LO (first term only),
	  \LNLO\ (first two terms), and full NLO (all terms), respectively.
	  The dotted curve on top of the solid line and the dot-double-dashed
	  lines are respectively the NLO and N2LO predictions for the square
	  well from the Taylor expansion using Eqs.~\eqref{eq:sq_well_matchingL}
		and \eqref{eq:LHS_matchingL_expansion}.}
	\label{fig:sqwell_curvesV4}
	\end{figure}
	%
	\begin{figure}[h]
	\centering
	\includegraphics[width=0.6\textwidth]
	{Extrapolation/sq_well_error_log_plotsV4_v2}
	\caption{Error plots of the energy correction at each
	  $L$ for the square well of Fig.~\ref{fig:sqwell_curvesV4} ($V_0 =
	  4$) predicted at different orders by
	  Eq.~\eqref{eq:complete_E_correction_NLO} and by the Taylor expansion
	  method, each compared to the exact energy.
	  Lines proportional to $L e^{-4\kinf L}$ (dashes) and $L^2 e^{-6\kinf L}$
		(with arbitrary normalization) are plotted for comparison to
	  anticipated error slopes. }
	\label{fig:sq_well_error_log_plotsV4}
	\end{figure}
	%
	Figure~\ref{fig:sqwell_curvesV4} compares the energy corrections
	for the general S-matrix method at LO and NLO for a representative
	square-well potential with one bound state to the exact energies.  The
	Taylor expansion results for the square well at NLO and N2LO (which is
	proportional to $e^{-6\kinf L}$) are also plotted.  We note that the
	predictions are systematically improved as higher-order terms are
	included and that keeping terms only up to \LNLO~overestimates the
	energy correction.  Also as seen in Fig.~\ref{fig:sqwell_curvesV4},
	the full NLO energy correction predicted by
	Eq.~\eqref{eq:complete_E_correction_NLO}, with $w_2$ determined by
	Eq.~\eqref{eq:get_w2_sq_well}, matches the `exact' NLO result obtained
	by Taylor expansion.  This confirms that
	Eq.~\eqref{eq:complete_E_correction_NLO} is indeed the complete energy
	correction at NLO.

	To see if the errors decrease with the implied systematics, we plot
	the difference of actual energy corrections and the energy corrections
	predicted at different orders on a log-linear scale in
	Fig.~\ref{fig:sq_well_error_log_plotsV4}.  We observe that the errors
	successively decrease at each fixed $L$ as we go from LO to NLO to
	N2LO.  The up triangles in Fig.~\ref{fig:sq_well_error_log_plotsV4} are
	$\Delta E_{\rm actual} - \Delta E_{\rm LO}$.  From
	Eq.~\eqref{eq:complete_E_correction_NLO} the dominant omitted
	correction in $\Delta E_{\rm LO}$ is proportional to $L e^{-4 \kinf L}$.
	As seen in Fig.~\ref{fig:sq_well_error_log_plotsV4}, the slope
	of $\Delta E_{\rm actual} - \Delta E_{\rm LO}$ is roughly $L
	e^{-4 \kinf L}$, as expected.  We also note that $\Delta E_{\rm \LNLO}$
	is only a marginal improvement over $\Delta E_{\rm LO}$ and that
	$\Delta E_{\rm actual} - \Delta E_{\rm NLO}$ has the expected slope of
	$L^2 e^{-6 \kinf L}$.  We again see a perfect agreement between the
	results obtained from the S-matrix method
	(Eq.~\eqref{eq:complete_E_correction_NLO}) and those obtained from the
	Taylor expansion of Eq.~\eqref{eq:sq_well_matchingL}.  We have also
	studied deeper square wells with more than one bound state
	and verified that our results apply even in the presence of excited
	states.

	\begin{figure}[h]
	\centering
	\includegraphics[width=0.55\textwidth]
	{Extrapolation/sqwell_curves_v2}
	\caption{Bound-state energy for a square well of depth
	  $V_0=1.83$ (units with $R=1$), which simulates a deuteron, from
	  solving the Schr\"odinger equation with a Dirichlet bc at $r=L$.
		The horizontal dotted line is the exact energy,
	  $\Einf = -0.1321$ and the other curves are as the same as in
	  Fig.~\ref{fig:sqwell_curvesV4}.}
	\label{fig:sqwell_curves}
	\end{figure}
	%
	\begin{figure}[h]
	\centering
	\includegraphics[width=0.55\textwidth]
	{Extrapolation/sq_well_error_log_plots_v2}
	\caption{Comparison of the actual energy correction due
	  to truncation to the energy correction predicted to different orders
	  by Eq.~\eqref{eq:complete_E_correction_NLO} for a square well
	  (Eq.~\eqref{eq:Vsw}) with $V_0 = 1.83$ and $R=1$.}
	\label{fig:sq_well_error_log_plots}
	\end{figure}
	%
	In Figs.~\ref{fig:sqwell_curves} and
	\ref{fig:sq_well_error_log_plots}, the same analysis is done but now
	with the depth of the square well adjusted so that the exact binding
	energy is the same as the deuteron binding energy scaled to the units
	$\hbar=1$, $\mu=1$ and $R=1$. An important difference in this case
	compared to the deeper square well is that the \LNLO~prediction gives
	a very close estimate for the truncated energies.  However as seen in
	Fig.~\ref{fig:sq_well_error_log_plots}, the improvement achieved by
	the \LNLO~prediction is not systematic with $L$.  At large $L$,
	$\Delta E-\Delta E_{\rm \LNLO}$ has the same slope as $\Delta E -
	\Delta E_{\rm LO}$ and is not the dominant NLO correction.  In this
	regard, the proximity of \LNLO~ prediction to the actual data in
	Fig.~\ref{fig:sqwell_curves} should not be over-emphasized.
	%
	\begin{figure}[h]
	\centering
	\includegraphics[width=0.6\textwidth]
	{Extrapolation/deuteron_curves_v2}
	\caption{Deuteron energy versus $L_2$ (see
	  Eq.~\eqref{eq:L2_def}) for the chiral N$^3$LO (500\,MeV) potential
	  of Ref.~\cite{Entem:2003ft}.  To eliminate the UV contamination we
	  only plot results for $\hw > 49$~MeV.  The dashed, dot-dashed and solid
	  lines are respectively the LO (first term in
	  Eq.~\eqref{eq:complete_E_correction_NLO}), \LNLO~(first two terms in
	  Eq.~\eqref{eq:complete_E_correction_NLO}) and the full NLO (all the
	  terms in Eq.~\eqref{eq:complete_E_correction_NLO}) predictions for
	  the energy correction.  The horizontal dotted line is the deuteron
	  energy.}
	\label{fig:deuteron_curves}
	\end{figure}
	%
	\begin{figure}[h]
	\centering
	\includegraphics[width=0.6\textwidth]
	{Extrapolation/deuteron_error_log_plots_v3}
	\caption{Comparison of the actual energy correction due
	  to HO basis truncation ($\hw$ restricted to be greater than $49$~MeV to
	  eliminate UV contamination) for the deuteron to the energy
	  correction predicted to different orders from
	  Eq.~\eqref{eq:complete_E_correction_NLO}.  For the parameter $w_2$ in
	  Eq.~\eqref{eq:complete_E_correction_NLO} we use the value reported
	  in \cite{Phillips:1999hh}.}
	\label{fig:deuteron_error_log_plots}
	\end{figure}
	%
	Figures~\ref{fig:deuteron_curves} and
	\ref{fig:deuteron_error_log_plots} show analogous results for the
	deuteron calculated with the chiral EFT potential of
	Ref.~\cite{Entem:2003ft}.  We use the HO basis and predict the ($l=0$) energy
	correction from Eq.~\eqref{eq:complete_E_correction_NLO} assuming a
	Dirichlet bc at $L_2$ given by Eq.~\eqref{eq:L2_def}.  We only include
	energies for which $\hw > 49$~MeV, which is sufficient to render UV
	corrections negligible.  For the parameter $w_2$ in
	Eq.~\eqref{eq:complete_E_correction_NLO} we use $w_2
	= 0.389$ as reported in \cite{Phillips:1999hh}.  We also note that the
	$\rho_d$ value reported in \cite{Phillips:1999hh} satisfies
	Eq.~\eqref{eq:rhoD_gamma_rel}, where $\ANC$ now is the $s$-wave ANC.
	The $y$-axis minimum is dictated by the limited precision of
	the ANC and $w_2$ values.
	We notice again that the close agreement of the \LNLO~prediction to
	the deuteron data is not systematic while the full corrections to the
	LO and NLO predictions have the anticipated slopes except at large
	$L_2$.  In the next Subsec.~\ref{subsec:higher_angular_momenta} we
	extend our formulas to $l>0$, which
	enables us to include contributions from the $d$-wave at LO.  This
	becomes noticable on the error plot for large $L_2$ (see
	Fig.~\ref{fig:deuteron_error_plot_l2}).
  %
	\begin{figure}
	\centering
 	\includegraphics[width=0.6\columnwidth]
	{Extrapolation/srgwidfithbaromega40every13_v2}
 	\caption{Deuteron energy versus $L_2$ for the
  	potential of Ref.~\cite{Entem:2003ft} evolved by the SRG to four
   	different resolutions (specified by $\lambda$).  To eliminate the
   	UV contamination we only keep points for which $\hw >40$.  The
   	horizontal dotted line is the deuteron binding energy.}
	\label{fig:SRG_data_collapse}
	\end{figure}
	%

	As a final test of the universal applicability of the correction
	formula Eq.~\eqref{eq:complete_E_correction_NLO}, we consider a sequence
	of unitarily equivalent	potentials for the deuteron.  In particular,
	we use the similarity renormalization group (SRG)~\cite{Bogner:2009bt}
	to evolve the initial Entem-Machleidt
	potential to four values of the SRG evolution parameter $\lambda$.
	Because the transformation is exactly unitary (up to very small
	numerical errors) at the two-body level, the measurable quantities
	such as phase shifts, bound-state energies, and ANCs are unchanged.
	From Eq.~\eqref{eq:complete_E_correction_NLO} or more generally from
	Eq.~\eqref{eq:basiceq}, we see that the IR energy correction can written
	in terms of observables ($S$-matrix near the bound state can be
	parametrized in terms of low-energy observables) and therefore should be
	independent of the SRG scale.  This is verified in
	Fig.~\ref{fig:SRG_data_collapse}.

	\begin{figure}
		\centering
	 \includegraphics[width=0.6\textwidth]
	 {Extrapolation/deutrnNmaxsrg_v2}
	 \caption{The same SRG-evolved potentials as in
	   Fig.~\ref{fig:SRG_data_collapse} are used to generate energies, but
	   with $N$ fixed at (a) 8 and (b) 12 and no restriction on
	   $\hw$.  Thus UV corrections are not negligible everywhere.  The
	   horizontal dotted line is the deuteron binding energy.}
	 \label{fig:SRG_data_collapse_contamination}
	\end{figure}
	%
	As $\lambda$ decreases, the SRG systematically reduces the coupling
	between high-momentum and low-momentum potential matrix elements, thereby
	lowering the effective UV cutoff.  Thus these potentials are useful
	tools to assess the role of UV corrections.  This is exploited in
	Fig.~\ref{fig:SRG_data_collapse_contamination} where we relax the condition
	that the UV corrections are small compared to IR corrections.  In
	particular, we fix $N$ at 8 and 12 and scan through the full range of
	$\hw$.  We observe that with increasing $L_2$, each of the curves with
	a given $\lambda$ eventually deviates from the universal curve,
	first with $\lambda = 3.0\fmi$ and then later with decreasing
	$\lambda$ or with higher $N$.  We can understand this in terms of the
	behavior of the induced UV cutoff.  For fixed $N$, Eq.~\eqref{eq:L2_def}
	tells us that increasing $L_2$ means increasing $b$ (or decreasing
	$\hw$).  But at fixed $N$, $\LamUV \propto 1/b$, so the UV cutoff will
	be decreasing and the corresponding UV energy correction increasing.
	Thus the curves at fixed $\lambda$ correspond to the curves
	seen in conventional plots of energy versus $\hw$ (e.g., see
	Ref.~\cite{Bogner:2007rx}).  The softer potentials (lower $\lambda$)
	will have lower intrinsic UV cutoffs and therefore they are only
	affected for larger $L_2$. The minima for each $\lambda$ are when IR
	and UV corrections are roughly equal.

	\medskip
	\subsubsection{Comparison to conventional extrapolation schemes}

	The extrapolation formulas in Eqs.~\eqref{eq:complete_IR_scaling} and
	\eqref{eq:complete_E_correction_NLO} with $L=L_2$ are theoretically founded.
	Thus the functional form for the energy extrapolation that we have is an
	exponential in $L$.  As mentioned in the introduction to this chapter,
	a popular	phenomenological choice is an exponential in $N$ extrapolation
	\big(Eq.~\eqref{eq:exp_Nmax_extrapolation}\big).  From
	Eq.~\eqref{eq:L2_def} we see that exponential in $N$ extrapolation
	corresponds to gaussian in $L$.  Authors of
	Ref.~\cite{Tolle:2012cx} investigated the convergence properties of
	genuine and smeared contact interactions in
	an effective theory of trapped bosons and found that the smearing
	changed a power law dependence of the convergence to an exponential
	dependence.  Here we will consider all three functional dependences on
	$L$: exponential, Gaussian, and power law.
	%
	\begin{figure}[h]
	\centering
	\includegraphics[width=0.6\textwidth]
	{Extrapolation/gaussianV5_IRfits_v2}
	\caption{The IR energy correction $\Delta E_L$ versus $L_2$
		for a Gaussian potential well Eq.~\eqref{eq:Vg} with
		$V_0 = 5$ (and $\hbar = \mu = R=1$) using a wide range of $N$ and $\hw$.
		%The exact ground-state energy is $\Einf=-1.2717$.
		The energies are fitted with (a) exponential, (b) Gaussian,
		and (c) power law dependence on $L_2$.}
	\label{fig:Vgauss_other_forms}
	\end{figure}
	%
	\begin{figure}[h]
	\centering
	\includegraphics[width=0.6\textwidth]
	{Extrapolation/deuteronIRfits_v2}
	\caption{The IR energy correction $\Delta E_L$
		versus $L_2$ for the deuteron calculated with the chiral EFT potential from
		Ref.~\cite{Entem:2003ft} using a wide range of $N$ and $\hw$.
		%The exact ground-state energy is $\Einf=-2.2246\,\mbox{MeV}$.
		The energies are fitted with (a) exponential, (b) Gaussian,
		and (c) power law dependence on $L_2$.}
	\label{fig:deuteron_other_forms}
	\end{figure}
	%
	A purely empirical test can be made for our models and the deuteron
	because we can calculate the exact $\Einf$, plot $\Delta E(L_2) \equiv
	E(L_2) - \Einf$ against $L_2$, and then attempt to fit each of the
	three choices of $\Delta E(L_2)$.  Figure~\ref{fig:Vgauss_other_forms}
	shows the results for a representative model potential (a Gaussian)
	with moderate depth while Fig.~\ref{fig:deuteron_other_forms} shows
	the results for the deuteron.  The plots are made so that the
	candidate form would yield a straight line if followed precisely.  We
	see that the exponential form is an excellent fit for the model
	throughout the range of $L_2$ and a reasonable but not perfect fit for
	the deuteron.  The not perfect fit for the deuteron can be explained
	by noting that it is a weakly bound state.  We see from
	Figs.~\ref{fig:deuteron_curves} and \ref{fig:deuteron_error_log_plots}
	that for deuteron, the correction will be a sum of exponential terms.
	In contrast to the exponential extrapolation, Gaussian and power law fits
	fail over the full range of $L_2$.  This is consistent with
	Tolle {\it et al.}~\cite{Tolle:2012cx}.  For limited ranges of $L_2$ a
	Gaussian does	provide a reasonable fit (and should give a good extrapolation
	for	$\Einf$ if close enough to convergence).  This is consistent with the
	apparent success of exponential in $N$ extrapolation observed in the
	literature.  However, we see that globally exponential in $L$ is clearly
	superior.  Moreover, the exponent and coefficient (see
	Eq.~\eqref{eq:complete_IR_scaling}) are physically motivated whereas in the
	exponential in $N$ case (Eq.~\eqref{eq:exp_Nmax_extrapolation}) these
	parameters are fit to data.


	\subsection{Higher angular momenta}
	\label{subsec:higher_angular_momenta}

	The deuteron ground state is a mixture of an $s$ and a $d$ state, and
	the $s$ and $d$ asymptotic normalization coefficients (as well as the
	$d$-to-$s$ state ratio of about 2.5\%) are observables.  The
	extrapolation formulas so far were derived for $s$
	states, and it is of interest to extend these to nonzero angular
	momenta $l$.  We do so in two steps.  First, we show that $L_2$ is also
	the relevant effective hard-wall radius for oscillator wave functions
	with nonzero angular momenta.  Second, we derive the energy correction
	for nonzero angular momenta.

	\medskip
	\subsubsection{$L$ for nonzero angular momenta}

	For the derivation of the relevant IR length scale at $l>0$ we closely
	follow the derivation for $l=0$ presented in
	Subsec.~\ref{subsec:tale_of_tails}.  We compute the smallest eigenvalue
	$\kappa^2$ of the squared momentum operator $\hat{p}^2$ in a finite
	oscillator basis and identify $\kappa =x_l/L$ (with $x_l$ being the
	smallest positive zero of the spherical Bessel function $j_l$). This
	identification, and the form of the corresponding eigenfunctions are,
	of course, guided by the Dirichlet bc at $r=L$.  We set the oscillator
	length $b=1$.  Because this is the only length scale here, the results
	are general and can be extended to any $b$ with simple rescaling.
	The normalized radial	oscillator wave function of energy
	%
	\beq
	  E = 2n + l + 3/2
	\eeq
	%
	is $\psi_{nl}(r)=u_{nl}(r)/r$ with
	%
	\beq
	u_{nl}(r)=\sqrt{2n!\over\Gamma(n+l+3/2)} r^{l+1} e^{-r^2/2}
	L_{n}^{l+1/2}(r^2) \;.
	\eeq
	%
	Here, $L_n^{l+1/2}$ denotes the generalized Laguerre polynomial.

	In this basis, the operator $\hat{p}^2$ of the momentum squared is
	tridiagonal with matrix elements
	%
	\begin{align}
	\langle u_{ml}|\hat{p}^2|u_{nl}\rangle &= (2n+l+3/2)\delta_m^n \nonumber\\
	   & \qquad \null +\sqrt{n+1}
	   \sqrt{n+l+3/2}\,\delta_m^{n+1} \nonumber \\
	  & \qquad \null +\sqrt{n}\sqrt{n+l+1/2}\,\delta_m^{n-1} \;.
	\end{align}
	%
	For the eigenfunction of $\hat{p}^2$ with smallest eigenvalue
	$\kappa^2$ at angular momentum $l$, we make the ansatz
	$\psi_{\kappa l}(r)/r$ with
	%
	\bea
	\label{eq:eigen_l_gen}
	\psi_{\kappa l}(r) =\left\{\begin{array}{cc}
	\kappa r j_l(\kappa r) \;, & 0\le \kappa r \le x_l \;, \\
	0 \;, & \kappa r > x_l \;.\end{array}\right.
	\eea
	%
	Here, $j_l$ is the regular spherical Bessel function and $x_l$ is its
	smallest positive zero.  Clearly, these eigenfunctions are those of a
	particle in a spherical cavity with a Dirichlet bc at $x_l/\kappa$. In
	an infinite basis, the wave function $\psi_{\kappa l}(r)/r$ is an
	eigenfunction of $\hat{p}^2$ for any non-negative value of $\kappa$.
	In a finite oscillator basis, only discrete momenta $\kappa$ are
	allowed.  For their computation we expand the eigenfunction as
	%
	\beq
	\psi_{\kappa l}(r) = \sum_{m=0}^{n} c_m(\kappa) u_{ml}(r) \;,
	\eeq
	%
	where we supress the dependence of the admixture coefficients
	$c_m(\kappa)$ on $l$, which is kept fixed throughout this derivation.

	The last row of the matrix eigenvalue problem for $\hat{p}^2$ is
	%
	\beq
	\label{quantize}
	(2n+l+3/2 -\kappa^2)c_n(\kappa) = -\sqrt{n}\sqrt{n+l+1/2}\, c_{n-1} \;,
	\eeq
	%
	and this becomes the quantization condition for $\kappa$.  The direct
	computation of the coefficients $c_n(\kappa)$ seems difficult.
	Instead, we make a Fourier-Bessel expansion
	%
	\beq
	\label{fourierbessel}
	\psi_{\kappa l}(r)= \sqrt{2\over \pi}\int\limits_0^\infty\! dk\,
	   \tilde{\psi}_{\kappa l}(k)\, kr j_l(kr) \;,
	\eeq
	%
	and use
	%
	\beq
	kr j_l(kr) =\sqrt{\pi\over 2}\sum_{n=0}^\infty (-1)^n u_{nl}(k) u_{nl}(r) \;.
	\eeq
	%
	Thus,
	%
	\beq
	\psi_{\kappa l}(r)= \sum_{n=0}^\infty (-1)^n u_{nl}(r)
	\int\limits_0^\infty\! dk\, \tilde{\psi}_{\kappa l}(k) u_{nl}(k) \;,
	\eeq
	%
	and the admixture coefficients are therefore
	%
	\beq
	c_n(\kappa)= (-1)^n \int\limits_0^\infty dk\, \tilde{\psi}_{\kappa l}(k)
	u_{nl}(k) \;.
	\eeq
	%

	So far, our formal manipulations have been exact.  We now employ an
	asymptotic approximation of the generalized Laguerre polynomials
	(which enters the $u_{nl}(k)$) in terms of Bessel functions, valid for
	$n\gg 1$, see Eq.~15 of Ref.~\cite{Deano2013}.  This yields
	%
	\begin{align}
	u_{nl}(k) \approx {2^{1-n}\over \pi^{1/4}} \sqrt{(2n+2l+1)!\over (n+l)! n!}
	   \,(4n+2l+3)^{-{l+1\over 2}}
	  %\nonumber \\
	   \sqrt{4n+2l+3} k\, j_l(\sqrt{4n+2l+3}k) \;,
	\end{align}
	%
	and
	%
	\begin{align}
	c_n(\kappa) \approx C_{nl} \sqrt{2\over \pi}\int\limits_0^\infty dk \,
	\tilde{\psi}_{\kappa l}(k)	\sqrt{4n+2l+3} k
	 j_l(\sqrt{4n+2l+3}k) \;.
	\end{align}
	%
	Here, $C_{nl}$ is a constant that does not depend on $\kappa$.  The key
	point is that the asymptotic expansion in terms of Bessel functions
	allows us now to employ the definition in Eq.~\eqref{fourierbessel} to
	evaluate the integral
	%
	\begin{align}
	 \sqrt{2\over \pi}\int\limits_0^\infty dk\, & \tilde{\psi}_{\kappa l}(k)
	\sqrt{4n+2l+3} k\, j_l(\sqrt{4n+2l+3}k) \nonumber \\
	&=  \psi_{\kappa l}(\sqrt{4n+2l+3}) \nonumber \\
	&= \sqrt{4n+2l+3}\, \kappa\, j_l(\sqrt{4n+2l+3}\kappa) \;.
	\end{align}
	%
	Putting it all together, we find
	%
	\begin{align}
	c_n(\kappa) = {2^{1/2-n}(-1)^n \pi^{1/4} \over (4n+2l+3)^{l/2}}
	\sqrt{(2n+2l+1)!\over (n+l)!n!}
	\kappa\, j_l(\sqrt{4n+2l+3}\kappa) \;.
	\end{align}
	%
	We insert this expression for $c_n(\kappa)$ into the quantization
	condition of Eq.~\eqref{quantize} and make the ansatz
	%
	\beq
	\kappa ={x_l\over\sqrt{4n+2l+3+2\Delta}} \ .
	\eeq
	%
	Assuming the limit $n\gg 1$ and $n\gg l$ in the quantization condition
	then yields
	%
	\beq
	\Delta=2 \ .
	\eeq
	%
	Thus, $\Delta$ does not depend on $l$ in this limit, and the result
	is consistent with the $l=0$ result of Ref.~\cite{More:2013rma}. In
	other words, the extent of the position space in finite oscillator
	basis with maximum radial quantum number $n$ and angular momentum $l$ is
	%
	\bea
	\label{L2}
	  L_2 &=& \sqrt{2(2n+l+3/2+2)}b \nonumber\\
	&=& \sqrt{2(N+3/2+2)}b \;,
	\eea
	%
	in accord with Eq.~\eqref{eq:L2_def}.

	Table~\ref{tab:lowest_kappa} shows numerical comparisons for $l=0,1,2$
	and a range of $n$ of the exact minimum momentum $\kappa$ and the
	estimate $x_l/L_2$ (with $x_0=\pi$, $x_1\approx 4.49341$, $x_2\approx
	5.76346$).  The estimates are accurate approximations of the exact
	results even for small $N=2n+l$, but the accuracy decreases somewhat
	with increasing orbital angular momentum. In some practical
	calculations it might thus be of advantage to directly employ the
	numerical results for $L_2$ instead of the approximate analytical
	expression~(Eq.~\eqref{L2}).

	\begin{table}[h]
	\begin{tabular}{c|c|c|c||c|c|c|c||c|c|c|c}
	$l$& $n$ & $\kappa$  &${x_l/ L_2}$ &
	$l$& $n$ & $\kappa$  &${x_l/ L_2}$ &
	$l$& $n$ & $\kappa$  &${x_l/ L_2}$ \\
	\hline
	0 &  0  & 1.2247  & 1.1874   &
	1 &  0  & 1.5811  & 1.4978   &
	2 &  0  & 1.8708  & 1.7378   \\
	0 &  1  & 0.9586  & 0.9472   &
	1 &  1  & 1.2764  & 1.2463   &
	2 &  1  & 1.5423  & 1.4881   \\
	0 &  2  & 0.8163  & 0.8112   &
	1 &  2  & 1.1047  & 1.0898   &
	2 &  2  & 1.3509  & 1.3222   \\
	0  &  3  &  0.7236  &  0.7207  &
	1  &  3  &  0.9892  &  0.9805  &
	2 &    3  &    1.2191  &     1.2018    \\
	0   &  4   &   0.6568   &    0.6551    &
	1 &    4 &     0.9042   &    0.8987  &
	2 &    4  &    1.1207  &     1.1092   \\
	0  &   5   &   0.6058   &    0.6046    &
	1 &    5 &     0.8382   &    0.8344  &
	2 &    5   &   1.0432   &    1.0352    \\
	0  &   6   &   0.5651   &    0.5642    &
	1 &    6 &     0.7850   &    0.7822  &
	2  &   6  &    0.9801  &     0.9742  \\
	0   &  7 &     0.5316    &   0.5310    &
	1  &   7 &     0.7408   &    0.7387  &
	2 &    7  &    0.9274  &     0.9229   \\
	0  &   8    &  0.5035    &   0.5031    &
	1  &   8 &     0.7033   &    0.7018  &
	2 &    8  &    0.8824  &     0.8789   \\
	0  &  9  &    0.4795   &    0.4791     &
	1 &    9 &     0.6711   &    0.6698  &
	2  &   9   &   0.8435  &     0.8407    \\
	0  &   10 &   0.4585  &     0.4582    &
	1 &    10 &   0.6429  &     0.6419   &
	2  &   10   & 0.8093 &      0.8070  \\
	\end{tabular}
	\caption{Comparison of the exact lowest momentum $\kappa$ with the analytical
		estimate $x_l/L_2$ for $l=0, 1, 2$ and $0 \leq n \leq 10$.}
	\label{tab:lowest_kappa}
	\end{table}

	\medskip
	\subsubsection{Energy correction for finite angular momentum}

	Let us extend our $l=0$ result for $[\Delta E]_{\rm LO}$ to $l>0$ following
	the	$S$-matrix method in Subsec.~\ref{subsec:make_cash}.
	For orbital angular momentum $l$, the asymptotic wave function is
	%
	\beq
  u_L(r) \overset{r \gg R}{\longrightarrow}  k_L r\Bigl(h_l^{(1)}(ik_L r) -
	{h_l^{(1)}(ik_L L)\over h_l^{(1)}(-ik_L L)}
	h_l^{(1)}(-ik_Lr)\Bigr) \;.
  \label{eq:uLasympl}
	\eeq
	%
	Here, $h_l^{(1)}$ denotes the spherical Hankel function of the first
	kind (or the spherical Bessel function of the third
	kind)~\cite{abramowitz1964}.  By definition $u_L(L)=0$.

	In complete analogy to the case of $s$ waves (e.g., using
	Eqs.~\eqref{eq:basiceq} and \eqref{eq:purepole} for general $l$), the
	correction $\Delta E$ of the energy at leading order is
	%
	\beq
	[\Delta E]_{\rm LO} = - \kinf \left(\gamma_\infty^{(l)}\right)^2
	{h_l^{(1)}(ik_L L)\over h_l^{(1)}(-ik_L L)} \ .
	\label{eq:ELO_general_l}
	\eeq
	%
	We note that
	\beq
	{h_l^{(1)}(ix)\over h_l^{(1)}(-ix)} \approx - e^{-2x}
	\eeq
	for $x\gg 1$.  In particular, for $l=1$
	%
	\beq
	[\Delta E]_{\rm LO} =\kinf \left(\gamma_\infty^{(1)}\right)^2
	{\kinf L+1\over \kinf L-1} \, e^{-2\kinf L}  \;,
	\label{eq:DeltaE_l1}
	\eeq
	%
	and for $l=2$
	%
	\beq
	[\Delta E]_{\rm LO} = \kinf \left(\gamma_\infty^{(2)}\right)^2
	{(\kinf L)^2 +3 \kinf L +3\over (\kinf L)^2-3\kinf L +3}\, e^{-2\kinf L} \;.
	\label{eq:DeltaE_l2}
	\eeq
	These corrections are tested in Fig.~\ref{fig:sq_well_error_plot_higher_l}.
	%
	\begin{figure}[h]
	\centering
	\includegraphics[width=0.6\textwidth]
	{Extrapolation/sq_well_err_P_D_states}
	\caption{Error plots of the energy correction at each $L$ for (a) $l = 1$ and
	(b) $l = 2$ square-well states predicted at leading order by
	Eqs.~\eqref{eq:DeltaE_l1} and \eqref{eq:DeltaE_l2} compared to the exact
	energy.  Lines proportional to the expected \LNLO~residual errors are plotted
	for comparison.}
	\label{fig:sq_well_error_plot_higher_l}
	\end{figure}
	%
	For coupled channels, the leading energy correction will be the sum
	of the LO corrections for the individual angular momenta.
	We note that lattices with periodic bc lead to energy shifts that
	depend on the angular momentum~\cite{Konig:2011nz}.
	In contrast, the basis truncations we consider in this work are
	variational and thus always yield a positive energy correction.

	\begin{figure}[h]
	\centering
	\includegraphics[width=0.6\textwidth]
	{Extrapolation/deuteron_error_log_plots_sd_v3test}
	\caption{Residual error for the deuteron energy due to
	  HO basis truncation as a function of $L = L_2$ (with $\hw > 49$~MeV
	  to eliminate UV contamination) after subtracting $l=0$ energy
	  corrections at different orders from
	  Eq.~\eqref{eq:complete_E_correction_NLO} and the $l=2$ correction
	  from Eq.~\eqref{eq:DeltaE_l2}.  For the parameter $w_2$ in
	  Eq.~\eqref{eq:complete_E_correction_NLO} we use the value reported
	  in \cite{Phillips:1999hh}.}
	\label{fig:deuteron_error_plot_l2}
	\end{figure}
	%
	We return to the deuteron and take
	$|\gamma_\infty^{(2)}/\gamma_\infty^{(0)}| \approx 0.0226/0.8843$ from
	Ref.~\cite{Machleidt:2011zz}.  Then
	%
	\beq
	[\Delta E]_{\rm LO} =\kinf \left(\gamma_\infty^{(0)}\right)^2
	e^{-2\kinf L}
	  \left[ 1 +
	\left|\frac{\gamma_\infty^{(2)}}{\gamma_\infty^{(0)}}\right|^2
	\frac{(\kinf L)^2 +3 \kinf L +3}{(\kinf L)^2-3\kinf L +3}\right]
	\;.
	\eeq
	%
	This formula is tested in Fig.~\ref{fig:deuteron_error_plot_l2} with
	the same deuteron calculations as in
	Fig.~\ref{fig:deuteron_error_log_plots}.  We note that the deviation
	after subtraction of the NLO ($l=0$) result does not exhibit the
	$\exp(-6k_\infty L)$ falloff but is rather consistent with an
	$\exp(-4\kinf L)$ falloff at large $L$.  We attribute this to the
	missing LO $d$-state correction.  Due to the small value of the
	$d$-to-$s$ state ratio, the $d$-wave correction is small, but it makes
	a perceptible shift of the $s$-wave LO result.  When added to the NLO
	$l=0$ correction, the large $L_2$ behavior of the error is brought
	somewhat closer in line with the predicted dependence of $L^2
	e^{-6\kinf L}$.  We note, however, that the NLO correction is not complete
	due to the missing $l=2$ correction.  Calculating NLO corrections for $l>0$
	remains an open question.  Particularly, because that would entail taking
	into account the admixture between different channels.

	\subsection{Radii and phase shifts}
	\label{subsec:radii_phase_shifts}

	\medskip
	\subsubsection{Radii}

	Along with binding energies, another nuclear observable that we looked at
	was the radius squared.  Figure~\ref{fig:deuteron_radii} shows the
	numerical results for the squared radius for the deuteron calculated in the
	HO basis.  Analogous to Figs.~\ref{fig:spatial_cutoff_vs_HO_square_well} and
	\ref{fig:spatial_cutoff_vs_HO_deuteron}, we see that the results for
	the squared radius fall on a continuous curve with minimal spread when
	plotted as a function of $L_2$ (but not as a function of $L_0$).
	%
	\begin{figure}[h]
	\centering
	\includegraphics[width=0.6\textwidth]
	{Extrapolation/Deuteron_radii_n3lo500_inset_v4}
	\caption{Deuteron radius squared versus $L_0$ (top) and
  	$L_2$ (bottom) for the Entem-Machleidt 500\,MeV N$^3$LO
  	potential~\cite{Entem:2003ft}.  The horizontal dotted lines mark the
  	exact radius squared $r^2_{\infty} = 3.9006~{\rm fm}^2$. The insets
  	show a magnification of data at smaller lengths $L_n$.}
	\label{fig:deuteron_radii}
	\end{figure}
  %

	Though the squared radius is a long-ranged
	operator, its matrix elements will still be modified at short distances
	by renormalizations or  similarity transformations of the Hamiltonian,
	see, e.g., Ref.~\cite{Stetcu:2004wh}.  Thus we cannot expect an extrapolation
	law for the radius that depends entirely on observables.
	Instead, we seek a formula that identifies the $L$ dependence but leaves
	parameters to be fit.  We define
	%
	\beq
	  \rsqav_L = \rsqinf + \Delta\rsqav_L \;,
	\eeq
	%
	where
	%
	\beq
	  \Delta\rsqav_L =
	  \frac{\int_0^L |u_L(r)|^2\, r^2\,dr}{\int_0^L |u_L(r)|^2\, dr}
	  - \frac{\int_0^\infty |u_\infty(r)|^2\, r^2\,dr}{\int_0^\infty
		|u_\infty(r)|^2\, dr} \;.
	  \label{eq:Delta-rsq}
	\eeq
	%
	The strategy is to isolate the polynomial $L$ dependence by splitting the
	necessary integrals into an interior part and an exterior part:
	\beq
  \int_0^L\! r^n |u_L(r)|^2\, dr =
  \int_0^R\! r^n |u_L(r)|^2\, dr + \int_R^L\! r^n |u_L(r)|^2\, dr \;,
  \label{eq:rnintegral}
	\eeq
	%
	where $R$ is sufficiently large so that the asymptotic form of $u_L(r)$ from
	Eq.~\eqref{eq:uLasymp2} can be used in the second integral.
	Our expression for $\Delta\rsqav_L$ is independent of the normalization
	of $u_L(r)$, so we are free to choose it so that the large $r$ form is
	exactly given by Eq.~\eqref{eq:uLasymp2}.

	The first integral in Eq.~\eqref{eq:rnintegral} will depend on the details
	of the interior wave function	and therefore on the potential, but
	the linear energy method shows us that to $\mathcal{O}(e^{-2\kinf L})$
	the $L$ dependence is isolated.  In particular, the dependence on $L$ of
	$u_L(r)$ in	Eq.~\eqref{eq:linear_energy_approx} is confined to
	$\Delta E_L = \kinf\ANC^2 e^{-2\kinf L}$ because $du_E(r)/dE|_{\Einf}$ for
	$r < R$ is independent of $L$ with our choice of normalization.  Thus the
	integral over $r$ cannot introduce polynomial $L$ dependence and we can
	conclude that
	%
	\beq
  \int_0^R\! r^n |u_L(r)|^2\, dr = \mathcal{O}(L^0) e^{-2\kinf L}
	     + \mathcal{O}(e^{-4\kinf L}) \;.
	\eeq
	%
	The $\mathcal{O}(L^0)$ coefficient will depend on the potential, so we
	will treat it as a parameter to be fit.

	The second integral can be directly evaluated to $\mathcal{O}(e^{-2\kinf L})$
	using Eq.~\eqref{eq:uLasymp2} and $[k_L]_{LO} = \kinf - \ANC^2 e^{-2\kinf L}$
	to expand $|u_L(r)|^2$.  For $n=0$ we find
	%
	\begin{align}
	  \int_R^L\! &|u_L(r)|^2\, dr =  \frac{1}{2\kinf}e^{-2\kinf R} + \Bigl[
	  \frac{\ANC^2}{\kinf} \Bigl(R + \frac{1}{2\kinf}\Bigr)e^{-2\kinf R} + 2R -
		2L \Bigr]e^{-2\kinf L} + \mathcal{O}(e^{-4\kinf L}) \;,
	  \label{eq:asympnorm}
	\end{align}
	%
	and for $n=2$ we find
	%
	\begin{align}
  	\int_R^L\! r^2 |u_L(r)|^2\, dr =
  	\frac{1}{2\kinf^3}\Bigl[ \frac12 + \kinf R + (\kinf R)^2 \Bigr]
		e^{-2\kinf R}
    \nonumber \\
		\null +  \Bigl[ \frac{\ANC^2}{\kinf^4} \Bigl( \frac34 + \frac32\kinf R +
		\frac32(\kinf R)^2 + (\kinf R)^3 \Bigr)e^{-2\kinf R}  \nonumber \\
    \null + \frac{1}{\kinf^3} \Bigl( \frac23 (\kinf R)^3 - \kinf L - \frac23
		(\kinf L)^3 \Bigr) \Bigr]e^{-2\kinf L}
    % \nonumber \\
		+ \mathcal{O}(e^{-4\kinf L})  \;.
    \label{eq:asymprsq}
	\end{align}
	%
	Note that it is necessary to keep the expansion of $|u_L(r)|^2$ up to
	$e^{-4\kinf L}$	until after doing the integrals because terms proportional
	to $e^{-4\kinf L} e^{2\kinf r}$ will be leading order.

	When we use Eqs.~\eqref{eq:asympnorm} and \eqref{eq:asymprsq} and our previous
	result for the interior integrals in Eq.~\eqref{eq:Delta-rsq}, expanding
	consistently to $\mathcal{O}(e^{-2\kinf L})$, we will mix $R$-dependent
	terms with the $L$ dependence.  However, we can immediately conclude that
	the general form to this order is (with $\beta \equiv 2\kinf L$)
	%
	\bea
	 	\rsqav_L \approx {\rsqinf}[ 1 - (c_0 \beta^3 +c_1 \beta + c_2) e^{-\beta}]
	  \;.
		\label{rad}
	\eea
	%
	Here, $\rsqinf$, $c_0$, $c_1$, and $c_2$ are fit parameters while $\kinf$
	should be determined from fitting the energy. 	This form has been verified
	explicitly for finite-range model potentials (e.g., square well
	and delta shell). 	The approximation in Eq.~\eqref{rad} should be valid
	in the asymptotic regime $\beta\gg 1$.  In practice, one needs
	$\beta\gtrsim 3$ so that the dominant $\beta^3$ correction is
	approximately an order of magnitude larger than the subleading terms
	(with $c_1$ and $c_2$ expected to be roughly the same size as
	$c_0$ or smaller).

	If we take the zero-range limit $R\to 0$ of the potential, we
	arrive at the simple expression
	%
	\beq
	 \frac{\Delta\rsqav_L}{\rsqinf}
	 \approx -\left(\frac{(2\kinf L)^3}{3} -4\right) e^{-2\kinf L} \;.
	 \label{eq:rsqzerorange}
	\eeq
	%
	Note that in this limit the correction becomes independent of the potential.
	Equation~\eqref{eq:rsqzerorange} suggests that for a short-range potential,
	the $c_1$ and $c_2$ terms will give comparable contributions for moderate
	$\beta$, and therefore will be difficult to determine reliably.

	\begin{figure}[h]
		\centering
		\begin{subfigure}[c]{0.45\textwidth}
			\centering
			\includegraphics[width=\textwidth]
			{Extrapolation/H2_radius_sq_vs_L9to15}
			\caption{Fit region: $L$ from $9$ to $15$ fm.}
			\label{fig:deuteron_radius_fit_9to15}
		\end{subfigure}
		\hspace{0.05\textwidth}
		\begin{subfigure}[c]{0.45\textwidth}
			\centering
			\includegraphics[width=\textwidth]
			{Extrapolation/H2_radius_sq_vs_L12to17}
			\caption{Fit region: $L$ from $12$ to $17$ fm.}
			\label{fig:deuteron_radius_fit_12to17}
		\end{subfigure}
		\caption{Deuteron radius squared versus $L_2$ for the chiral
		  N$^3$LO (500\,MeV) potential of Ref.~\cite{Entem:2003ft}.  To
			eliminate the UV contamination we only plot results for $\hw > 49$~MeV.
			The solid, dot-dashed, and dashed lines are results from fitting
			Eq.~\eqref{rad} in the shaded region to find $\rsqinf$ and one,
			two, or three of the $c_i$ constants, respectively.
  		The horizontal dotted line is the deuteron radius squared.}
		\label{fig:deuteron_radius_fit}
	\end{figure}
	%
	Sample fits of Eq.~\eqref{rad} for the deuteron are shown in
	Fig.~\ref{fig:deuteron_radius_fit}.  Results are obtained for fitting
	one, two, and all three $c_i$ constants to radii calculated
	with the same truncated oscillator basis parameters used for
	Fig.~\ref{fig:deuteron_curves}.  The fit region is for $L_2$
	between 9 and 15\,fm, where the calculations only show a small
	amount of curvature and between 12 and 17\,fm.  All points are equally
	weighted.
	%
	\begin{table}[h]
		\begin{tabular}{ |c|c|c|c|c| }
			\hline
	 		\multicolumn{5}{|c|}{fit region: $L$ from $9$ to $15$ fm. Exact:
	 		$\rsqinf = 3.901$ fm$^2$, $\kinf = 0.232$ fm$^{-1}$} \\ \hline
			& $\rsqinf$ & $c_0$ & $c_1$ & $c_2$ \\ \hline
			$1$ term & $3.945 \pm 0.006$ & $0.331 \pm 0.002$ & -- & -- \\ %\hline
	 		$2$ terms & $3.865 \pm 0.001$ & $0.266 \pm 0.001$ & $1.131 \pm 0.018$ & --
			\\ %\hline
	 		$3$ terms & $3.887 \pm 0.001$ & $0.308 \pm 0.002$ & $-1.239 \pm 0.135$ &
			$7.167 \pm 0.408$ \\ \hline
	 \end{tabular}
	 \caption{Coefficients from fitting Eq.~\eqref{rad} to the deuteron data
	    using one, two, or three $c_i$ constants as shown in
			Fig.~\ref{fig:deuteron_radius_fit_9to15}}
	 \label{tab:coeff_deuteron_radius_fit_9to15}
  \end{table}
	%
	\begin{table}
		\begin{tabular}{ |c|c|c|c|c| }
	  \hline
	  \multicolumn{5}{|c|}{fit region: $L$ from $12$ to $17$ fm. Exact:
		$\rsqinf = 3.901$ fm$^2$, $\kinf = 0.232$ fm$^{-1}$} \\ \hline
	  & $\rsqinf$ & $c_0$ & $c_1$ & $c_2$ \\ \hline
	  $1$ term & $3.899 \pm 0.001$ & $0.312 \pm 0.000$ & -- & -- \\ %\hline
	  $2$ terms & $3.888 \pm 0.001$ & $0.293 \pm 0.001$ & $0.503 \pm 0.026$ & --
		\\ %\hline
	  $3$ terms & $3.898 \pm 0.001$ & $0.339 \pm 0.006$ & $-3.577 \pm 0.508$ &
		$15.339 \pm 1.908$ \\ \hline
	  \end{tabular}
		\caption{Coefficients from fitting Eq.~\eqref{rad} to the deuteron data
 	    using one, two, or three $c_i$ constants as shown in
 			Fig.~\ref{fig:deuteron_radius_fit_12to17}}
		\label{tab:coeff_deuteron_radius_fit_12to17}
	\end{table}
	%
	The values for $\rsqinf$ and the coefficients $c_i$'s from the fits in
	Fig.~\ref{fig:deuteron_radius_fit} are reported in
	Tables~\ref{tab:coeff_deuteron_radius_fit_9to15} and
	\ref{tab:coeff_deuteron_radius_fit_12to17}.
	For all of these fits, the value of $c_0$ is
	fairly stable, ranging from 0.27 to 0.33 (note that
	$c_0 = 1/3$ in the zero-range limit).  In contrast, $c_1$
	and $c_2$ are not well determined (even the sign of $c_1$
	varies).  This is consistent with
	fits using the square-well potential, where analytic expressions
	for the $c_i$s can be found.  We find that $\rsqinf$
	and $c_0$ are well determined by fits in analogous
	regions but that $c_1$ and $c_2$ are not.
	If we push the analysis by taking the fit region between 7 and 13\,fm,
	the $\rsqinf$ prediction using only $c_0$ breaks down, giving
	4.21\,fm$^2$.  However, the fit with all three $c_i$s is still
	reasonable, giving 3.86\,fm$^2$.  This indicates the importance of using
	the correct functional form for extrapolation.
	Further studies are needed to test how these trends might carry
	over to $A>2$ nuclei.  It is worthwhile to note that this approach can
	be generalized to any coordinate space operator.

	\medskip
	\subsubsection{Phase shifts}

	The argument for computing scattering phase shifts is as
	follows: The oscillator basis appears as a spherical box of size
	$L$.  For low momenta we have $L=L_2$, but at higher momentum $L$
	deviates slightly from $L_2$, and can be determined from the
	eigenvalues of the operator $p^2$.  Thus, the positive-energy states
	computed in the oscillator basis can be used to extract phase shifts.

	In a fixed harmonic oscillator basis ($N,\hbar\Omega$), the
	computation of the phase shifts for a given partial wave $^{2S+1}l_J$
	with orbital angular momentum $l$ proceeds as follows: First, one
	computes the discrete eigenvalues $p_i^2$ of the operator $p^2$ for
	orbital angular momentum $l$.  Second, we need to determine the
	momentum dependent box size $L_i=L(p_i)$.  Assuming that the $i^{\rm th}$
	momentum eigenstate is the $i^{\rm th}$ eigenstate of a
	spherical box, we must determine the $i^{\rm th}$ zero of the
	spherical Bessel function.  Thus $j_l(p_iL_i/\hbar) = 0 $ determines
	$L(p_i)$.  We evaluate the smooth function $ L(p)$ for arbitrary
	momentum $p$ by interpolating between the discrete momenta
	$p_i$.  Third, we compute the discrete positive energies $E_i =
	\hbar^2k_i^2/(2m)$ of the neutron-proton system in relative
	coordinates for the partial wave $^{2S+1}l_J$, and compute the phase
	shifts from the Dirichlet boundary condition at $r=L$, i.e.
	%
	\beq
	\tan\delta_l(k_i) = { j_l(k_iL(\hbar k_i)) \over \eta_l(k_iL(\hbar k_i))}\;.
	\eeq
	%
	Here $\eta_l$ is the spherical Neumann function.  In practice one
	repeats this procedure for several values of $\hbar\Omega$ in order to
	get sufficiently many datapoints that fall onto a smooth curve.
	Note that for $E_i$ or $k_i$ are obtained from diagonalizing the nuclear
	Hamiltonian in HO basis whereas $p_i$ are obtained from diagonalizing the
	momentum squared operator.

	\begin{figure}[h]
	\centering
	\includegraphics[width=0.6\textwidth]
	{Extrapolation/1s0_phaseshift_v3}
	\caption{The $^1$S$_0$ phase shifts (in degrees) of the N$^3$LO
	  chiral interaction (solid line) compared to the phase shifts
	  computed directly in the harmonic oscillator basis (circles).}
	\label{fig:1s0phase}
	\end{figure}
	%
	\begin{figure}[h]
	\centering
	\includegraphics[width=0.6\textwidth]
	{Extrapolation/3p1_phaseshift_v2}
	\caption{The $^3$P$_1$ phase shifts (in degrees) of the N$^3$LO
	  chiral interaction (solid line) compared to the phase shifts
	  computed directly in the harmonic oscillator basis (circles).}
	\label{fig:3p1phase}
	\end{figure}
	%
	As examples we compute the scattering phase shifts for the $^1$S$_0$ and
	$^3$P$_1$	partial waves in model spaces with $N=32$ and
	$\hbar\Omega=20,22,\ldots,40$~MeV.  Our calculations are based on the
	Entem-Machleidt 500\,MeV chiral EFT N$^3$LO
	potential~\cite{Entem:2003ft}.  Figures~\ref{fig:1s0phase} and
	\ref{fig:3p1phase} show the results and compares them to the numerically
	exact phase	shifts.  For smaller $N$ than our current choice, the
	computed phase shifts start to deviate from exact phase shifts at
	higher energies.  However, if one is interested only in low-energy
	phase shifts and observables such as the scattering length and the
	effective range, a smaller harmonic oscillator basis is sufficient.

	There are other methods to compute scattering phase shifts in the
	harmonic oscillator basis.  Bang \emph{et al.} \cite{bang2000} used the
	method of harmonic oscillator representation of scattering equations
	(HORSE) for this purpose, and more recent
	works~\cite{Luu:2010hw,Stetcu:2009ic} computed phase shifts to develop an
	EFT for nuclear interactions directly in the oscillator
	basis~\cite{Stetcu:2006ey}.  References~\cite{Luu:2010hw,Stetcu:2009ic}
	build on the results by Busch {\it et al.}~\cite{busch1998} and their
	generalization~\cite{Bhattacharyya:2006fg} to finite range corrections,
	and extract scattering information from the energy shifts of bound
	states in a harmonic oscillator potential.  The resulting EFTs are
	quite efficient for contact interactions and systems such as ultracold
	trapped fermions, but nuclear potentials with a finite range require
	an extrapolation of $\Omega\to 0$~\cite{Luu:2010hw}.  The approach
	presented in this Subsection is more direct, as no external oscillator
	potential is employed.  We note that the phase shift analysis presented
	here can be extended to coupled channels as well.

	Finally, we note again that the approach of this Section can be
	utilized in other localized basis sets.  All that is required is the
	diagonalization of the operator $p^2$ in the employed basis set, which
	yields the (momentum dependent) box size.


	\section[Ultraviolet story]
	{Ultraviolet story \footnote{Based on \cite{Konig:2014hma}}}
	\label{sec:UV_story}

  In Sec.~\ref{sec:IR_story} we worked in the region where the UV errors were
	small and focused on the IR errors.  However, for many methods full
	suppression of the UV errors is not feasible and the need to understand
	UV corrections remains.  In all cases the UV effect is a systematic error
	that must be quantified.  In addition, this error worsens for harder
	nucleon--nucleon potentials	that may still be of interest.
	The understanding of the UV errors formed the focus of our work in
	Ref.~\cite{Konig:2014hma}.  The author of this thesis contributed to the work
	presented in \cite{Konig:2014hma}.  However, the UV effort was mainly
	spearheaded by Sebastian K\"{o}nig, the post-doc in the group.  To avoid
	misappropriation of credit, only a summary of results from
	\cite{Konig:2014hma} will be presented in this section.
	The summary presented here hopes to elucidate how the UV results tie into
	our broader agenda of developing reliable extrapolations schemes for nuclear
	calculations.  We refer the reader to \cite{Konig:2014hma} for the additional
	UV extrapolation details.

	We follow the strategy of Sec.~\ref{sec:IR_story} by focusing on the
	two-body problem and exactly solvable examples to establish the true
	UV behavior for the	simple systems and then make correspondence to the
	nuclear systems.

	\subsection{Duality and momentum-space boxes}
	\label{subsec:UV_cutoff_duality}

	To briefly recap results in Subsec.~\ref{subsec:tale_of_tails}---we
	demonstrated there that a
	truncated oscillator basis with highest excitation energy
	$N\Omega$ effectively imposes a spherical hard-wall boundary condition
	at a radius depending on $N$ and $b$.  The optimal effective radius
	$\Leff$ can be determined by matching the smallest eigenvalue
	$\kappa^2$ of the squared momentum operator $p^2$ in the finite
	basis to the corresponding eigenvalue of the spherical box, namely
	$\kappa=\pi/L$ (for $\ell=0$).  The value can be established numerically,
	but an accurate approximation for the two-body system is
	%
	\begin{equation}
	\label{eq:L2_def}
	  \Leff = L_2\equiv\sqrt{2(N+3/2+2)}b \,.
	\end{equation}
	%
	Note that $L_2$ differs by $\mathcal{O}(1/N)$ from the naive estimate
	$L_0\equiv\sqrt{2(N+3/2)}b$.  In localized bases that differ from the
	harmonic oscillator, $L$ can also be determined from a numerical
	diagonalization of the operator $p^2$.

	The dual nature of the harmonic oscillator Hamiltonian
	\begin{equation}
	 H_\mathrm{HO} = \frac{p^2}{2\mu} + \frac{\mu\Omega^2r^2}{2}
	\label{eq:H-HO}
	\end{equation}
	(\ie, under $p \leftrightarrow \mu\Omega r$) implies that the
	truncation of the basis will effectively impose a sharp cutoff at a
	momentum $\Lameff$ depending only on $N$ and $b$.  The analog matching
	condition leads us to consider the smallest eigenvalue (denoted $\rho$) of
	the operator $r^2$ evaluated in that truncated basis.  This eigenvalue is
	identical to the smallest (squared) distance that can be realized in the
	oscillator basis.  Thus it corresponds to a lattice spacing on a grid
	and therefore sets the highest momentum available.  As we see
	in Fig.~\ref{fig:deuteron_lambdas}, the square root of the largest
	eigenvalue of the squared momentum operator, which might be a natural guess
	for the effective UV cutoff, is not an accurate estimate for $\Lameff$.
	From steps completely analogous (dual) to those given in
	Subsec.~\ref{subsec:tale_of_tails} for the IR case, we find that
	the solution (in a subspace with fixed angular momentum $\ell$) is
	%
	\begin{equation}
	\rho = \frac{x_\ell b}{\sqrt2}\left(\Nmax+\frac32+\Delta\right)^{\!-1/2}
	\label{eq:rho-Nmax-Delta}
	\end{equation}
	%
	with $\Delta=2$ to leading order.  The constant $x_\ell$ in the prefactor is
	the first positive zero of the spherical Bessel function $j_\ell$.  Since the
	UV cutoff is given by $x_\ell/\rho$, it drops out again in our final
	result:
	%
	\begin{equation}
	\Lambda_2 \equiv \sqrt{2(\Nmax+3/2+2)}/b \,.
	\label{eq:Lambda-2-simple}
	\end{equation}
	%
	Hence, we have shown that the proper effective UV cutoff imposed by
	the basis truncation is given by $\Lambda_2$, which differs by a correction
	term from the naive estimate
	%
	\begin{equation}
	 \Lambda_0 \equiv \sqrt{2(\Nmax+3/2)}/b
	\label{eq:Lambda-0}
	\end{equation}
	%
	that one obtains by simply considering the maximum single-particle energy
	level represented by the truncated basis.  We note that subleading
	corrections	to $\Delta=2$, which by duality apply equally to the IR and UV
	cutoff, are	derived in Appendix of Ref.~\cite{Konig:2014hma}.

	\begin{figure}[thbp]
	\centering
	\includegraphics[width=0.6\textwidth]
	{Extrapolation/Lambda0_Lambda2_kmax}
	\caption{Relative error of deuteron binding energy
	  plotted vs. lengths $\Lambda_2$, $\Lambda_0$, and $\Lambda_{\kappa,
	  {\rm max}}$ (multiplied by factors 2, 1, and $1/2$, respectively,
	  to separate the curves.  Inset: The same values on a linear
	  scale and without the separation factors.}
	\label{fig:deuteron_lambdas}
	\end{figure}
	Fig.~\ref{fig:deuteron_lambdas} shows the relative error when plotted
	against three cutoff variables, $\Lambda_2$, $\Lambda_0$, and
	$\Lambda_{\kappa_{\rm max}}$.
	The	calculations use the 500\,MeV N$^3$LO nucleon-nucleon $NN$ potential
	of Ref.~\cite{Entem:2003ft}, evolved by the	SRG~\cite{Bogner:2006pc} to
	$\lambda = 2\,\fmi$.
	$\Lambda_{\kappa_{\rm max}}$ is defined as the square root
	of the largest eigenvalue of the squared momentum operator in the finite
	oscillator basis, which one might naively expect to be a natural choice.
	However, of the cases considered this actually gives the largest scatter
	in data.  From the fact that we get an essentially
	smooth curve  only for $\Lambda_2$, we conclude that this identification of
	the	relevant UV cutoff is correct.
	%
	\begin{figure}[thbp]
	\centering
	\includegraphics[width=0.6\textwidth]%
	{Extrapolation/deuteron_deltaE_vs_cutLamT_HO_Lam2_no_inset}
	\caption{Calculations of the relative error in the
	   deuteron energy as a function of $\Lambda_2(\Nmax,\hw)$.  Circles
	   represent a wide range of oscillator parameters $\Nmax$ and
	   $\hw$ that are IR converged.  The series of lines shows energies for
	   which the Hamiltonian has been smoothly cutoff with exponent $n$.
	   The solid line corresponds to a sharp cutoff.}
	\label{fig:deuteron_vs_cutLam}
	\end{figure}
	%
	In Fig.~\ref{fig:deuteron_vs_cutLam}, deuteron calculations are plotted as
	a function of
	$\Lambda = \Lambda_2(\Nmax,\Omega)$ along with several other functions
	of $\Lambda$ given by the relative error from the same Hamiltonian,
	but now smoothly cut off as
	%
	\begin{equation}
	 H_{\rm cut}(k,k') = \ee^{-(k^2/\Lambda^2)^n} H(k,k')
	 \ee^{-(k'{}^2/\Lambda^2)^n} \,,
	\end{equation}
	%
	for $n=2,4,8$ and $\infty$.  The latter corresponds to a sharp cutoff.
	We find that the curve from a sharp cutoff tracks the
	truncated-oscillator points through many orders of magnitude.
	This validates the claim that the error from oscillator basis
	truncation is well reproduced by applying instead a sharp
	cutoff in momentum at $\Lambda_2$.


	\subsection{Separable approximations}
	\label{subsec:Separable_approximation}

	We showed in Ref.~\cite{Konig:2014hma} that for a separable interaction of
	the form
	\begin{equation}
	V(k',k) = g\,\eta(k') \eta(k) \,,
	\label{eq:pot-RegContact}
	\end{equation}
	UV energy correction formula can be exactly derived.  For
	potential in Eq.~\eqref{eq:pot-RegContact}, the cutoff dependent
	binding momentum $\kappa_\lambda$ is given by the quantization condition
	\begin{equation}
	 {-}1 = 4\pi a \int_0^\Lambda\!\dd k\,
	 \frac{k^2\,\eta^2_\lambda(k)}{\kappa^2_\Lambda+k^2} \,,
	\label{eq:quant-sep}
	\end{equation}
	which is straightforward to solve numerically.

	However, most	interactions used in practical calculations do not have this
	convenient simple form (at least not in nuclear physics).  Still,
	as shown in Ref.~\cite{Konig:2014hma},	Eq.~\eqref{eq:quant-sep} can be put
	to some use using separable approximations.
	Methods to obtain separable approximations for a given potential have
	been known and used for quite a while (see, \eg,
	Refs.~\cite{Harms:1970hd,Ernst:1973zzb,elgaroy1998}).
	The technique we use is called \emph{unitary pole approximation
	(UPA)}~\cite{Ernst:1973zzb,Lovelace:1964aa}.  Assuming that for an
	arbitrary potential $\hat{V}$ we know a (bound) eigenstate $\ket{\psi}$, we
	can	construct a rank-1 separable approximation in momentum space by setting
	%
	\begin{equation}
	 \hat{V}_\text{sep} =
	 \frac{\ket{\eta}\bra{\eta}}
	 {\mbraket{\psi}{\hat{V}}{\psi}} =
	 \frac{\hat{V}\ket{\psi}\bra{\psi}\hat{V}}
	 {\mbraket{\psi}{\hat{V}}{\psi}} \,.
	\label{eq:UPA-op}
	\end{equation}
	%
	In other words, we have
	%
	\begin{equation}
	 \eta(k) = \mbraket{k}{\hat{V}}{\psi}
	 \label{eq:form_factor_def}
	\end{equation}
	%
	for the momentum-space ``form factor,'' and the coupling strength
	$g=\mbraket{\psi}{\hat{V}}{\psi}$ is, of course, independent of any
	particular representation.  From Eq.~\eqref{eq:UPA-op} one immediately
	sees that
	%
	\begin{equation}
	 \hat{V}_\text{sep}\ket{\psi} = \hat{V}\ket{\psi} \,.
	\label{eq:Vsep-psi-V-psi}
	\end{equation}
	%
	This means that the separable approximation is constructed in such a
	way that it exactly reproduces the state $\ket{\psi}$ used for its
	construction.  The potential from Eq.~\eqref{eq:UPA-op} reproduces the exact
	half off-shell T-matrix at the energy corresponding to the state
	$\psi$, and more sophisticated approximations (separable potential of
	rank $>1$) can be constructed by using more than a single
	state~\cite{Ernst:1973zzb}.  Since we are only interested in
	performing the UV extrapolation for a single state, however, the
	rank-1 approximation is sufficient.  To assess
	to what extent it actually reflects the UV behavior of a calculation
	based on the \emph{original} potential, we first considered some
	examples where the separable approximation can be constructed
	analytically such as the square well (Eq.~\eqref{eq:Vsw}) and the
	P\"{o}schl-Teller potential of the form
	%
	\begin{equation}
	\VPT(r) = -\frac{\alpha^2 \beta(\beta-1)}{\cosh^2(\alpha r)}\;.
	\label{eq:V-PT}
	\end{equation}
	%
	For given values of $\alpha$ and $\beta$, this potential has an
	analytically known bound-state spectrum.  Motivated by the success of the
	separable approximation (Eq.~\eqref{eq:UPA-op}) for the toy models,
	we moved on to the deuteron.  Here we will just look at a few representative
	results for the deuteron.

	A difficulty in applying the separable approximation directly to the deuteron
	is that the form factor $n(k)$ in Eq.~\eqref{eq:form_factor_def} depends on
	the deuteron wave function.  The exact wave function of course can not be
	calculated due to the truncation in the HO basis.  We use the best wave
	function available from the largest oscillator space and set
	\beq
	\eta(k) = \mbraket{k}{\hat{V}}{\psi}_{\rm HO,~best}\;.
	\label{eq:eta_best}
	\eeq
	%
	As we know, deuteron has both $S$- and $D$-wave components.  This is taken
	into account by letting $\eta \to \eta_S ^2 + \eta_D ^2$.

	As derived in Ref.~\cite{Konig:2014hma}, the simplest fit formula inspired
	by separable approximation is
	\begin{equation}
	 \kappa_\Lambda = \kappa_\infty
	  - A \int\nolimits_\Lambda^\infty \dd k\,\eta(k)^2 \,,
	\label{eq:fit-eta-simple}
	\end{equation}
	%
	In Figs.~\ref{fig:deuteron_sep_fit_av18}, \ref{fig:deuteron_sep_fit_Epelbaum},
	and \ref{fig:deuteron_sep_fit_Epelbaum_evolved} we show the results
	obtained for the deuteron from fitting to Eq.~\eqref{eq:fit-eta-simple}.
	We compare the result for separable fit to two phenomenological choices.
	The exponential fit
	\beq
	\kappa_\Lambda = \kappa_\infty - a \, \ee^{-b \Lambda} \;,
	\label{eq:exp_fit_mom_UV}
	\eeq
	and the gaussian fit
	\beq
	\kappa_\Lambda = \kappa_\infty - a \, \ee^{-b \Lambda^2} \;.
	\label{eq:gauss_fit_mom_UV}
	\eeq
	%
	\begin{figure}[thbp]
	\centering
	\includegraphics[width=0.6\textwidth]%
	{Extrapolation/Deut-AV18_bare-10-24-32}
	\caption{Comparison of UV extrapolations for a deuteron state
   calculated with the AV18 potential of Ref.~\cite{Wiringa:1994wb}.
   Circles: oscillator results.
   Dotted line: exponential extrapolation (Eq.~\eqref{eq:exp_fit_mom_UV}).
   Dashed line: Gaussian extrapolation (Eq.~\eqref{eq:gauss_fit_mom_UV}).
   Solid line: simplest separable extrapolation (Eq.~\eqref{eq:fit-eta-simple}).
   Dotted horizontal lines indicate the exact result for the binding
   momentum.}
	\label{fig:deuteron_sep_fit_av18}
	\end{figure}
	%
	\begin{figure}[thbp]
	\centering
	\includegraphics[width=0.6\textwidth]%
	{Extrapolation/Deut-EGM550_bare-9-14-20}
	\caption{Calculations of UV extrapolations for a deuteron state
	   calculated with the Epelbaum~\etal N3LO ($550$/$600~\MeV$ cutoff)
	   potential of Ref.~\cite{Epelbaum:2004fk}.  The legend description is the
	   same as in Fig.~\ref{fig:deuteron_sep_fit_av18}.}
	\label{fig:deuteron_sep_fit_Epelbaum}
	\end{figure}
	%
	\begin{figure}[thbp]
	\centering
	\includegraphics[width=0.6\textwidth]%
	{Extrapolation/Deut-EGM550_srg-4-4-12}
	\caption{Calculations of UV extrapolations for a deuteron state
		 calculated with the Epelbaum~\etal N3LO ($550$/$600~\MeV$ cutoff)
		 potential of Ref.~\cite{Epelbaum:2004fk}.  The legend description is the
		 same as in Fig.~\ref{fig:deuteron_sep_fit_av18}.}
	\label{fig:deuteron_sep_fit_Epelbaum_evolved}
	\end{figure}
	%
	We see from Figs.~\ref{fig:deuteron_sep_fit_av18},
	\ref{fig:deuteron_sep_fit_Epelbaum}, and
	\ref{fig:deuteron_sep_fit_Epelbaum_evolved} that the fit from the
	separable approximation is superior to phenomenological fits.
	The separable approximation allows extrapolation even when we are far from
	convergence (this is especially evident in
	Fig.~\ref{fig:deuteron_sep_fit_av18}).  It is also worthwhile to note
	that the separable fit (Eq.~\eqref{eq:fit-eta-simple}) has just
	two fit parameters $\kappainf$ and $A$, whereas the phenomenological
	fits have three free parameters $\kappainf$, $a$, and $b$.  The reason
	that the separable fit does well is that it puts in the information
	we already know about the state through Eq.~\eqref{eq:eta_best}.

	The IR and UV corrections exhibit a complementary mix of universal and
	non-universal characteristics.  The IR corrections are dictated by
	asymptotic behavior and are consequently determined by observables,
	independent of the details of the interaction.  So unitarily
	equivalent potentials---such as those	generated by renormalization-group
	running---will have the same corrections.
	In contrast, because they probe short-range features, UV corrections
	depend on	the details of the interaction (and the state under
	consideration).  However, the IR correction depends on the number of
	nucleons, whereas as we will see in Subsec.~\ref{subsec:UV_front}
	the UV correction is expected to scale simply with the number of particles.


	\section{Moving forward and related developments}

  In this section, we will list the open questions with respect to both IR
	and UV extrapolations.  As mentioned before, this approach of mapping the
	HO truncation to IR and UV cutoffs and using them to obtain physically
	motivated extrapolation formulas was rigorously developed for the first
	time by us \cite{More:2013rma,Furnstahl:2013vda,Konig:2014hma}.  This
	pioneering work has sprouted many new developments by extending our work.
	We will briefly touch upon some of these related developments.

	\subsection{IR front}
	\label{subsec:IR_front}

  \medskip
	\subsubsection{Open questions}

	As discussed towards the end of Subsec.~\ref{subsec:higher_angular_momenta},
	the NLO IR correction is incomplete due to the missing $l=2$ correction.
	It might be challenging to derive NLO corrections to the binding
	energies for nuclei with $A> 2$, particular for nuclei with nonzero
	ground-state spin.  Here, many different orbital anglar momenta can
	contribute to the ground-state wave function, and one would presumably
	need to know the admixture of the different channels quite accurately.
	Our results show that nonzero orbital angular momenta yield
	corrections in inverse powers of $\kinf L$ to the LO energy
	extrapolation.  On the other hand, the leading contributions to
	bound-state energies in finite model spaces fall off as $\exp{(-2\kinf
	L)}$ for all orbital angular momenta.  This makes extrapolations
	feasible in practice.

	The formulation in terms of S-matrix analytic structure is closely
	related to methods used to analyze break-up reactions, which provides a
	link to $A>2$ extrapolations.  Indeed, in Ref.~\cite{Furnstahl:2012qg}
	the basic form of the LO extrapolation proportional to $e^{-2\kinf L}$
	was based on interpreting $\kinf$ in terms of the one-particle
	separation energy.  More generally, the asymptotic many-body wave
	function is dominated by configurations corresponding to the break-up
	channels with the lowest separation energies and it is
	their modification by the hard wall at $L$ that will be associated
	with the energy shift $\Delta E_L$.  This is in turn dominantly
	described by the S-matrix near poles at the corresponding separation
	binding momenta.  Future work will seek to clarify the precise nature
	of the more general expansion (including the effects of the Coulomb
	interaction) and whether it will be possible to quantitatively extract
	asymptotic normalization constants.

	\paragraph{Relation to L\"{u}scher-type formulas}

	We saw in Subsec.~\ref{subsec:lattice_theories} that lattice theories
	have an inherent IR and UV cutoff.
	Starting with the seminal work of L\"uscher~\cite{Luscher:1985dn}, a
	wide variety of formulas have been derived for the energy shift of
	bound states in finite-volume lattice calculations.  The usual
	application is to simulations that use periodic boundary conditions in
	cubic boxes (e.g., see Ref.~\cite{Konig:2011ti}).  The recent work by
	Pine and Lee~\cite{Lee:2010km,Pine:2012zv} extend the derivation to
	hard-wall boundary conditions using effective field theory for
	zero-range interactions and the method of images.  The result for
	$\Delta E_L$ in a three-dimensional cubic box has a different
	functional form than found here (the leading exponential is multiplied
	by $1/L$ with that geometry) and the subleading corrections are
	parametrically larger.

	However, because the HO truncation we consider is in partial waves,
	the one-dimensional analysis and formula from Ref.~\cite{Pine:2012zv}
	are applicable (because $\kinf$ and $\ANC$ are asymptotic quantities,
	the result for zero-range interaction is actually general for
	short-range interactions).  The method of images can be applied in a
	one-dimensional box of size $2L$ after specializing to a particular
	partial wave and then extending the space to odd solutions in $r$ from
	$-\infty$ to $+\infty$.  The leading-order finite-volume correction
	agrees with Eq.~\eqref{eq:complete_IR_scaling}, and the first omitted term
	is of the same order.
	The methods presented in \cite{Lee:2010km,Pine:2012zv}
	can be used to extend the present formulas to higher
	orders and more general cases, including coupled channels.  This area
	is ripe for investigation.

	Another area of investigation is how the trends for operator extrapolation
	carry over for $A > 2$.

	\medskip
	\subsubsection{Related developments}

	The results presented in this chapter have exclusively been for the two-body
	case.
	%
	\begin{figure}[thbp]
	\centering
	\includegraphics[width=0.6\textwidth]%
	{Extrapolation/triton_energy_vs_L2_srg_v3_widmu_log}
	\caption{Residual error for triton plotted as a function of $L_2$ (here
 		calculated with the deuteron-neutron reduced mass) for the two- and
		three-nucleon	potential in Ref.~\cite{Jurgenson:2009qs} unitarily evolved
		by the SRG to four different resolutions (specified by $\lambda$) with the
		same binding energy~\cite{Jurgenson:2009qs,Jurgenson:2010wy}.
		$\kinf$ here is the lowest separation energy (triton breaking up
		into deuteron and neutron).}
	\label{fig:triton_vs_L2}
	\end{figure}
	%
	Figure~\ref{fig:triton_vs_L2} shows $\Delta E = E_{\rm HO} - \Einf$ for triton
	plotted as a function of $L_2$.  Recall from Eq.~\eqref{eq:L2_def} that
	evaluating $L_2$ involves calculating the oscillator length $b$.  In
	Fig.~\ref{fig:triton_vs_L2}, we use the deuteron-neutron reduced mass,
	$\mu = 2/3 M_N$, to calculate $b$ and thereby $L_2$.  There are a few
	interesting observations to be made about Fig.~\ref{fig:triton_vs_L2}.
	Triton energies when plotted as a function of $L_2$ lie on a single line.
	Also, as in Fig.~\ref{fig:SRG_data_collapse}, triton energies from potentials
	evolved to various SRG $\lambda$'s, fall on the same line.  This indicates
	that $L_2$ is the correct length even for the three-body case and the
	three-body IR correction can also be written in terms of observables.
	Moreover the falloff is proportional to $e^{-2 \kinf L_2}$, with
	$\kinf$ being the lowest separation energy, as expected.


  There has been a lot of work on extending the results for IR energy corrections
	presented in this thesis to the many-body case.  This has been documented in
	Refs.~\cite{Furnstahl:2014hca, Wendt:2015nba}.  In
	Subsec.~\ref{subsec:radii_phase_shifts} we looked at the
	extrapolation of the radius-squared operator.  The authors of
	Ref.~\cite{Odell:2015xlw} extended this to the extrapolation of quadrupole
	moments and transitions for the deuteron.

	As mentioned previously, the approach of mapping the HO truncation into a
	hard-wall boundary condition (in both position as well as momentum space)
	can be used for any localized basis.  This has been explored for the case
	of Coulomb-Sturmian basis \cite{Caprio_Coulomb_Sturmian}.

	\subsection{UV front}
	\label{subsec:UV_front}

	The dependence of UV corrections on the number of nucleons $A$ is not
	yet established theoretically, but the tests in \cite{Konig:2014hma}
	seem to indicate that the cutoff dependence of $\Delta E$ for the
	many-body case is the same as in the two-body case, just scaled by an
	$A$-dependent overall constant.
	%
	\begin{figure}[thbp]
	\centering
	\includegraphics[width=0.6\textwidth]%
	{Extrapolation/momentum_distribution_ratio_modified}
	\caption{The ratio of the momentum distributions in nucleus to the deuteron
		momentum distributions.  The dashed, dotted, dot-dashed, long dashed,
		dot-long dashed lines correspond to $^3$He, $^4$He, $^{16}$O, $^{56}$Fe,
		and nuclear matter respectively.  Figure taken from
		\cite{CiofidegliAtti:1995qe}.}
	\label{fig:mom_distribution_ratio}
	\end{figure}
	%
	This can be understood
	from general considerations of short-range correlations~\cite{Kimball:1973aa}
	or more systematically using the operator product
	expansion~\cite{Bogner:2012zm,Hofmann:2013aa}.  If there is a common two-body
	part, it may determine the dominant $\Lambda_2$ dependence
	with the rest providing the $A$-dependent scale factor.  This behavior would
	be consistent with the observation of a universal shape for high-momentum
	tails in momentum distributions in
	Fig.~\ref{fig:mom_distribution_ratio} (or the corresponding short-distance
	behavior)~\cite{CiofidegliAtti:1995qe,Feldmeier:2011qy}.  Connecting the
	two-body UV extrapolation results to the many-body case remains an open
	question.

	In this chapter we worked in the region where either the IR or the UV
	errors were dominant.  We saw in
	Fig.~\ref{fig:SRG_data_collapse_contamination} how UV contamination spoils
	the data collapse for IR extrapolation.  However, it is not always possible
	to isolate the IR and the UV contributions.  We therefore need reliable
	extrapolation schemes which can be employed when both the IR and UV errors
	are comparable.  Ref.~\cite{Jurgenson:2013yya} combined phenomenological
	UV errors \footnote{Phenomenological form for the UV error is found to be
	Gaussian in the cutoff $\Lambda_2$.  Ref.~\cite{Konig:2014hma} discusses how
	the Gaussian form arises.} with leading IR errors and used extrapolation of
	the	form
	%
	\beq
	E(\Lambda_2, L_2) = \Einf + B_0 \ee^{-2 \Lambda_2^2/ B_1^2} + B_2
		\ee^{-2 \kinf L_2} \;.
	\label{eq:combine_IR_UV_phenomenological}
	\eeq
	%
	\begin{figure}[thbp]
	\centering
	\includegraphics[width=0.6\textwidth]%
	{Extrapolation/Li7_Eb_vs_hw_kvnn10_srg_lam2p0_NNN_combined_Nmax6_Kval1_L2_contributions_labeled}
	\caption{Ground-state energy for $^7$Li with IR (vertical dashed lines) and
		UV (vertical dotted lines) corrections from
		Eq.~\eqref{eq:combine_IR_UV_phenomenological} added to predict $\Einf$
		values.  The horizontal dashed line is the global $\Einf$.
		Figure taken from \cite{Jurgenson:2013yya}.}
	\label{fig:combine_IR_UV_Jurgenson}
	\end{figure}
	%
	As seen in Fig.~\ref{fig:combine_IR_UV_Jurgenson} this simple addition of IR
	and UV errors seems to work well.  More work will be needed to place this
	on a sound theoretical foundation.

	The authors of Ref.~\cite{Binder:2015trg} developed interactions from chiral
	EFT that are tailored to the HO basis.  In their approach, the UV convergence
	with respect to the model space is implemented by construction (through
	refitting of LECs) and IR convergence is achieved by enlarging the model
	space for the kinetic energy.  This exhibited a fast convergence of
	ground-state energies and radii for nuclei up to $^{132}$Sn.  Thus,
	the development of reliable extrapolation schemes is indeed pushing the
	ab-initio frontier to heavier nuclei.
