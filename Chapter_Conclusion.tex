% !TEX root = More_PhD_Thesis.tex
\cleardoublepage
\chapter{Epilogue}

	This thesis presented author's original work over the past four years.
	Most of the work has already been published.  The thesis offers more
	motivation for our work, added details about calculations, and new insights.

	The work in Chapter~\ref{chap:Extrapolation} started as author's warm-up
	problem during the summer following his first year.  We kept finding
	interesting results, and the warm-up problem turned into a full-fledged
	project resulting in three publications
	\cite{More:2013rma,Furnstahl:2013vda,Konig:2014hma}.
	Our work focused on two-body systems, though in principle, the
	two-body results usually don't need extrapolation.
	However, availability of exact answers allowed us to test our results.
	Our results pioneered
	the development of physically motivated extrapolation schemes in LENP.  
	The work in Refs.~\cite{Furnstahl:2014hca,Wendt:2015nba,Binder:2015trg}
	showed a way to extend our work to many-body nuclei.

	The uncertainty in the scale and scheme dependence of nuclear structure
	and reactions component made it difficult to make robust predictions
	for experiments (cf.~Fig.~\ref{fig:0nu_double_beta_matrix_elements}).
	To tackle this issue, we followed the same approach as in
	Chapter~\ref{chap:Extrapolation}, i.e., we started with a two-body
	system which is more tractable.

	In Sec.~\ref{sec:factorization_summary}, we listed some of the direct
	extensions of our work.  Here we will discuss some of the broader topics
	relevant to our analysis in Chapter~\ref{chap:factorization}.  To
	begin with we looked at the deuteron disintegration reaction.  Its
	time-reversed version $n + p \rightarrow d + \gamma$ is appealing as
	well due to its relevance to the big bang nucleosynthesis.  It would be
	instructive to see how our analysis carries over to this reaction.

	Convince people here how your work is important. Make reference to
	neutrinoless
	double beta decay and other developments going on in the field. For instance
	n + p --> d + gamma relation to nuclear astrophysics.

	Make connection to EMC effect
