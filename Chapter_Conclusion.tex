% !TEX root = More_PhD_Thesis.tex
\cleardoublepage
\chapter{Epilogue}

	This thesis presented the author's original work over the past four years.
	Most of the work has already been published.  The thesis offers more
	motivation for our work, added details about calculations, and
	presents new insights along with recent developments.

	The work in Chapter~\ref{chap:Extrapolation} started as author's warm-up
	problem during the summer following his first year.
	The key development was the mapping between the harmonic oscillator
	truncation and the hard-wall boundary condition.  
	We kept finding
	interesting results, and the warm-up problem turned into a full-fledged
	project resulting in three publications
	\cite{More:2013rma,Furnstahl:2013vda,Konig:2014hma}.
	Our work focused on two-body systems, though in principle, the
	two-body results usually don't need extrapolation.
	However, availability of exact answers allowed us to test our results.
	Our results pioneered
	the development of physically motivated extrapolation schemes in LENP.
	The work in Refs.~\cite{Furnstahl:2014hca,Wendt:2015nba,Binder:2015trg}
	showed a way to extend our work to many-body nuclei.

	The uncertainty in the scale and scheme dependence of nuclear structure
	and reactions components made it difficult to make robust predictions
	for experiments (cf.~Fig.~\ref{fig:0nu_double_beta_matrix_elements}).
	To tackle this issue, we followed the same approach as in
	Chapter~\ref{chap:Extrapolation}, i.e., we started with a two-body
	system which is more tractable.  We found that the scale dependence
	depends strongly on kinematics, but in a systematic way and therefore
	can be understood.

	In Sec.~\ref{sec:factorization_summary}, we listed some of the direct
	extensions of our work.  Here we will discuss some of the broader topics
	relevant to our analysis in Chapter~\ref{chap:factorization}.  To
	begin with we looked at the deuteron disintegration reaction.  Its
	time-reversed version $n + p \rightarrow d + \gamma$ is appealing as
	well due to its relevance to the big bang nucleosynthesis.  It would be
	instructive to see how our analysis carries over to this reaction.

	One of the outstanding mysteries of nuclear physics is the \emph{EMC effect}
	named after the European Muon Collaboration that discovered it $32$
	years ago \cite{Aubert:1983xm}.  They observed that the probability of
	deep inelastic scattering (DIS) off a quark is significantly different from
	the same probability in a free nucleon.
	%
	\begin{figure}[htbp]
	 \centering
	 \includegraphics[width=0.7\textwidth]%
	 {Conclusion/EMC_summary}
	 \caption{The EMC effect in different nuclei \cite{Norton:2003cb}.
	 $x$ is the Bjorken-$x$. }
	 \label{fig:EMC_summary}
	\end{figure}
	%
	Figure~\ref{fig:EMC_summary} shows EMC effect for various nuclei.
	Given that nuclei are weakly bound (maximum of $8.8 {\rm~MeV}$ per nucleon)
	compared to the energy transfer in DIS (order of GeV), the deviation of
	ratio in Fig.~\ref{fig:EMC_summary} by up to 20\% from unity was
	unexpected.  A complete understanding of this curious EMC effect still
	remains elusive.

	Experiments at Jefferson Lab indicate that there is a correlation
	between the two-nucleon short-range correlations and the EMC effect
	(cf.~Fig.~\ref{fig:EMC_SRC_correlation}).
	%
	\begin{figure}[htbp]
	 \centering
	 \includegraphics[width=0.5\textwidth]%
	 {Conclusion/EMC_SRC_correlation}
	 \caption{The relationship between the number of two-nucleon correlated
	 pairs $a_2(A/d)$, and the strength of the EMC effect.  The later is
	 characterized by the slope of EMC effect in $0.3 < x < 0.7$.
	 Figure from \cite{Long_range_plan}.}
	 \label{fig:EMC_SRC_correlation}
	\end{figure}
	%
	However, as we have already seen (cf.~Figs.~\ref{fig:SRG_evolution_wf}
	and \ref{fig:wavefunction_evolution_deuteron_D_state}), SRG evolved
	wave functions do not have the SRCs.  Instead the SRC physics is accounted
	for by the evolution of the operator.  Studies analogous to the one
	presented in Chapter~\ref{chap:factorization} will help elucidate the
	model dependence of SRCs, and will be valuable for understanding the
	EMC effect.

	An on-going debate in the LENP community is the nature and interpretation
	of the spectroscopic factors \cite{Furnstahl:2010wd}.  Spectroscopic factors
	involve nuclear wave functions and are thus scale and scheme dependent.
	Nonetheless, in experimental analysis they are often treated as observables
	with no scale/scheme dependence.  This is often because the dependence is
	unclear.  An exercise similar to our analysis in
	Chapter~\ref{chap:factorization} will shed a light on the scale/scheme
	dependence of the spectroscopic factors.  This in essence, will
	bridge the gap between theory and experiments resolving the issues
	associated with the extraction of nuclear properties from the experiments.
