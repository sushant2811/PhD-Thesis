% !TEX root = More_PhD_Thesis.tex
\cleardoublepage
\chapter[Factorization]{Factorization
	\footnote{Based on \cite{More:2015tpa}}}
	\label{chap:factorization}

	\section{Motivation}

	Most of the information we know about nuclear interactions and the properties
	of the nuclei comes from some kind of scattering experiments (either elastic
	or inelastic).  In such experiments we scatter a known probe off a nucleus
	and	extract information about nuclear interactions by looking at the final
	outcome of the scattering experiment.
	%
	\begin{figure}[thbp]
	\centering
	\includegraphics[width=0.5\textwidth]%
	{Factorization/knock_out_schematic.png}
	\caption{Schematic of a nucleon knockout reaction.}
	\label{fig:knock_out_schematic}
	\end{figure}
	%
	\begin{figure}[thbp]
	\centering
	\includegraphics[width=0.6\textwidth]%
	{Factorization/factorization_schematic_annotated}
	\caption{Schematic illustration of factorization between nuclear structure
		and reactions component.}
	\label{fig:factorization_schematic}
	\end{figure}
	%
	Figure~\ref{fig:knock_out_schematic} shows a schematic for a nucleon knockout
	reaction where the probe is electrons which in turn interact with the nucleus
	by emitting virtual photons.

	The process of extracting nuclear properties from such experiments relies
	on the assumption that the effects of the probe are well understood and can
	be separated from the nuclear interactions we are trying to study.  This is
	the	\emph{factorization} between the nuclear structure and the nuclear
	reaction components illustrated schematically in
	Fig.~\ref{fig:factorization_schematic}.  The reaction component describes the
	probe and the structure includes the description of the initial and final
	states.
	This factorization between the structure and reaction components depends
	on the renormalization scale and scheme.  In some physical systems
	(e.g., in cold atoms near unitarity~\cite{Hoinka:2013fsa}), the scale and
	scheme dependence is very weak and can be safely neglected.  In some other
	physical systems such as in deep inelastic scattering (DIS) in high-energy
	QCD, the scale and scheme dependence is very manifest.
	%
	\begin{figure}[thbp]
	\centering
	\includegraphics[width=0.55\textwidth]%
	{Factorization/factorization_schematic_highE_QCD}
	\caption{Factorization in high-energy QCD.  $x$ is the Bjorken-$x$ and it
		denotes the fraction of momentum of the nucleon carried by the parton
		under consideration.  $a$ denotes the parton flavor.}
	\label{fig:factorization_schematic_highE_QCD}
	\end{figure}
	%
	Figure~\ref{fig:factorization_schematic_highE_QCD} illustrates the
	factorization in DIS.  The form factor $F_2$ of the nucleon
	(which up to some kinematic factors is the cross section) is given by the
	convolution of the long-distance parton density and the short-distance
	Wilson coefficient.  In this case, the parton density forms the structure
	part which is non-perturbative and the Wilson coefficient form the reactions
	part which can be calculated in perturbative QCD.  These separation
	between long- and short-distance physics is not unique, and is defined by
	the factorization scale $\mu_f$.  To minimize the contribution of logarithms
	that can disturb the perturbative expansion, $\mu_f$ is chosen to be equal to
	the	magnitude of the four-momentum transfer $Q$.
	The form factor $F_2$ (because it is
	related to the observable cross section) is independent of $\mu_f$, but
	the individual components are not.  As a consequence, the parton density
	(or distribution) function $f_a(x, Q^2)$ runs with $Q^2$.
	%
	\begin{figure}[thbp]
	\centering
	\includegraphics[width=0.55\textwidth]%
	{Factorization/uquark_parton_distribution_3d_surf}
	\caption{Parton distribution for the up quarks in the proton as a function
		of $x$ and $Q^2$.  Figure taken from \cite{Furnstahl:2013dsa}.}
	\label{fig:uquark_PDF_3D}
	\end{figure}
	%
	This is demonstrated in Fig.~\ref{fig:uquark_PDF_3D}.  The parton
	distributions
	$f_a(x, Q^2)$ and $f_a(x, Q_0^2)$	at two different $Q^2$ are related by
	DGLAP evolution or the Altarelli-Parisi equations \cite{Altarelli:1977zs}.
	Thus the scale dependence of the structure and reaction components is well
	understood in high-energy QCD.

	The situation is far from well settled in low-energy nuclear physics.
	Nuclear
	structure has conventionally been treated largely separate from nuclear
	reactions
	(e.g., the two volumes of Feshbach's \emph{Theoretical Nuclear Physics} are
	divided
	this way).  The nuclear structure community usually dealt with calculating
	time-independent properties such as nuclear binding energies, excitation
	spectra, radii, so on whereas the nuclear reaction experts worked on
	disintegration,	knock-out and
	transfer reactions.  However, both the communities invariably use inputs
	from the other side, and the consistency and universality of different
	components is not
	always guaranteed. 	To go back to the high-energy QCD analogy, the parton
	distribution functions (PDFs) $f_a(x, Q)$ extracted from the DIS are
	universal, in	the sense that they are process-independent.  For instance,
	the PDFs extracted
	from the DIS can be used for making prediction for the Drell-Yan process.
	The analogous process independence in the extracted quantities has not yet
	been achieved in low-energy nuclear physics.  This leads to ambiguous
	uncertainty	quantification when the nuclear properties extracted from one
	process	(cf.~Fig.~\ref{fig:factorization_schematic}) are used as an input to
	predict something	else.

	The factorization in low-energy nuclear physics is illustrated in
	Fig.~\ref{fig:low_energy_factorization}.
	%
	\begin{figure}[thbp]
	\centering
	\includegraphics[width=0.6\textwidth]%
	{Factorization/low_energy_factorization}
	\caption{Schematic illustration of factorization in low-energy nuclear
		physics.}
	\label{fig:low_energy_factorization}
	\end{figure}
	%
	The observable cross section in this case is a convolution of the
	spectroscopic factor
	and single-particle cross section.  However, there are many open questions
	such as	what are the nuclear properties that we can extract and what is the
	scale/scheme dependence of the extracted properties.

	The Similarity Renormalization Group (or the SRG) transformations were
	introduced in	Subsec.~\ref{subsec:SRG_intro}.  We noted that SRG
	transformations are a class of unitary transformations that soften nuclear
	Hamiltonians and lead to accelerated convergence of observables.
	%
	\begin{figure}[thbp]
	\centering
	\includegraphics[width=0.6\textwidth]%
	{Factorization/vsrg_T_3S1_kvnn_06_md_3dplot3}
	\caption{Deuteron momentum distribution at different SRG resolutions
		$\lambda$.  The evolved momentum distribution does not have the short-range
		correlations (SRCs).  Figure from \cite{Furnstahl:2013dsa}.}
	\label{fig:momentum_distribution}
	\end{figure}
	%
	Figure~\ref{fig:momentum_distribution} shows the momentum distribution for
	the deuteron as a function of the SRG scale and momentum.  Note that
	Fig.~\ref{fig:momentum_distribution} is analogous to
	Fig.~\ref{fig:uquark_PDF_3D}.  The SRG evolution gets rid of the
	high-momentum components and therefore the evolved momentum distributions
	don't have the short-range correlations (SRCs).
	Figure~\ref{fig:momentum_distribution} makes it clear that the
	high-momentum tail of the momentum distribution is dramatically resolution
	dependent.  Yet it is common in the literature that
	high-momentum components are treated as measurable, at least
	implicitly~\cite{Frankfurt:2008zv,Arrington:2011xs,Rios:2013zqa,
	Boeglin:2015cha}.
	In fact, what can be extracted is the momentum distribution at some scale,
	and	with the specification of a scheme.  This makes momentum distributions
	model	dependent~\cite{Ford:2014yua, Sammarruca:2015hba}.

	%
	\begin{figure}[thbp]
	\centering
	\includegraphics[width=0.6\textwidth]%
	{Factorization/wave_functions_evolution}
	\caption{$D$-state wave functions for the deuteron for the AV18 potential and
		the AV18 potential evolved to two SRG $\lambda$'s.}
	\label{fig:wavefunction_evolution_deuteron_D_state}
	\end{figure}
	%
	Figure~\ref{fig:wavefunction_evolution_deuteron_D_state} shows the $D$-state
	wave function for deuteron.  We see that just like the momentum distributions,
	SRG transformed wave functions do not have the high-momentum components.
	Therefore,
	if we use the SRG evolved wave function for calculating cross section for a
	process involving high-momentum probe, then the only way we get the same
	answer as with the unevolved wave function is if the relevant operator
	changed as well.  Thus, with SRG evolution, the high-momentum physics is
	shuffled from the wave function (nuclear structure) to the operator
	(nuclear reaction component).  This is reminiscent of chiral EFTs where the
	renormalization replaces the high-momentum modes in intermediate states
	by contact interactions (see Fig.~\ref{fig:replace_loop_contact}).
	%
	\begin{figure}[thbp]
	\centering
	\includegraphics[width=0.55\textwidth]%
	{Factorization/fig_vsrgreplace3}
	\caption{High-momentum modes in intermediate states replaced by contact
		interactions.  Figure from \cite{Furnstahl:2013dsa}.}
	\label{fig:replace_loop_contact}
	\end{figure}
	%
	Discussion so far shows how SRG makes the scale dependence of factorization
	explicit.  SRG transformations come with the momentum scale $\lambda$ and
	%
	\begin{figure}[thbp]
	\centering
	\begin{overpic}
   [width=0.6\textwidth]%
	 {Factorization/factorization_part.pdf}
   \put(20,33){\textcolor{black}{\large $p < \lambda$}}
   \put(67,33){\textcolor{black}{\large $p > \lambda$}}
   \end{overpic}
	\caption{The SRG scale $\lambda$ sets the natural scale for factorization.}
	\label{fig:SRG_factorization}
	\end{figure}
	%
	this sets the scale for factorization.  As seen in the
	Fig.~\ref{fig:SRG_factorization}, we have a natural separation that the piece
	which involves momenta less than $\lambda$ forms the long-distance part and
	the piece that involves momenta greater than $\lambda$ forms the
	short-distance part.

	Consider the differential cross section given by the overlap matrix element
	of initial and final states.
	\beq
	\frac{d \sigma}{d \Omega} \propto \left|
	\mbraket{\psi_f}{\widehat{O}}{\psi_i} \right|^2 \;.
	\eeq
	The SRG evolved wave function is given by $\ket{\psi_i^\lambda} =
	U_{\lambda} \ket{\psi_i}$, where $U_{\lambda}$ is the unitary matrix
	associated with the SRG transformation.  The cross section is an
	experimental observable and should be independent of our choice of
	the SRG scale.  If we evolve all the components
	of the matrix element consistently
	\beq
	\mbraket{\psi_f}{\wh{O}}{\psi_i} = \mbraket{
	\underbrace{\psi_f U_{\lambda}^\dag}_{\psi_f^\lambda}}
	{\underbrace{U_\lambda \wh{O} U_{\lambda}^{\dag}}_{\wh{O}^\lambda}}
	{\underbrace{U_{\lambda} \psi_i}_{\psi_i^{\lambda}}} \;,
	\label{eq:matrix_element_invariance}
	\eeq
	then the evolved matrix element is same as the unevolved one and the
	observable cross section is unchanged.

	In general, to be consistent between structure and reactions one must
	calculate	cross sections or decay rates within a single framework.
	That is, one must	use
	the same Hamiltonian and consistent operators throughout the calculation
	(which means the same scale and scheme).  Such consistent calculations have
	existed	for some time for few-body nuclei (e.g.,
	see~\cite{Epelbaum:2008ga,Hammer:2012id,Carlson:2014vla,Marcucci:2015rca})
	and	are becoming increasingly feasible for heavier nuclei because of advances
	in reaction technology, such as using complex basis states to handle
	continuum	physics.  Recent examples in the literature include
	No Core Shell Model Resonating Group Method
	(NCSM/RGM)~\cite{Quaglioni:2015via}, coupled cluster~\cite{Bacca:2013dma},
	and lattice EFT calculations~\cite{Pine:2013zja}.
	But there are many open	questions about constructing consistent currents and
	how to compare results from two such calculations.
	Some work along this direction which includes the evolution of the operator
	has recently been done.  Anderson \etal looked at the static properties
	of the deuteron such as momentum distributions, radii, and form factors under
	SRG evolution; they found no pathologies in the evolved operators, and
	the evolution effects were small for low-momentum observables
	\cite{Anderson:2010aq}.  Schuster \etal found in their work on radii and
	dipole transition matrix elements in light nuclei that the evolution effects
	are as important as three-body forces
	\cite{Schuster:2014lga,Schuster:2013sda}.
	Neff \etal looked at the SRG transformed density operators and concluded that
	it is essential to use evolved operators for observables sensitive to
	short-range physics \cite{Neff:2015xda}.  But all this work was done for
	expectation values of the operator, i.e, the state on the either side of the
	matrix element was the same.  In particular, there was no work which dealt
	with the issues related to operator evolution when we have a transition to
	continuum.  This is what we sought to address in \cite{More:2015tpa}.

	The electron scattering knock-out process is particularly interesting because
	of the connection to past, present, and planned
	experiments~\cite{Boffi:1996, LENP_white_paper2015}.
	The conditions for clean factorization
	of structure and reactions in this context is closely related to the impact of
	3N forces,
	two-body currents, and final-state interactions, which have not been cleanly
	understood as yet~\cite{Furnstahl:2010wd}.
	All of this becomes particularly relevant for high-momentum-transfer electron
	scattering.%
	\footnote{Note that high-momentum transfers imply high-resolution
	\emph{probes}, which is different from the resolution induced by the SRG
	scale.  How the latter should be chosen to best
	accommodate the former is a key unanswered question.}
	This physics is conventionally explained in terms of short-range correlation
	(SRC) phenomenology~\cite{Frankfurt:2008zv,Atti:2015eda}.  SRCs are two- or
	higher-body components of the nuclear wave function with high relative
	momentum and low center-of-mass momentum.  These explanations would seem to
	present a	puzzle for descriptions of nuclei with low-momentum Hamiltonians,
	for which	SRCs are essentially absent from the wave functions.

	This puzzle is resolved by the unitary transformations that mandate the
	invariance of cross section (cf.~Eq.~\eqref{eq:matrix_element_invariance}).
	The physics that was described by SRCs in the
	wave functions must shift to a different component, such as a two-body
	contribution from the current (cf.~Fig.~\ref{fig:replace_loop_contact}).
	This may appear to complicate the reaction
	problem just as we have simplified the structure part, but past work and
	analogies to other processes suggests that factorization may in fact
	become cleaner~\cite{Anderson:2010aq,Bogner:2012zm}.  One of our goals is to
	elucidate this issue, although we have only begun to do so in
	\cite{More:2015tpa}.

	In particular, we take the first steps in exploring the interplay of
	structure and reaction as a function of kinematic variables and SRG decoupling
	scale $\lambda$ in a controlled calculation of a knock-out process.  There are
	various complications for such processes.  With RG evolution, a
	Hamiltonian---even with only a two-body potential initially---will develop
	many-body components as the decoupling scale decreases.  Similarly, a one-body
	current will develop two- and higher-body components.

	Our strategy is to avoid dealing with all of these complications
	simultaneously
	by considering the cleanest knock-out process: deuteron electrodisintegration
	with only an initial one-body current.  With a two-body system, there are no
	three-body forces or three-body currents to contend with.  Yet it still
	includes several key ingredients to investigate: i) the wave function will
	evolve with changes in resolution; ii) at the same time, the one-body current
	develops two-body components, which are simply managed; and iii) there are
	final-state interactions (FSI).  It is these ingredients that will mix under
	the RG evolution.  We can focus on different effects or isolate parts of the
	wave function by choice of kinematics.  For example, we can examine when the
	impulse approximation is best and to what extent that is a
	resolution-dependent assessment.


	\section{Test ground: Deuteron disintegration}

	\subsection{Formalism}
	\label{subsec:formalism}

	Deuteron electrodisintegration is the simplest nucleon-knockout process
	and has been considered as a test ground for various $NN$ models for a
	long time (see, for example, Refs.~\cite{Arenhovel:2004bc,Boeglin:2015cha}).
	It has also been well studied
	experimentally~\cite{Gilad:1998wia,Egiyan:2007qj}.
	The absence of three-body currents and forces
	makes it an ideal starting point for studying the interplay with SRG evolution
	of the deuteron wave function, current, and final-state interactions.

	We follow the approach of Ref.~\cite{Yang:2013rza}, which we briefly review.
	The kinematics for the process in the laboratory frame is shown in
	Fig.~\ref{fig:deut_dis_kinematics}.
	%
	\begin{figure}[htbp]
	 \centering
	 \includegraphics[width=0.6\textwidth]%
	 {Factorization/Kinematics}
	 \caption{The geometry of the electro-disintegration process in
	 the lab frame.  The virtual photon disassociates the deuteron into the proton
	 and the neutron (not shown in this figure).}
	 \label{fig:deut_dis_kinematics}
	\end{figure}
	%
	The virtual photon from electron scattering transfers enough energy and
	momentum to break up the deuteron into a proton and neutron.  The
	differential cross section for deuteron electrodisintegration for unpolarized
	scattering in the lab	frame is given by~\cite{Arenhovel:1988qh}
	%
	\begin{multline}
	 \frac{\dd^3 \sigma}{\dd {k^\prime}^{\rm lab} \dd\Omega_e^{\rm lab}
	 \dd \Omega_p^{\rm lab}}
	 = \frac{\alpha}{6 \, \pi^2} \frac{{k^\prime}^{\rm lab}}
	 {k^{\rm lab} (Q^2)^2}
	 \Big[
	  v_L \fL + v_T f_T \\
	  \null + v_{TT} f_{TT} \cos 2 \phi_p^{\rm lab}
	  + v_{LT} f_{LT} \cos \phi_p^{\rm lab}
	 \Big] \,.
	\end{multline}
	%
	Here $\Omega_e^{\rm lab}$ and $\Omega_p^{\rm lab}$ are the solid angles of the
	electron and the proton, $k^{\rm lab}$ and ${k^\prime}^{\rm lab}$ are the
	magnitude of incoming and outgoing electron 3-momenta,  $Q^2$ is the
	4-momentum-squared of the virtual photon, and $\alpha$ is the fine structure
	constant.  $\phi_p^{\rm lab}$ is the angle between the scattering plane
	containing the electrons and the plane spanned by outgoing nucleons.
	$v_L\,, v_T\,, \ldots$ are electron kinematic factors, and $\fL, f_T, \ldots$
	are the deuteron structure functions.  These structure functions contain all
	the dynamic information about the process.
	The four structure functions are independent and can be separated by combining
	cross-section measurements carried out with appropriate kinematic
	settings~\cite{Kasdorp:1997ba}.  Structure functions are thus cross sections
	up to kinematic factors and are independent of the SRG scale $\lambda$.
	They are analogous of the form factor $F_2$ we saw in the DIS case
	(cf.~Fig.~\ref{fig:factorization_schematic_highE_QCD}).  In our
	work we focus on the longitudinal structure function $\fL$, following the
	approach of Ref.~\cite{Yang:2013rza}.

	\medskip

	\subsubsection{Calculating $\fL$}

	As in Ref.~\cite{Yang:2013rza}, we carry out the calculations in
	the center-of-mass frame of the outgoing proton-neutron pair.
	In this frame the photon
	four-momentum is $(\omega,\mbf{q})$, which can be obtained from the
	initial electron energy and $\theta_e$, the electron scattering angle.
	We denote the momentum of the outgoing proton by $\mathbf{\pp}$ and
	take $\mbf{q}$ to be along $z$-axis.  The angles of $\mbf{\pp}$ are
	denoted by $\Omega_{\mbf{\pp}} = (\thetacm, \phicm)$.

	The longitudinal structure function can be written as
	%
	\begin{equation}
	 \fL = \sum_{\substack{S_f, m_{s_f}\\ \mJd}}
	 \ampT_{S_f,m_{s_f},\mu = 0,\mJd}\!(\thetacm, \phicm) \,
	 \ampT_{S_f, m_{s_f},\mu = 0,\mJd}^\ast\!(\thetacm, \phicm) \,,
	\label{eq:f_L_from_T}
	\end{equation}
	%
	where $S_f$ and $m_{s_f}$ are the spin quantum numbers of the final
	neutron-proton state, $\mu$ is the polarization index of the virtual photon,
	and $\mJd$ is the angular momentum of the initial deuteron
	state. The amplitude $\ampT$ is given by~\cite{Arenhoevel:1992xu}
	%
	\begin{equation}
	 \ampT_{S,\msf,\mu,\mJd}
	 = -\pi \sqrt{2\alpha|\mathbf{\pp}|E_p E_d / M_d}
	 \,\la \psi_{f} \,|\, J_{\mu}(\mathbf{q}) \,| \psi_i \ra \,,
	\label{eq:T_definiton}
	\end{equation}
	%
	where $\bra{\psi_f}$ is the final-state wavefunction of the outgoing
	neutron-proton pair, $\ket{\psi_i}$ is the initial deuteron state, and
	$J_{\mu} (\mbf{q})$ is the current operator that describes the momentum
	transferred by the photon.
	%
	The variables in Eq.~\eqref{eq:T_definiton} are:
	%
	\begin{itemize}
	\item fine-structure constant $\alpha$;
	\item outgoing proton (neutron) 3-momentum $\mbf{\pp} \, (-\mbf{\pp})$;
	\item proton energy $\displaystyle E_p = \sqrt{M^2 + \mbf{\pp}^2}$,
	where $M$ is the average of proton and neutron mass
	\item deuteron energy $\displaystyle E_d = \sqrt{M_d^2 + \mbf{q}^2} $,
	where $M_d$ is the mass of the deuteron.
	\end{itemize}
	%
	As mentioned before, all of these quantities are in the center-of-mass
	frame of the outgoing nucleons.

	For $f_L$, $\mu = 0$ and therefore only $J_0$ contributes.
	The one-body current matrix element is given by
	%
	\begin{multline}
	 \la \mbf{k}_1 \, T_1| \, J_0(\mbf{q}) \, | \,\mbf{k}_2 \, T\!=\!0 \ra
	 = \frac{1}{2} \big(G_E^p + (-1)^{T_1} G_E^n\big) \,
	 \delta(\mbf{k}_1 - \mbf{k}_2 - \mbf{q}/2) \\
	 \null + \frac{1}{2} \big((-1)^{T_1} G_E^p +  G_E^n\big) \, \delta(\mbf{k}_1
	 - \mbf{k}_2 + \mbf{q}/2) \,,
	\label{eq:J0_def}
	\end{multline}
	%
	where $G_E^p$ and $G_E^n$ are the electric form factors of the proton and the
	neutron, and the deuteron state has isospin $T=0$.

	The final-state wave function of the outgoing proton-neutron pair can be
	written	as
	%
	\begin{equation}
	 |\psi_f\ra = | \phi \ra + G_0 (E^\prime) \, t(E^\prime) \,| \phi \ra \,,
	\label{eq:psi_f_def}
	\end{equation}
	%
	where $\ket{\phi}$ denotes a relative plane wave, $\GreensFn$ and $t$
	are the Green's function and the $t$-matrix respectively, and  $E^\prime =
	\mbf{\pp}^2/M$ is the energy of the outgoing nucleons.  The second term in
	Eq.~\eqref{eq:psi_f_def} describes the interaction between the outgoing
	nucleons.  As the momentum associated with the plane wave $\ket{\phi}$ is
	$\pp$, the $t$-matrix $t(E^\prime)$ that enters our calculation is always
	half on-shell.

	%
	\begin{figure}[htbp]
	 \centering
	 \includegraphics[width=0.35\textwidth]%
	 {Factorization/IA_thesis_hi}

	 \small{(a) Impulse Approximation (IA) \\[1.5 em]}

   \includegraphics[width=0.35\textwidth]%
	 {Factorization/FSI_thesis_hi}

   \small{(b) Final State Interaction (FSI)}

	 \caption{(a) The first term on the right side of Eq.~\eqref{eq:psi_f_def}.
	 The outgoing nucleons do not interact.
	 (b) The second term on the right side of Eq.~\eqref{eq:psi_f_def}.  The
	 outgoing nucleons interact through the NN potential.  Figures taken from
	 \cite{Ibrahim:2006lta}. }
	 \label{fig:IA_FSI_schematic}
	\end{figure}
	%
	In the impulse approximation (IA) as defined here, the interaction between the
	outgoing nucleons is ignored and $|\psi_f \ra_{\rm IA} \equiv | \phi \ra$.
	A schematic of IA and FSI contributions is shown in
	Fig.~\ref{fig:IA_FSI_schematic}.
	The plane wave $|\phi\ra$ will have both isospin $0$ and $1$ components.
	The current $J_0$, $\GreensFn$, and the $t$-matrix are diagonal in spin space.
	The deuteron has spin $S=1$ and therefore the final state will also have
	$S=1$. 	Hence, we have
	%
	\begin{eqnarray}
	 \ket{\phi} & \equiv & \ket{\mbf{\pp}\, S\!=\!1 \, \msf \psi_T} \nonumber \\
	 & = & \dfrac{1}{2} \sum_{T = 0,1}
	 \big(\ket{\mbf{\pp} \, S\!=\!1 \, \msf}
		+ (-1)^T \ket{{-}\mbf{\pp} \, S\!=\!1 \, \msf} \big) \,\ket{T} \,.
	\label{eq:phi_def}
	\end{eqnarray}
	%
	Using Eqs.~\eqref{eq:J0_def} and~\eqref{eq:phi_def}, the overlap matrix
	element	in IA becomes
	%
	\begin{multline}
	 \la \psi_f | \, J_0 \, |\psi_i \ra_{\rm IA}
	 = \sqrt{\frac{2}{\pi}} \sum_{L_{d} = 0, 2}
	 \CG{L_d}{\mJd - \msf}{S\!=\!1}{\msf}{J\!=\!1}{\mJd} \\
	 \null \times \Big[
	  G_E^p \, \psi_{L_d}(|\mbf{\pp} - \mbf{q}/2|)
	  \,Y_{L_d,\mJd-m_{s_f}}\!(\Omega_{\mbf{\pp} - \mbf{q}/2}) \\
	  \null + G_E^n \, \psi_{L_d}(|\mbf{\pp} + \mbf{q}/2|)
	  \,Y_{L_d,\mJd-m_{s_f}}\!(\Omega_{\mbf{\pp} + \mbf{q}/2})
	 \Big] \,,
	\label{eq:overlap_IA}
	\end{multline}
	%
	where $\Omega_{\mbf{\pp} \pm \mbf{q}/2}$ is the solid angle between the
	unit vector $\hat{z}$ and $\mbf{\pp}\pm\mbf{q}/2$.  $\psi_{L_d}$ is the
	deuteron wave function in momentum space defined as
	%
	\begin{equation}
	 \braket{k_1 \, J_1 \, m_{J_1} \, L_1 \, S_1 \, T_1}{\psi_i}
	 = \psi_{L_1}(k_1)
	  \delta_{J_1,1}\delta_{m_{J_1},\mJd}
	 \delta_{L_1,L_d}\delta_{S_1,1}\delta_{T_1,0} \,.
  \end{equation}
	%
	The $S$-wave $(L=0)$ and $D$-wave $(L=2)$ components of the deuteron wave
	function satisfy the normalization condition
	%
	\begin{equation}
	 \frac{2}{\pi} \int dp \, p^2 \, \big(\psi_0^2(p) + \psi_2^2(p)\big) = 1 \,.
	\end{equation}
	%
	In deriving Eq.~\eqref{eq:overlap_IA} we have used the property of the
	spherical harmonics that
	%
	\begin{equation}
	 Y_{lm}(\pi-\theta,\phi+\pi) = (-1)^l \, Y_{lm}(\theta,\phi) \,.
	\end{equation}
	%
	In our work we follow the conventions of Ref.~\cite{Landau:1989}.
	Deriving Eq.~\eqref{eq:overlap_IA} also uses partial wave expansion
	\beq
	\ket{\mbf{k}} = \sqrt{\frac{2}{\pi}} \sum_{l, m} Y_{l \, m}^\ast
	(\Omega_{\mbf{k}}) \ket{k \, l \, m} \;,
	\eeq
	the normalization condition
	\beq
	\braket{p}{k} = \frac{\pi}{2} \frac{\delta(p - k)}{p^2} \;,
	\eeq
	and, the Clebsch-Gordan completeness relation
	\beq
	\ket{l \, m \, S\!=\!1 \, m_s} = \sum_{J, \, m_J} \ket{J \, m_J \, l \, S\!=\!1}
	\braket{J \, m_J \, l \, S\!=\!1}{l \, m \, S\!=\!1\, m_s} \;.
	\eeq
	%
	Because
	$\thetacm$ and $\phicm$ are the angles of $\mbf{\pp}$, $\Omega_{\mbf{\pp} -
	\mbf{q}/2} \equiv \big(\thetacprime(\pp, \thetacm, q), \phicm \big)$
	and $\Omega_{\mbf{\pp} + \mbf{q}/2} \equiv
	\big(\thetacdoubleprime(\pp, \thetacm, q),
	\phicm \big)$, where
	%
	\begin{equation}
	 \thetacprime (\pp, \thetacm, q) = \cos^{-1}
	 \!\left(
	  \frac{\pp \cos \thetacm - q/2}{\sqrt{{\pp}^2 - \pp q \cos\thetacm + q^2/4}}
	 \right)
	\label{eq:theta_c_prime_def}
	\end{equation}
	%
	and
	%
	\begin{equation}
	 \thetacdoubleprime (\pp, \thetacm, q) = \cos^{-1}
	 \!\left(
	  \frac{\pp \cos \thetacm + q/2}{\sqrt{{\pp}^2 + \pp q \cos\thetacm + q^2/4}}
	 \right) \,.
	\label{eq:theta_c_double_prime_def}
	\end{equation}
	%
	The expressions for $\thetacprime$ and $\thetacdoubleprime$ can be obtained by
	elementary trigonometry.  Note that Eqs.~\eqref{eq:theta_c_prime_def} and
	\eqref{eq:theta_c_double_prime_def} reproduce the correct $\pp = 0$ and
	$q = 0$ limit.

	The overlap matrix element including the final-state interactions (FSI) is
	given by
	%
	\begin{equation}
	 \la \psi_f| \, J_0 \,|\psi_i \ra
	 = \underbrace{\la \phi| \, J_0 \,|\psi_i \ra}_{\rm IA}
	 + \underbrace{\la \phi| t^\dag \, \GreensFn^\dag\, J_0 \,|\psi_i \ra}_
	 {\rm FSI} \,.
	\label{eq:overlap_IA_plus_FSI}
	\end{equation}
	%
	The first term on the right side of Eq.~\eqref{eq:overlap_IA_plus_FSI}
	has already been evaluated in Eq.~\eqref{eq:overlap_IA}.  Therefore, the term
	we still need to evaluate is $\la \phi| t^\dag \, \GreensFn^\dag\, J_0
	\,|\psi_i \ra$.  The $t$-matrix is most conveniently calculated in a
	partial-wave basis.  Hence, the FSI term is evaluated by inserting complete
	sets of states in the form
	%
	\begin{equation}
	 1 = \frac{2}{\pi} \sum_{\substack{L, S \\ J,  m_J }}
	 \sum_{T = 0, 1} \int dp \,p^2 \, |p \, J \, m_J \, L \, S \, T \ra
	 \, \la p \, J\, m_J \, L \, S \, T | \,.
	\label{eq:completeness_partial_wave}
	\end{equation}
	%
	The outgoing plane-wave state in the partial-wave basis is given by
	%
	\begin{multline}
	 \la \phi | \, k_1 \, J_1 \, m_{J_1} \, L_1 \, S\!=\!1 \, T_1 \ra
	  =  \frac{1}{2} \, {\sqrt\frac{2}{\pi}} \, \frac{\pi}{2} \,
	 \frac{\delta(\pp - k_1)}{k_1 ^2}
	 \CG{L_1}{m_{J_1} - \msf}{S\!=\!1}{\msf}{J_1}{m_{J_1}}
	 \, \\
	 \null \times \big(1 + (-1)^{T_1} (-1)^{L_1}\big)
	 \,Y_{L_1,m_{J_1} - \msf}\!(\thetacm, \phicm) \,.
	\label{eq:phi_def_pw}
	\end{multline}
	%
	The Green's
	function is diagonal in $J$, $m_J$, $L$, $S$, and $T$, so we have
	%
	\begin{equation}
	 \la k_1 | \, \GreensFn^\dag \,| k_2 \ra
	 = \frac{\pi}{2} \, \frac{\delta(k_1 - k_2)}{k_1^2}
	 \, \frac{M}{{\pp}^2 - k_1 ^2 - i\epsilon} \,.
	\label{eq:G0_def_pw}
	\end{equation}

	We also need to express the current in Eq.~\eqref{eq:J0_def} in the
	partial-wave basis.  To begin with, let us just work with first term in
	Eq.~\eqref{eq:J0_def}, which we denote by $J_0 ^-$.  In the partial-wave
	basis, it is written as
	%
	\begin{multline}
	 \mbraket{k_1 \, J_1 \, \mJd \, L_1 \, S\!=\!1 \, T_1}{J_0^{-}}
	 {\,k_2 \, J\!=\!1 \, \mJd \, L_2 \, S\!=\!1 \, T\!=\!0}
	 = \frac{\pi^2}{2} \, \big(G_E ^p + (-1)^{T_1} \, G_E^n\big) \\
	 \null \times \sum_{\widetilde{m}_s = -1}^1 \int\!\dcostheta
	 \, \braket{J_1 \, \mJd}{L_1 \, \mJd - \wt{m}_s \, S\!=\!1 \,\wt{m}_s}
	 \,P_{L_1}^{\mJd - \widetilde{m}_s}\!(\cos\theta) \\
	 \null \times
	 \,P_{L_2}^{\mJd - \widetilde{m}_s}\!\big(\cos\thetacprime(k_1, \theta, q)
	 \big) \frac{\delta\big(k_2-\sqrt{k_1^2-k_1 q \cos\theta + q^2/4}\big)}
	 {k_2^2} \\
	 \null \times
	 \CG{L_2}{\mJd - \widetilde{m}_s}{S\!=\!1}{\widetilde{m}_s}{J\!=\!1}{\mJd} \,.
	\label{eq:J_0_minus_def_pw}
	\end{multline}
	%
	Here $\mJd$ is the deuteron quantum number, which is preserved throughout.
	We have used the deuteron quantum numbers in the ket in anticipation
	that we will always evaluate the matrix element of $J_0$ with the deuteron
	wave function on the right.   $\thetacprime$ is as defined in
	Eq.~\eqref{eq:theta_c_prime_def}.  In deriving Eq.~\eqref{eq:J_0_minus_def_pw}
	we have also made use of the relation \cite{Jerry_thesis}
	%
	\begin{equation}
	 \int Y_{lm} ^\ast(\theta, \varphi) \,
	 Y_{l^\prime m^\prime}(\alpha^\prime,\varphi)
	 \,\dd\!\cos \theta \, \dd \varphi
	 = 2\pi \delta_{m m^\prime}
	 \int\!\dcostheta
	 P_l^m(\cos\theta) \, P_{l^\prime}^m(\cos \alpha^\prime) \,.
	\end{equation}

	Equations~\eqref{eq:phi_def_pw}, \eqref{eq:G0_def_pw},
	and~\eqref{eq:J_0_minus_def_pw} can be combined to obtain
	%
	\begin{multline}
	 \mbraket{\phi}{t^\dag \, \GreensFn^\dag \, J_0^-}{\psi_i}
	 = \sqrt{\frac{2}{\pi}} \,\frac{M}{\hbar c}
	 \sum_{T_1 = 0,1} \big(G_E ^p + (-1)^{T_1} \, G_E^n\big)
	 \sum_{L_1 = 0}^{L_{\rm max}} \big(1 + (-1)^{T_1} (-1)^{L_1}\big) \\
	 \null \times Y_{L_1 , \mJd - \msf}\!(\thetacm, \phicm)
	 \sum_{J_1 = |L_1 - 1|}^{L+1}
	 \CG{L_1}{\mJd - \msf}{S\!=\!1}{\msf}{J_1}{\mJd}
	 \sum_{L_2 = 0}^{L_{\rm max}} \int \dd k_2 \, k_2^2 \\
	 \null \times t^\ast(k_2, \pp, L_2, L_1, J_1, S\!=\!1, T_1)
	 \sum_{\widetilde{m}_s = -1}^1
	 \braket{J_1 \, \mJd}{L_2 \, \mJd - \wt{m}_s \, S\!=\!1 \, \wt{m}_s} \\
	 \null \times \sum_{L_d = 0,2}
	 \CG{L_d}{\mJd - \widetilde{m}_s}{S\!=\!1}{\widetilde{m}_s}{J\!=\!1}{\mJd}
	 \int \dcostheta
	 \, \frac{1}{{\pp}^2 - k_2^2 - i \epsilon} \\
	 \null \times
	 P_{L_2}^{\mJd - \widetilde{m}_s}(\cos\theta) \,
	 P_{L_d}^{\mJd - \widetilde{m}_s}\!\big(\cos\thetacprime(k_2, \theta, q)\big)
	 \, \psi_{L_d}\Big(\!\sqrt{{k_2}^2 - k_2 \, q \, \cos\theta + q^2/4}\Big) \,.
	\label{eq:phi_t_g0_J0_minus}
	\end{multline}
	%
	Note that the matrix element of the $t$-matrix in
	Eq.~\eqref{eq:phi_t_g0_J0_minus} should strictly be written as
	${t^{\ast}}({E^\prime} = {\pp}^2/M; k_2, \pp, L_2, L_1, J_1, S\!=\!1, T_1)$.  However,
	keeping in mind that the $t$-matrix in this chapter is always evaluated
	half on-shell, we drop the $E^\prime$ index for the sake of brevity.
	To evaluate the hermitian conjugate, we use the property
	%
	\beq
	t^\dag (\pp, k_2, L_1, L_2, J_1, S\!=\!1, T_1)
	= t^\ast (k_2, \pp, L_2, L_1, J_1, S\!=\!1, T_1) \;.
	\eeq
	%

	We denote the second term in the one-body current Eq.~\eqref{eq:J0_def} by
	$J_0^+$.  The expression for $\mbraket{\phi}{t^\dag \, \GreensFn^\dag \,
	J_0^+}{\psi_i}$ is analogous to Eq.~\eqref{eq:phi_t_g0_J0_minus}, the only
	differences being that the form-factor coefficient is $(-1)^{T_1} G_E^p +
	G_E^n$ and the input arguments for the second associated Legendre polynomial
	and the deuteron wave function are different.  The two factors respectively
	become $\displaystyle P_{L_d}^{\mJd - \widetilde{m}_s}\!
	\big(\cos\thetacdoubleprime(k_2, \theta, q)\big)$
	and $\psi_{L_d}\big(\sqrt{{k_2}^2 + k_2 \,q \, \cos\theta + q^2/4}\big)$,
	where	$\thetacdoubleprime$ is defined in
	Eq.~\eqref{eq:theta_c_double_prime_def}.
	It can be shown that
	$\mbraket{\phi}{t^\dag \, \GreensFn^\dag \, J_0^+}{\psi_i}
	= \mbraket{\phi}{t^\dag \, \GreensFn^\dag \, J_0^-}{\psi_i}$.  Thus,
	%
	\begin{equation}
	 \mbraket{\phi}{t^\dag \, \GreensFn^\dag \, J_0}{\psi_i}
	 = 2 \, \mbraket{\phi}{t^\dag \, \GreensFn^\dag \, J_0^-}{\psi_i} \,.
	\label{eq:J0_minus_twice_relation}
	\end{equation}
	%
	Using this we can evaluate the overlap matrix element in
	Eq.~\eqref{eq:overlap_IA_plus_FSI}.  As outlined in Eqs.~\eqref{eq:f_L_from_T}
	and \eqref{eq:T_definiton}, this matrix element is related to the
	longitudinal structure function $\fL$.  Recall that the deuteron spin is
	conserved throughout and therefore $S_f = 1$ in Eq.~\eqref{eq:f_L_from_T}.

	In Subsec.~\ref{subsec:Results} we present results for $\fL$ both in
	the IA and including the FSI.  These results match those of
	Ref.~\cite{Yang:2013rza,Arenhoevel:1992xu}, verifying the accuracy of the
	calculations presented above.

	\subsection{Evolution setup}
	\label{subsec:evolution_setup}

	As outlined previously, we want to investigate the effect of
	unitary transformations on calculations of $\fL$.
	Let us start by looking at the IA matrix element:
	%
	\begin{eqnarray}
	 \la \phi | J_0 | \psi_i \ra
	 &=& \la \phi | U^\dag \, U \, J_0 \, U^\dag \, U \, | \psi_i \ra \nonumber \\
	 &=& \underbrace{\la\phi|\widetilde{U}^\dag J_0^\lambda|\psi_i^\lambda\ra}_{A}
	 + \underbrace{\la \phi | \, J_0^\lambda | \psi_i^\lambda \ra}_{B} \,,
	\label{eq:A_B_split_up}
	\end{eqnarray}
	%
	where we decompose the unitary matrix $U$ into the identity and a residual
	$\widetilde{U}$,
	%
	\begin{equation}
	 U = I + \widetilde{U} \,.
	\label{eq:U_decomposition}
	\end{equation}
	%
	The matrix $\widetilde{U}$ is smooth and therefore amenable to interpolation.
	The $U$ matrix is calculated following the approach in \cite{Anderson:2010aq}.
	The terms in Eq.~\eqref{eq:A_B_split_up} can be further split into
	%
	\begin{equation}
	 \la \phi | \, J_0^\lambda | \psi_i^\lambda \ra
	 = \underbrace{\la \phi | \widetilde{U}\, J_0 \, \widetilde{U}^\dag |
	  \psi_i^\lambda \ra}_{B_1}
	 + \underbrace{\la \phi | \widetilde{U}\, J_0 \, | \psi_i^\lambda \ra}_{B_2}
	 + \underbrace{\la \phi | \, J_0 \, \widetilde{U}^\dag | \psi_i^\lambda
	  \ra}_{B_3}
	 + \underbrace{\la \phi |\, J_0 \, | \psi_i^\lambda \ra}_{B_4}
	\label{eq:B_split_up}
  \end{equation}
 	%
 	and
 	%
 	\begin{equation}
	 \la \phi | \widetilde{U}^\dag \, J_0^\lambda | \psi_i^\lambda \ra \\[0.25em]
	 = \underbrace{\la \phi | \widetilde{U}^\dag \, \widetilde{U}\, J_0 \,
	  \widetilde{U}^\dag | \psi_i^\lambda \ra}_{A_1}
	 + \underbrace{\la \phi | \widetilde{U}^\dag \, \widetilde{U}\, J_0 |
	  \psi_i^\lambda \ra}_{A_2}
	 + \underbrace{\la \phi | \widetilde{U}^\dag J_0 \,
	  \widetilde{U}^\dag | \psi_i^\lambda \ra}_{A_3}
	 + \underbrace{\la\phi | \widetilde{U}^\dag J_0 | \psi_i^\lambda\ra}_{A_4} \,.
	\label{eq:A_split_up}
 	\end{equation}
	%
	The $B_4$ term is the same as in Eq.~\eqref{eq:overlap_IA}, but with the
	deuteron wave function replaced by the evolved version $\psi_{L_d}^\lambda$.
	Inserting complete sets of partial-wave basis states as in
	Eq.~\eqref{eq:completeness_partial_wave} and using Eqs.~\eqref{eq:phi_def_pw}
	and \eqref{eq:J_0_minus_def_pw}, we can obtain the expressions for $B_1$,
	$B_2$, $B_3$ and $A_1,\ldots,A_4$.  These expressions are given in
	Appendix~\ref{Appendix:sec:evolution_expressions}.

	Using the expressions for $A_1,\ldots, A_4$ and $B_1,\ldots, B_4$, we can
	obtain results for $\fL$ in the IA with one or more components of the overlap
	matrix element $\mbraket{\phi}{J_0}{\psi}$ evolved.  When calculated in IA,
	$\fL$ with all components evolved matches its unevolved counterpart, as shown
	later	in Subsec.~\ref{subsec:Results}.
	%
	The robust agreement between the evolved and unevolved answers indicates that
	the expressions derived for $A_1, \ldots, B_4$ are correct and that there is
	no error in generating the $U$-matrices.  In
	Sec.~\ref{subsec:numerical_implementation} we provide some details about the
	numerical implementation of the equations presented here.

	Let us now take into account the FSI and study the effects
	of evolution.  The overlap matrix element should again be unchanged under
	evolution,
	%
	\begin{equation}
	 \mbraket{\psi_f}{J_0}{\psi_i}
	 = \mbraket{\psi_f^\lambda}{J_0^\lambda}{\psi_i^\lambda} \,,
	\end{equation}
	%
	where $\psi_f$ is given by Eq.~\eqref{eq:psi_f_def}.  Furthermore,
	%
	\begin{equation}
	 |\psi_f ^\lambda \ra = | \phi \ra + G_0 \, t_\lambda |\phi \ra \,,
	\label{eq:psi_f_lam_def}
	\end{equation}
	%
	where $t_\lambda$ is the evolved $t$-matrix, \ie, the $t$-matrix obtained
	by solving the Lippmann--Schwinger equation using the evolved potential, as
	discussed in Appendix~\ref{Appendix:sec:evolution_final_state}.  Thus
	%
	\begin{equation}
	 \mbraket{\psi_f^\lambda}{J_0^\lambda}{\psi_i^\lambda}
	 = \underbrace{\mbraket{\phi}{J_0^\lambda}{\psi_i^\lambda}}_{B}
	 + \underbrace{\mbraket{\phi}{t_\lambda ^\dag \, G_0^\dag
	  \, J_0^\lambda}{\psi_i^\lambda}}_{F} \,.
	\end{equation}
	%
	The term $B$ is the same that we already encountered in
	Eq.~\eqref{eq:A_B_split_up}.  The term $F$ can also be split up into four
	terms:
	%
	\begin{multline}
	 \mbraket{\phi}{t_\lambda^\dag\,G_0^\dag\,J_0^\lambda}{\psi_i^\lambda}
	 = \underbrace{\mbraket{\phi}{t_\lambda ^\dag \, G_0^\dag \, \widetilde{U}
	  \, J_0 \, \widetilde{U}^\dag}{\psi_i^\lambda}}_{F_1}
	 + \underbrace{\mbraket{\phi}{t_\lambda ^\dag \, G_0^\dag \, \widetilde{U}
	  \, J_0 }{\psi_i^\lambda}}_{F_2} \\
	 \null + \underbrace{\mbraket{\phi}{t_\lambda ^\dag \, G_0^\dag
	  \, J_0 \, \widetilde{U}^\dag}{\psi_i^\lambda}}_{F_3}
	 + \underbrace{\mbraket{\phi}{t_\lambda ^\dag \, G_0^\dag
	  \, J_0}{\psi_i^\lambda}}_{F_4} \,.
	\label{eq:F_split_up}
	\end{multline}
	%
	The expression for $F_4$ can easily be obtained from
	Eqs.~\eqref{eq:phi_t_g0_J0_minus} and~\eqref{eq:J0_minus_twice_relation} by
	replacing the deuteron wave function and the $t$-matrix by their evolved
	counterparts.  As before, we insert complete sets of partial-wave basis states
	using Eq.~\eqref{eq:completeness_partial_wave} and evaluate $F_3$, $F_2$,
	and $F_1$; see Eqs.~\eqref{eq:F3}, \eqref{eq:F2}, and \eqref{eq:F1}.
	%
	Figures in Subsec.~\ref{subsec:Results} compare $\fL$ calculated
	from the matrix element with all components evolved to the unevolved $\fL$.
	We find an excellent
	agreement, validating the expressions for $F_1, \ldots, F_4$.

	\medskip

	\subsubsection{First-order analytical calculation}

	Recall that from Eqs.~\eqref{eq:f_L_from_T} and~\eqref{eq:T_definiton} we have
	%
	\begin{equation}
	 \fL \propto \sum_{\msf, \mJd}\left|\mbraket{\psi_f}{J_0}{\psi_i}\right|^2 \,.
	\label{eq:fl_prop_matrix_element}
	\end{equation}
	%
	When all three components---the final state, the current, and the initial
	state---are evolved consistently, then $\fL$ is unchanged.  However, if we
	miss evolving a component, then we obtain a different result.  It is
	instructive to
	illustrate this through a first-order analytical calculation.%
	\footnote{An analogous calculation based on field redefinitions appears
	in Ref.~\cite{Furnstahl:2001xq}.}

	Let us look at the effects due to the evolution of individual components for a
	general matrix element $\mbraket{\psi_f}{\widehat{O}}{\psi_i}$.  The evolved
	initial state is given by
	%
	\begin{equation}
	 \ket{\psi_i^\lambda} \equiv U \, \ket{\psi_i}
	 = \ket{\psi_i} + \widetilde{U} \, \ket{\psi_i} \,,
	\label{eq:psi_i_evolution}
	\end{equation}
	%
	where $\widetilde{U}$ is the smooth part of the $U$-matrix defined in
	Eq.~\eqref{eq:U_decomposition}.  Similarly, we can write down the expressions
	for the evolved final state and the evolved operator as
	%
	\begin{equation}
	 \bra{\psi_f^\lambda} \equiv \bra{\psi_f} \, U^\dag
	 = \bra{\psi_f} - \bra{\psi_f}\,\widetilde{U}
	\label{eq:psi_f_evolution}
	\end{equation}
	%
	and
	%
	\begin{equation}
	 \widehat{O}^\lambda
	 \equiv U \, \widehat{O} \, U^\dag = \widehat{O}
	 + \widetilde{U}\, \widehat{O} - \widehat{O} \, \widetilde{U}
	 + \mathcal{O}(\widetilde{U}^2) \,.
	\label{eq:O_evolution}
	\end{equation}
	%
	We assume here that $\widetilde{U}$ is small compared to $I$ (which can always
	be ensured by choosing the SRG $\lambda$ large enough) and therefore keep
	terms	only up to linear order in $\widetilde{U}$.  Using
	Eqs.~\eqref{eq:psi_i_evolution}, \eqref{eq:psi_f_evolution},
	and~\eqref{eq:O_evolution}, we get an expression for the evolved matrix
	element	in terms of the unevolved one and changes to individual components
	due to evolution:
	%
	\begin{multline}
	 \mbraket{\psi_f^\lambda}{\widehat{O}^\lambda}{\psi_i^\lambda}
	 = \mbraket{\psi_f}{\widehat{O}}{\psi_i} - \underbrace{\mbraket{\psi_f}
	  {\widetilde{U} \, \widehat{O}}{\psi_i}}_{\delta \bra{\psi_f}} \\
	 \null + \underbrace{\mbraket{\psi_f}{\widetilde{U} \, \widehat{O}}{\psi_i}
	 - \mbraket{\psi_f}{\widehat{O} \, \widetilde{U}}{\psi_i}}_{\delta\widehat{O}}
	 + \underbrace{
	  \mbraket{\psi_f} {\widehat{O} \, \widetilde{U}}{\psi_i}
	 }_{\delta\ket{\psi_i}}
	\end{multline}
	%
	\begin{equation}
	\Longrightarrow \mbraket{\psi_f^\lambda}{\widehat{O}^\lambda}{\psi_i^\lambda}
	 = \mbraket{\psi_f}{\widehat{O}}{\psi_i} + \mathcal{O}(\widetilde{U}^2) \,.
	\end{equation}
	%
	We see that the change due to evolution in the operator is equal and opposite
	to the sum of changes due to the evolution of the initial and final states.
	We also find that changes in each of the components are of the same order,
	and	that they mix;  this feature persists to higher order.  Therefore, if one
	misses evolving an individual component, one will not reproduce the unevolved
	answer.  It is interesting to analyze how this is a function of kinematics
	and will be a subject of Subsec.~\ref{subsec:Results}.


	\subsection{Numerical implementation}
	\label{subsec:numerical_implementation}

	There are various practical issues in the calculation of evolved matrix
	elements that are worth detailing.  We use C++11 for our numerical
	implementation of	the expressions discussed in the previous section.  Matrix
	elements with a	significant	number of components evolved are computationally
	quite expensive due to a large number of nested sums and integrals (see in
	particular Appendix~\ref{Appendix:sec:evolution_expressions}).

	The deuteron wave function and $NN$ $t$-matrix are obtained by discretizing
	the	Schr\"{o}dinger and Lippmann--Schwinger equations, respectively; these
	equations
	are also used to interpolate the $t$-matrix and wave function to points not on
	the discretized mesh.  For example, if we write the momentum-space Schrödinger
	equation---neglecting channel coupling here for simplicity---as
	%
	\begin{eqnarray}
	 \psi(p) =& \int \dd q\,q^2\,G_0(-E_B,q) V(p,q)\,\psi(q) \nonumber \\
	 \rightarrow& \sum_{i} w_i\,q_i^2\,G_0(-E_B,q_i) V(p,q_i)\,\psi(q_i) \,,
	\label{eq:SG-simple}
  \end{eqnarray}
	%
	it can be solved numerically as a simple matrix equation by setting
	$p\in\{q_i\}$.  For any $p=p_0$ not on this mesh, the sum in
	Eq.~\eqref{eq:SG-simple} can then be evaluated to get $\psi(p_0)$.  This
	technique is based on what has been introduced in connection with
	contour-deformation methods in break-up scattering
	calculations~\cite{Hetherington:1965zza,Schmid:1974}.  For more details on
	interpolation of the $t$-matrix and wave function, please refer to
	Appendix~\ref{Appendix:t_matrix_details}.

	To interpolate the potential, which is stored on a momentum-space grid, we use
	the  two-dimensional cubic spline algorithm from ALGLIB~\cite{ALGLIB:0915}.
	In order to avoid unnecessary recalculation of expensive quantities---in
	particular of the off-shell $t$-matrix---while still maintaining an
	implementation very	close to the expressions given in this paper, we make use
	of transparent caching
	techniques.\footnote{This means that the expensive calculation is only carried
	out once, the first time the corresponding function is called for a given set
	of arguments, while subsequent calls with the same arguments return the result
	directly, using a fast lookup.  All this is done without the \emph{calling}
	code being aware of the caching details.}  For most integrations, in
	particular those involving a  principal value, we use straightforward nested
	Gaussian quadrature
	rules; only in a few cases did we find it more efficient to use adaptive
	routines for multi-dimensional integrals.

	With these optimizations, the calculations can in principle still be run on a
	typical laptop computer.  In practice, we find it more convenient to use a
	small cluster, with parallelization implemented using the TBB
	library~\cite{TBB:0915}.  On a node with 48 cores, generating data for a
	meaningful plot (like those shown in Subsec.~\ref{subsec:Results}) can then
	be done	in less than an hour.  For higher resolution and accuracy, we used
	longer runs	with a larger number of data and integration mesh points.

	\subsection{Results}
	\label{subsec:Results}

	For our analysis, we studied the effect of evolution of individual components
	on $\fL$ for selected kinematics in the ranges $E^\prime = 10$--$100~\MeV$ and
	$\mbf{q}^2 = 0.25$--$25~\fm^{-2}$, where $E^\prime$ is the energy of outgoing
	nucleons and $\mbf{q}^2$ is the three-momentum transferred by the virtual
	photon; both are taken in the center-of-mass frame of the outgoing nucleons.
	This range was chosen to cover a variety of kinematics and motivated by the
	set covered in Ref.~\cite{Yang:2013rza}.  We use the Argonne $v_{18}$
	potential
	(AV18)~\cite{Wiringa:1994wb} for our calculations.  It is one of the widely
	used potentials for nuclear few-body reaction calculations, particularly those
	involving large momentum transfers~\cite{Carlson:1997qn,Carlson:2014vla}.

	How strong the evolution of individual components (or a subset thereof)
	affects	the result for $\fL$ depends on the kinematics.  One kinematic
	configuration of particular interest is the so-called quasi-free ridge.  As
	discussed in Subsec.~\ref{subsec:formalism}, the four-momentum transferred by
	the	virtual photon in the center-of-mass frame is $(\omega, \mbf{q})$.  The
	criterion for a configuration to lie on the quasi-free ridge is $\omega = 0$.
	Physically, this means that the nucleons in the deuteron are on their mass
	shell.  As shown in Ref.~\cite{Yang:2013rza}, at the quasi-free ridge the
	energy of the outgoing nucleons ($E^\prime$) and the photon momentum transfer
	are related by
	%
	\begin{equation}
	 \Ep = \sqrt{M_d^2 + \mbf{q}^2} - 2 M \,,
	 \label{eq:quasi_free_condition_exact}
	\end{equation}
	%
	which reduces to
	%
	\begin{equation}
	E^\prime~\text{(in~$\MeV$)} \approx 10 \, \mbf{q}^2~\text{(in~$\fm^{-2}$)} \,.
	\label{eq:quasi_free_condition}
	\end{equation}
	%
	The quasi-free
	condition in the center-of-mass frame is the same as the quasi-elastic
	condition in the lab frame.  There, the quasi-elastic ridge is defined by $W^2
	= m_p^2 \Rightarrow Q^2 = 2 \, \omega_\text{lab} \, m_p$, where $W$ is
	the invariant mass.  On the quasi-elastic ridge, the so-called missing
	momentum%
	\footnote{The missing momentum is defined as the difference of
	the measured proton momentum and the momentum transfer, $\mbf{p}_\text{miss}
	\equiv \mbf{p}_\text{lab}^\text{proton} - \mbf{q}_\text{lab}$.}
	vanishes, $p_\text{miss} = 0$.

	In Fig.~\ref{fig:evolution_at_qfr} we plot $\fL$ along the quasi-free ridge
	both in the impulse approximation (IA) and with the final-state interactions
	(FSI) included as a function of energy of the outgoing nucleons for a fixed
	angle, $\thetacm = 15^{\circ}$ of the outgoing proton.
	$\Ep$ and $\mbf{q}^2$ in Fig.~\ref{fig:evolution_at_qfr} are related by
	Eq.~\eqref{eq:quasi_free_condition_exact}.  Comparing the solid curve labeled
	$\mbraket{\psi_f}{J_0}{\psi_i}$ in the legend to the dashed curve (labeled
	$\mbraket{\phi}{J_0}{\psi_i}$) we find that FSI effects are minimal for
	configurations on the quasi-free ridge especially at large energies.
	%
	\begin{figure}[htbp]
	 \centering
	 \includegraphics[width=0.6\textwidth]%
	 {Factorization/evolution_at_quasi_free_ridge_exact_location_fine_mesh}
	 \caption{ $\fL$ calculated at various points on the quasi-free ridge for
	   $\thetacm = 15^{\circ}$ for the AV18 potential.
	   Legends indicate which component of the matrix element in
	   Eq.~\eqref{eq:fl_prop_matrix_element} used to calculate $\fL$ is evolved.
	   There are no appreciable evolution effects all
	   along the quasi-free ridge.  The effect due to evolution of
	   the final state is small as well and is not shown here to avoid clutter.
	   $\fL$ calculated in the impulse
	   approximation is also shown for comparison.}
	 \label{fig:evolution_at_qfr}
	\end{figure}
	%

	In an intuitive picture, this is because after the initial photon is absorbed,
	both the nucleons in the deuteron are on their mass shell at the quasi-free
	ridge, and therefore no FSI are needed to make the final-state particles real.
	As we move away from the ridge, FSI become more important, as additional
	energy-momentum transfer is required to put the neutron and the proton on
	shell	in the final state.  The difference between full $\fL$ and $\fL$ in IA
	at small
	energies is also seen to hold for few-body nuclei \cite{Bacca:2014tla}.

	Figure~\ref{fig:evolution_at_qfr} also shows $\fL$ calculated from evolving
	only one of the components of the matrix element in
	Eq.~\eqref{eq:fl_prop_matrix_element}.  We note that the effects of SRG
	evolution of the individual components are minimal at the quasi-free ridge as
	well.  The kinematics at the quasi-free ridge are such that only the
	long-range (low-momentum) part of the deuteron wave function is probed, the
	FSI remains	small under evolution, and then unitarity implies minimal
	evolution of the current.  As one moves away from the quasi-free ridge, the
	effects of evolution
	of individual components become prominent.  Note that
	$\mbraket{\psi_f}{J_0}{\psi_i} =
	\mbraket{\psi_f^\lambda}{J_0^\lambda}{\psi_i^\lambda}$ and therefore the
	unevolved vs.\ all-evolved $\fL$ overlap in Fig.~\ref{fig:evolution_at_qfr}.

	%
	\begin{figure}[htbp]
	 \centering
	 \includegraphics[width=0.6\textwidth]%
	 {Factorization/More-1p5}
	 \caption{`Phase space' of kinematics for
	 $\lambda = 1.5~\fm^{-1}$.  The effects of evolution
	 get progressively prominent as one moves further away from the quasi-free
	 ridge. The kinematics of the labeled points are considered later}
	 \label{fig:More-1p5}
	\end{figure}
	%
	Figure~\ref{fig:More-1p5} shows the `phase space' of kinematics for SRG
	$\lambda = 1.5~\rm{fm^{-1}}$.  The quasi-free ridge is along the solid line
	in Fig.~\ref{fig:More-1p5}.  In the shaded region the effects generated by the
	evolution of individual components are weak (only a few percent
	relative difference).
	As one moves away from the quasi-free ridge, these differences get
	progressively	more prominent.
	The terms `small' and `weak' in Fig.~\ref{fig:More-1p5} are used in a
	qualitative
	sense.  In the shaded region denoted by `weak effects', the effects of evolution
	are not easily discernible on a typical $\fL$ versus $\thetacm$ plot, as
	seen in Fig.~\ref{fig:100_10_quasi_free_fsi}, whereas in the region
	labeled by `strong effects', the differences due to evolution are evident on
	a plot (e.g., see Fig.~\ref{fig:30_16_fsi}).
	The size of the shaded region in Fig.~\ref{fig:More-1p5} depends
	on the SRG $\lambda$.  It is large for high $\lambda$'s and gets smaller as
	the	$\lambda$ is decreased (note that smaller SRG $\lambda$ means greater
	evolution).  Next, we look in detail at a few representative
	kinematics, indicated by points in Fig.~\ref{fig:More-1p5}.

  \medskip

	\subsubsection{At the quasifree ridge}

	As a representative of quasi-free kinematics, we choose
	$E^\prime = 100~\MeV$ and $\mbf{q}^2 = 10~\fm^{-2}$ and plot $\fL$ as a
	function of angle in Fig.~\ref{fig:100_10_quasi_free_fsi}.  The effect of
	including FSI is small for this configuration for all
	angles.  Also, the effects due to evolution of the individual components are
	too small to be discernible.  All this is consistent with the discussion in
	the	previous section.
	%
	\begin{figure}[htbp]
	 \centering
	 \includegraphics[width=0.6\textwidth]%
	 {Factorization/Dis-av18_1p5-100_10-4_50_30}
	 \caption{
	   $\fL$ calculated for $E^\prime = 100~\MeV$ and $\mbf{q}^2 = 10~\fm^{-2}$
	   (point ``1'' in Fig.~\ref{fig:More-1p5}) for the AV18 potential.
	   Legends indicate which component of the matrix element in
	   Eq.~\eqref{eq:fl_prop_matrix_element} used to calculate $\fL$ is evolved.
	   $\thetacm$ is the angle of the outgoing proton in the center-of-mass frame.
	   There are no discernible evolution effects for all
	   angles.  The effect due to evolution
	   of the final state is small as well and is not shown here to avoid clutter.
	   $\fL$ calculated in the IA,
	   $\mbraket{\phi}{J_0}{\psi_i}$, is also shown for comparison.
	   }
	 \label{fig:100_10_quasi_free_fsi}
	\end{figure}
	%

	\medskip
	\subsubsection{Near the quasi-free ridge}

	Next we look at the kinematics $E^\prime = 10~\MeV$ and $\mbf{q}^2 =
	4~\fm^{-2}$, which is near the quasi-free ridge.  This is the point ``2'' in
	Fig.~\ref{fig:More-1p5}.  As seen in Fig.~\ref{fig:10_4_fsi}, the different
	curves for $\fL$ obtained from evolving different components start to diverge.
	Figure~\ref{fig:10_4_fsi} also shows $\fL$ calculated in IA.  Comparing this
	to the full $\fL$ including FSI, we see that the effects due to evolution are
	small	compared to the FSI contributions.  This smallness prevents us from
	making any
	systematic observations about the effects due to evolution at this kinematics.
	We thus move on to kinematics which show more prominent effects.
	%
	\begin{figure}[htbp]
	 \centering
	 \includegraphics[width=0.6\textwidth]%
	 {Factorization/Dis-av18_1p5-10_4-4_50_30}
	 \caption{
		 $\fL$ calculated for $E^\prime = 10~\MeV$ and $\mbfq^2 = 4~\fm^{-2}$
		 (point ``2'' in Fig.~\ref{fig:More-1p5}) for the AV18 potential.
		 Legends indicate which component of the matrix element in
		 Eq.~\eqref{eq:fl_prop_matrix_element} used to calculate $\fL$ is evolved.
		 $\fL$ calculated in the IA, $\mbraket{\phi}{J_0}{\psi_i}$,
		 is also shown for comparison.  The effects due to evolution of individual
		 components on $\fL$ are discernible, but still small (compared to the FSI
		 contribution).  The effect due to evolution of the final state is small as
		 well and is not shown here to avoid clutter. }
	 \label{fig:10_4_fsi}
	\end{figure}

	\medskip
	\subsubsection{Below the quasi-free ridge}

	We next look in the region where $E^{\prime}~\text{(in~$\MeV$)}
	\ll 10\,\mbfq^2~\text{(in~$\fm^{-2}$)}$, \ie, below the quasi-free ridge in
	Fig.~\ref{fig:More-1p5}.  We look at two momentum transfers
	$\mbfq^2 = 16~\fm^{-2}$ and $\mbfq^2 = 25~\fm^{-2}$ for $\Ep = 30~\MeV$,
	which are points ``3'' and ``$3^\prime$'' in Fig.~\ref{fig:More-1p5}.
	Figures~\ref{fig:30_16_fsi} and~\ref{fig:30_25_fsi} indicate the effects on
	$\fL$	from evolving individual components of the matrix elements.
	It is noteworthy that in both cases evolution of the current gives a
	prominent enhancement, whereas evolution of the initial and final state gives
	a suppression.  When
	all	the components are evolved consistently, these changes combine and we
	recover	the unevolved answer for $\fL$.  This verifies the accurate
	implementation of	the equations derived in
	Subsec.~\ref{subsec:evolution_setup}.
	%
	\begin{figure}[htbp]
	 \centering
	 \includegraphics[width=0.6\textwidth]
	 {Factorization/Dis-av18_1p5-30_16-4_50_30}
	 \caption{$\fL$ calculated for
	 $E^\prime = 30~\MeV$ and $\mbfq^2 = 16~\fm^{-2}$
	 (point ``3'' in Fig.~\ref{fig:More-1p5})
	 for the AV18 potential.
	 Legends indicate which component of the matrix element in
	 Eq.~\eqref{eq:fl_prop_matrix_element} used to calculate $\fL$ is evolved.
	 Prominent enhancement with evolution of the
	 current only and suppression with evolution of the initial state and the final
	 state only, respectively.}
	 \label{fig:30_16_fsi}
	\end{figure}
	%
	\begin{figure}[htbp]
	 \centering
	 \includegraphics[width=0.6\textwidth]%
	 {Factorization/Dis-av18_1p5-30_25-4_50_30}
	 \caption{
	 	 $\fL$ calculated for $E^\prime = 30~\MeV$ and $\mbfq^2 = 25~\fm^{-2}$
		 (point ``3$^{\prime}$'' in Fig.~\ref{fig:More-1p5})
		 for the AV18 potential.
		 Legends indicate which component of the matrix element in
		 Eq.~\eqref{eq:fl_prop_matrix_element} used to calculate $\fL$ is evolved.
		 Prominent enhancement with evolution of the
		 current only and suppression with evolution of the initial state and the
		 final state only, respectively.}
	 \label{fig:30_25_fsi}
	\end{figure}
	%

	It is possible to qualitatively explain the behavior seen in
	Figs.~\ref{fig:30_16_fsi} and~\ref{fig:30_25_fsi}.  As noted in
	Eq.~\eqref{eq:overlap_IA_plus_FSI}, the overlap matrix element is given by the
	sum of the IA part and the FSI part.  Below the quasi-free ridge these two
	terms add constructively.  In this region, $\fL$ calculated in impulse
	approximation is smaller than $\fL$ calculated by including the final-state
	interactions.

	\paragraph{\emph{(a) Evolving the initial state}} Let us first consider the
	effect of evolving the initial state only.  We have
	%
	\begin{equation}
	 \mbraket{\psi_f}{J_0}{\psi_i^\lambda}
	 = \mbraket{\phi}{J_0}{\psi_i^\lambda}
	 + \mbraket{\phi}{t^\dag \, \GreensFn^\dag \, J_0}{\psi_i^\lambda} \,.
	\label{eq:overlap_IA_plus_FSI_evol_wf}
	\end{equation}
	%
	As seen in Eq.~\eqref{eq:overlap_IA}, in the term
	$\mbraket{\phi}{J_0}{\psi_i^\lambda}$ the deuteron wave function is
	probed between $|\pp - q/2|$ and $\pp + q/2$.  These numbers are
	$(1.2, 2.9)~\fm^{-1}$ and $(1.7, 3.4)~\fm^{-1}$ for $\Ep = 30~\MeV$,
	$\mbfq^2 = 16~\fm^{-2}$ and $\Ep = 30~\MeV$, $\mbfq^2 = 25~\fm^{-2}$,
	respectively.  The evolved deuteron wave function is significantly suppressed
	at these high momenta.
	This behavior is reflected in the deuteron momentum distribution plotted in
	Fig.~\ref{fig:deut_md}.  The deuteron momentum distribution $n(k)$ is
	proportional to the sum of the squares of $S$- and $D$- state deuteron
	wave functions.  Thus, the first (IA) term in
	Eq.~\eqref{eq:overlap_IA_plus_FSI_evol_wf} is much smaller than its
	unevolved counterpart in Eq.~\eqref{eq:overlap_IA_plus_FSI}, for all angles.
	We note that even though we only use the AV18 potential to
	study changes due to evolution, these changes will be significant for other
	potentials as well.
	%
	\begin{figure}[htbp]
	 \centering
	 \includegraphics[width=0.6\textwidth]%
	 {Factorization/deuteron_momentum_distribution}
	 \caption{
		 Momentum distribution for the deuteron for the AV18~\cite{Wiringa:1994wb},
		 CD-Bonn~\cite{Machleidt:2000ge},
		 and the Entem-Machleidt N$^3$LO chiral EFT~\cite{Entem:2003ft}
		 potentials, and for the AV18 potential evolved to two SRG $\lambda$'s.}
	 \label{fig:deut_md}
	\end{figure}
	%

	Evaluation of the second (FSI) term in
	Eq.~\eqref{eq:overlap_IA_plus_FSI_evol_wf} involves an integral over all
	momenta, as indicated in Eq.~\eqref{eq:phi_t_g0_J0_minus}.  We find that
	$|\mbraket{\phi}{t^\dag \, \GreensFn^\dag \, J_0}{\psi_i^\lambda}| <
	|\mbraket{\phi}{t^\dag \, \GreensFn^\dag \, J_0}{\psi_i}|$.  As mentioned
	before, because the terms $\mbraket{\phi}{J_0}{\psi_i}$ and
	$\mbraket{\phi}{t^\dag \, \GreensFn^\dag \, J_0}{\psi_i}$ add constructively
	below the quasi-free ridge and because the magnitude of both these terms
	decreases upon evolving the wave function, we have
	%
	\begin{equation}
	 |\mbraket{\psi_f}{J_0}{\psi_i^\lambda}| < |\mbraket{\psi_f}{J_0}{\psi_i}|
	\label{eq:evolv_wf_below_qfr} \,.
	\end{equation}
	%
	The above relation holds for most combinations of $\mJd$ and $\msf$.  For
	those	$\mJd$ and $\msf$ for which Eq.~\eqref{eq:evolv_wf_below_qfr} does not
	hold,	the absolute value of the matrix element is much smaller than for those
	for	which the Eq.~\eqref{eq:evolv_wf_below_qfr} \emph{does} hold, and
	therefore we have $\fL$ calculated from $\mbraket{\psi_f}{J_0}
	{\psi_i^\lambda}$ smaller than the
	$\fL$ calculated from $\mbraket{\psi_f}{J_0}{\psi_i}$, as seen in
	Figs.~\ref{fig:30_16_fsi} and~\ref{fig:30_25_fsi}.

	\paragraph{\emph{(b) Evolving the final state}}  As indicated in
	Eq.~\eqref{eq:psi_f_lam_def}, evolving the final state entails
	the evolution of the $t$-matrix.  The overlap matrix element therefore is
	%
	\begin{equation}
	 \mbraket{\psi_f^\lambda}{J_0}{\psi_i}
	 = \mbraket{\phi}{J_0}{\psi_i}
	 + \mbraket{\phi}{t^\dag_\lambda \, \GreensFn^\dag \, J_0}{\psi_i} \,.
	\label{eq:evol_final_state_overlap}
	\end{equation}
	%
	The IA term is the same as in the unevolved case.  The SRG evolution leaves the
	on-shell part of the $t$-matrix---which is directly related to
	observables---invariant.  The magnitude of the relevant off-shell $t$-matrix
	elements decreases
	on evolution, though.  As a result we have
	%
	\begin{equation}
	|\mbraket{\psi_f^\lambda}{J_0}{\psi_i}| < |\mbraket{\psi_f}{J_0}{\psi_i}| \,.
	\end{equation}
	%
	This is reflected in $\fL$ as calculated from the evolved final state, and
	seen in Figs.~\ref{fig:30_16_fsi} and~\ref{fig:30_25_fsi}.

	The effect of evolution of the initial state and the final state is to suppress
	$\fL$.  When all the three components are evolved, we reproduce the unevolved
	answer as indicated in Fig.~\ref{fig:30_16_fsi} and~\ref{fig:30_25_fsi}.  It is
	therefore required that we find a huge enhancement when just the current
	is evolved.

	The kinematics $\Ep = 30~\MeV$, $\mbfq^2 = 25~\fm^{-2}$ is further away from
	the quasi-free ridge than $\Ep = 30~\MeV$, $\mbfq^2 = 16~\fm^{-2}$.
	The evolution effects discussed above get
	progressively more prominent the further away one is from the quasifree
	ridge.  This can be verified by
	comparing the effects due to evolution of individual components in
	Figs.~\ref{fig:30_16_fsi} and~\ref{fig:30_25_fsi}.

	As remarked earlier, away from the quasi-free ridge the FSI
	become important.  Nonetheless, it is still instructive to look at $\fL$
	calculated in the IA at these kinematics.
	%
	\begin{figure}[htbp]
	 \centering
	 \includegraphics[width=0.6\textwidth]%
	 {Factorization/Dis-av18_1p5-30_16-4_50_30_IA}
	 \caption{
		 $\fL$ in IA $(\bra{\psi_f} \equiv \bra{\phi})$ calculated for
		 $E^\prime = 30~\MeV$ and $\mbfq^2 = 16~\fm^{-2}$ for the AV18 potential.
		 Legends indicate which component of the matrix element in
		 Eq.~\eqref{eq:fl_prop_matrix_element} used to calculate $\fL$ are evolved.}
	 \label{fig:30_16_ia}
	\end{figure}
	%
	\begin{figure}[htbp]
	 \centering
	 \includegraphics[width=0.6\textwidth]%
	 {Factorization/Dis-av18_1p5-30_25-4_50_30_IA}
	 \caption{
		 $\fL$ in IA $(\bra{\psi_f} \equiv \bra{\phi})$ calculated for
		 $E^\prime = 30~\MeV$ and $\mbfq^2 = 25~\fm^{-2}$ for the AV18 potential.
		 Legends indicate which component of the matrix element in
		 Eq.~\eqref{eq:fl_prop_matrix_element} used to calculate $\fL$ are evolved.}
	 \label{fig:30_25_ia}
	\end{figure}
	%
	Note that the (unevolved) $\fL$ calculated in the IA, shown
	in Figs.~\ref{fig:30_16_ia} and~\ref{fig:30_25_ia}, is smaller than the full
	$\fL$ that takes into account the final state interactions (cf.~the
	corresponding curves in Figs.~\ref{fig:30_16_fsi} and~\ref{fig:30_25_fsi}).
	This is consistent with the claim made earlier that below the quasi-free ridge
	the two terms in Eq.~\eqref{eq:overlap_IA_plus_FSI} add constructively.

	The results in Figs.~\ref{fig:30_16_ia} and~\ref{fig:30_25_ia} can again be
	qualitatively explained based on our discussion above.  The evolution of the
	deuteron wave function leads to suppression as the evolved wave function does
	not have strength at high momentum.  The evolved current thus leads to
	enhancement.  Evolution of both the current and the initial state decreases
	$\fL$ from just the evolved current value, but it is not until we evolve all
	three components---final state, current, and the initial state---that we
	recover the unevolved answer.

	\begin{figure}[htbp]
		\centering
		\begin{subfigure}[c]{0.6\textwidth}
			\centering
			\includegraphics[width=\textwidth]
			{Factorization/Dis-av18_4-30_25-4_40_20}
			\caption{SRG $\lambda = 4 {\rm{~fm}}^{-1}$}
			\label{fig:30_25_lam4}
		\end{subfigure}
		%\hspace{0.05\textwidth}
		\begin{subfigure}[c]{0.6\textwidth}
			\centering
			\includegraphics[width=\textwidth]
			{Factorization/Dis-av18_2-30_25-4_40_20}
			\caption{SRG $\lambda = 2 {\rm{~fm}}^{-1}$}
			\label{fig:30_25_lam2}
		\end{subfigure}
		\caption{$\fL$ calculated for $E^\prime = 30~\MeV$ and $\mbfq^2 = 25~\fm^{-2}$
			(point ``3$^{\prime}$'' in Fig.~\ref{fig:More-1p5})
			for the AV18 potential.
			Legends indicate which component of the matrix element in
			Eq.~\eqref{eq:fl_prop_matrix_element} used to calculate $\fL$ is evolved.
			The evolution is to SRG (a) $\lambda = 4 {\rm{~fm}}^{-1}$ and
			(b) $\lambda = 2 {\rm{~fm}}^{-1}$.}
		\label{fig:30_25_lam_fn}
	\end{figure}
	%
	As expected, the effect due to evolution increases with further evolution.
	This can be seen by comparing the plots in Fig.~\ref{fig:30_25_lam_fn} to
	Fig.~\ref{fig:30_25_fsi}.
	\begin{figure}[htbp]
	 \centering
	 \includegraphics[width=0.6\textwidth]%
	 {Factorization/fl_IA_J0_evolution_vs_lambda}
	 \caption{
		 $\fL$ in IA calculated at $\thetacm = 15^{\circ}$ for
		 $E^\prime = 30~\MeV$ and $\mbfq^2 = 25~\fm^{-2}$
		 for the AV18 potential when the current operator in
		 Eq.~\eqref{eq:fl_prop_matrix_element} used to calculate $\fL$ is evolved to
		 various SRG $\lambda$'s.  The horizontal dotted line is the unevolved
		 answer.}
	 \label{fig:J0_evolution_vs_lambda}
	\end{figure}
	%
	In Fig.~\ref{fig:J0_evolution_vs_lambda} we look at
	effects of the current-operator evolution on $\fL$ as a
	function of the SRG $\lambda$.  To isolate the effect of operator evolution,
	we only look at $\fL$ calculated in IA at a specific angle in
	Fig.~\ref{fig:J0_evolution_vs_lambda}.  Investigating details of the
	operator evolution forms the basis of ongoing work.

	\subsubsection{Above the quasi-free ridge}

	Finally, we look at an example from above the quasi-free ridge.
	Figure~\ref{fig:100_0p5_fsi} shows the effect of evolution of individual
	components on $\fL$ for $\Ep = 100~\MeV$ and $\mbfq^2 = 0.5~\fm^{-2}$, which
	is point ``4'' in Fig.~\ref{fig:More-1p5}.  The effects of evolution in this
	case are qualitatively different from those found below the quasi-free ridge.
	For instance, we see a peculiar suppression in $\fL$ calculated from the
	evolved	deuteron wave function at small angles, but an enhancement at large
	angles.  An	opposite behavior is observed for the final state.  It is again
	possible to	qualitatively explain these findings.
	%
	\begin{figure}[htbp]
	 \centering
	 \includegraphics[width=0.6\textwidth]%
	 {Factorization/Dis-av18_1p5-100_0p5-4_50_30}
	 \caption{
		 $\fL$ calculated for $E^\prime = 100~\MeV$ and
		 $\mbfq^2 = 0.5~\fm^{-2}$ (point ``4'' in Fig.~\ref{fig:More-1p5})
		 for the AV18 potential.
		 Legends indicate which component of the matrix element in
		 Eq.~\eqref{eq:fl_prop_matrix_element} used to calculate $\fL$ is evolved.
		 Opposite effects from the evolution of the
		 initial state and the final state.}
	 \label{fig:100_0p5_fsi}
	\end{figure}
	%

	\paragraph{\emph{(a) Evolving the initial state}}	 Above the quasi-free ridge,
	the IA and FSI terms in
	Eq.~\eqref{eq:overlap_IA_plus_FSI} add destructively.  This can be seen by
	comparing the unevolved $\fL$ curves in Figs.~\ref{fig:100_0p5_fsi}
	and~\ref{fig:100_0p5_ia}.  Including the FSI brings down
	the value of $\fL$ when one is above the quasi-free ridge.

	At small angles, the magnitude of the IA term in
	Eq.~\eqref{eq:overlap_IA_plus_FSI} is larger than that of the FSI term.  The
	deuteron wave function for this kinematics is probed between $1.2$ and
	$1.9~\fm^{-1}$.  With the wave-function evolution, the magnitude of the IA
	term in Eq.~\eqref{eq:overlap_IA_plus_FSI_evol_wf} decreases, whereas the
	magnitude of the FSI term in that equation slightly increases compared to its
	unevolved counterpart.  Still, at small angles, we have
	$|\mbraket{\phi}{J_0}{\psi_i^\lambda}| > |\mbraket{\phi}{t^\dag \,
	\GreensFn^\dag \, J_0}{\psi_i^\lambda}|$, which leads to
	%
	\begin{equation}
	 |\mbraket{\psi_f}{J_0}{\psi_i^\lambda}| < |\mbraket{\psi_f}{J_0}{\psi_i}| \,,
	 \label{eq:psi_i_evolution_small_angles_above_qfr}
	\end{equation}
	%
	and thus to the suppression of $\fL$ at small angles observed in
	Fig.~\ref{fig:100_0p5_fsi}.

	At large angles, the magnitude of the IA term in
	Eq.~\eqref{eq:overlap_IA_plus_FSI} is smaller than that of the FSI term.
	With the wave-function evolution, the magnitude of IA term decreases
	substantially (large momenta in the deuteron wave function are probed at
	large	angles, cf.~Eq.~\eqref{eq:overlap_IA}), whereas the FSI term in
	Eq.~\eqref{eq:overlap_IA_plus_FSI} remains almost the same.  This results in
	increasing the difference between the two terms in
	Eq.~\eqref{eq:overlap_IA_plus_FSI} as the SRG $\lambda$ is decreased.  As
	mentioned before, above the quasi-free ridge, the IA and FSI terms
	in Eq.~\eqref{eq:overlap_IA_plus_FSI} add destructively and we therefore end
	up with $|\mbraket{\psi_f}{J_0}{\psi_i^\lambda}| >
	|\mbraket{\psi_f}{J_0}{\psi_i}|$, leading to the observed enhancement at large
	angles upon evolution of the wave function (see Fig.~\ref{fig:100_0p5_fsi}).

	\paragraph{\emph{(b) Evolving the final state}}

	The expression to consider is Eq.~\eqref{eq:evol_final_state_overlap}.  With
	the evolution of the $t$-matrix, the magnitude of the term
	$\mbraket{\phi}{t^\dag_\lambda \, \GreensFn^\dag \, J_0}{\psi_i}$
	decreases, and because of the opposite relative signs of the two terms in
	Eq.~\eqref{eq:evol_final_state_overlap}---and
	because at small angles the magnitude of the IA term is larger than the
	FSI term---the net
	effect is $|\mbraket{\psi_f^\lambda}{J_0}{\psi_i}| >
	|\mbraket{\psi_f}{J_0}{\psi_i}|$.  This leads to an enhancement of $\fL$ with
	evolved final state at small angles, as seen in Fig.~\ref{fig:100_0p5_fsi}.

	At large angles the magnitude of the IA term in
	Eq.~\eqref{eq:evol_final_state_overlap} is smaller than that of the FSI term.
	With the evolution of the $t$-matrix, the magnitude of the FSI term decreases
	and the difference between the IA and the FSI terms decreases as well.  This
	leads to the observed overall suppression in $\fL$ at large angles due to the
	evolution of the final state seen in Fig.~\ref{fig:100_0p5_fsi}.  For those
	few	($\msf$, $\mJd$) combinations for which the above general observations do
	not	hold, the value of individual components is too small to make any
	qualitative	difference.

	%
	\begin{figure}[htbp]
	\centering
	\includegraphics[width=0.6\textwidth]
	{Factorization/Dis-av18_1p5-100_0p5-4_50_30_IA}
	\caption{
	 $\fL$ in IA $(\bra{\psi_f} \equiv \bra{\phi})$ calculated for
	 $E^\prime = 100~\MeV$ and $\mbfq^2 = 0.5~\fm^{-2}$
	 for the AV18 potential.
	 Legends indicate which component of the matrix element in
	 Eq.~\eqref{eq:fl_prop_matrix_element} used to calculate $\fL$ are evolved.}
	\label{fig:100_0p5_ia}
	\end{figure}
	%
	Figure~\ref{fig:100_0p5_ia} shows the effect of evolution of individual
	components on $\fL$ calculated in the IA for the kinematics
	under consideration.  Again the evolved deuteron wave function does not have
	strength at high momenta and therefore $\fL$ calculated from
	$\mbraket{\phi}{J_0}{\psi_i ^\lambda}$ has a lower value than its unevolved
	counterpart.

	Unitary evolution means that the effect of the evolved current is always
	such that it compensates the effect due to the evolution of the initial and
	final states.  Our ongoing work examines more directly the behavior of the
	current as it evolves to better understand how to carry over the results
	observed here to other reactions.

	\section{Summary and Outlook}

	Nuclear properties such as momentum distributions are extracted from
	experiment by invoking the factorization of structure, which includes
	descriptions of	initial and final states, and reaction, which includes the
	description of the probe components.
	%
	The factorization between reaction and structure depends on the scale and
	scheme chosen for doing calculations.  Unlike in high-energy QCD, this scale
	and scheme dependence of factorization is often not taken into account in
	low-energy nuclear physics calculations, but is potentially critical for
	interpreting experiment.
	%
	In our work we investigated this issue by looking at the simplest knockout
	reaction: deuteron electrodisintegration.  We used SRG transformations to test
	the sensitivity of the longitudinal structure function~$\fL$ to evolution of
	its	individual components: initial state, final state, and the current.

	We find that the effects of evolution depend on kinematics,
	but in a \emph{systematic} way.  Evolution effects are negligible at the
	quasi-free ridge,	indicating that the
	scale dependence of individual components is minimal there.  This is
	consistent with the quasi-free ridge mainly probing the long-range part of
	the	wave function, which is largely invariant under SRG evolution.  This is
	also the region where contributions from FSI to $\fL$ are minimal.
	%
	The effects get progressively more pronounced the further one moves away from
	the quasifree ridge.  The nature of these changes depends on whether one
	is above or below the quasifree ridge in the ‘“phase-space’” plot
	(Fig.~\ref{fig:More-1p5}).
	As indicated in Subsec.~\ref{subsec:Results}, these changes can also be
	explained qualitatively
	by looking at the overlap matrix elements.  This allows us to predict the
	effects due to evolution depending on kinematics.
	%

	Our results demonstrate that scale dependence needs to be taken into account
	for low-energy nuclear calculations.  While we showed this explicitly only
	for	the case of the longitudinal structure function in deuteron
	disintegration,	we expect the results should qualitatively
	carry over for other knock-out reactions as well.
	An area of active investigation is the extension of the formalism presented
	here to hard scattering processes.


	SRG transformations are routinely used in nuclear structure calculations
	because	they lead to accelerated convergence for observables like binding
	energies.  We
	demonstrated that SRG transformations can be used for nuclear knock-out
	reactions as well as long as the operator involved is also consistently
	evolved. 	The evolved operator appears to be complicated compared to the
	unevolved one,
	but the factorization might be cleaner with the evolved operator if one can
	exploit an operator product expansion~\cite{Anderson:2010aq,Bogner:2012zm}.
	This forms the basis of our ongoing work.

	We also plan to use pionless EFT as a framework to quantitatively study the
	effects of operator evolution.  It should be a good starting point to
	understand in detail how a one-body operator develops strength in two- and
	higher-body sectors upon evolution.  This can give insight on the issue of
	power	counting of operator evolution.  Pionless EFT has been employed
	previously to
	study deuteron electrodisintegration in Ref.~\cite{Christlmeier:2008ye}, where
	it was used to resolve a discrepancy between theory and experiment.

	Extending our work to many-body nuclei requires inclusion of 3N forces and 3N
	currents.  Consistent evolution in that case would entail evolution in both
	two	and three-body sectors.  However, SRG transformations have proven to be
	technically feasible for evolving three-body
	forces~\cite{Jurgenson:2009qs,Jurgenson:2010wy,Hebeler:2012pr,Wendt:2013bla}.
	Thus, extending our calculations to many-body nuclei would be computationally
	intensive, but is feasible in the existing framework.  Including the effects
	of FSI is challenging for many-body systems and has been
	possible only recently for light nuclei \cite{Bacca:2014tla, Lovato:2015qka}.
	It would be interesting to investigate if the scale and scheme dependence of
	factorization allows us to choose a scale where the FSI effects are minimal.
