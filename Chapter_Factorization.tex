% !TEX root = More_PhD_Thesis.tex
\cleardoublepage
\chapter[Factorization]{Factorization
	\footnote{Based on \cite{More:2015tpa}}}

	\section{Motivation}

	Most of the information we know about nuclear interactions and the properties
	of the nuclei comes from some kind of scattering experiments (either elastic
	or inelastic).  In such experiments we scatter a known probe off a nucleus
	and	extract information about nuclear interactions by looking at the final
	outcome of the scattering experiment.
	%
	\begin{figure}[thbp]
	\centering
	\includegraphics[width=0.5\textwidth]%
	{Factorization/knock_out_schematic.png}
	\caption{Schematic of a nucleon knockout reaction.}
	\label{fig:knock_out_schematic}
	\end{figure}
	%
	\begin{figure}[thbp]
	\centering
	\includegraphics[width=0.6\textwidth]%
	{Factorization/factorization_schematic_annotated}
	\caption{Schematic illustration of factorization between nuclear structure
		and reactions component.}
	\label{fig:factorization_schematic}
	\end{figure}
	%
	Figure~\ref{fig:knock_out_schematic} shows a schematic for a nucleon knockout
	reaction where the probe is electrons which in turn interact with the nucleus
	by emitting virtual photons.

	The process of extracting nuclear properties from such experiments relies
	on the assumption that the effects of the probe are well understood and can
	be separated from the nuclear interactions we are trying to study.  This is
	the	\emph{factorization} between the nuclear structure and the nuclear
	reaction components illustrated schematically in
	Fig.~\ref{fig:factorization_schematic}.  The reaction component describes the
	probe and the structure includes description of the initial and final states.
	This factorization between the structure and reaction components depends
	on the renormalization scale and scheme.  In some physical systems
	(e.g., in cold atoms near unitarity~\cite{Hoinka:2013fsa}), the scale and
	scheme dependence is very weak and can be safely neglected.  In some other
	physical system such as in deep inelastic scattering (DIS) in high-energy
	QCD, the scale and scheme dependence is very manifest.
	%
	\begin{figure}[thbp]
	\centering
	\includegraphics[width=0.55\textwidth]%
	{Factorization/factorization_schematic_highE_QCD}
	\caption{Factorization in high-energy QCD.  $x$ is the Bjorken-$x$ and it
		denotes the fraction of momentum of the nucleon carried by the parton
		under consideration.  $a$ denotes the parton flavor.}
	\label{fig:factorization_schematic_highE_QCD}
	\end{figure}
	%
	Figure~\ref{fig:factorization_schematic_highE_QCD} illustrates the
	factorization in DIS.  The form factor $F_2$ of the nucleon
	(which up to some kinematic factors is the cross section) is given by the
	convolution of the long-distance parton density and the short-distance
	Wilson coefficient.  In this case, the parton density forms the structure
	part which is non-perturbative and the Wilson coefficient form the reactions
	part which can be calculated in perturbative QCD.  These separation
	between long- and short-distance physics is not unique, and is defined by
	the factorization scale $\mu_f$.  To minimize the contribution of logarithms
	that can disturb the perturbative expansion, $\mu_f$ is chosen to be equal to
	the	magnitude of the four-momentum transfer $Q$.
	The form factor $F_2$ (because it is
	related to the observable cross section) is independent of $\mu_f$, but
	the individual components are not.  As a consequence, the parton density
	(or distribution) function $f_a(x, Q^2)$ runs with $Q^2$.
	%
	\begin{figure}[thbp]
	\centering
	\includegraphics[width=0.55\textwidth]%
	{Factorization/uquark_parton_distribution_3d_surf}
	\caption{Parton distribution for the up quarks in the proton as a function
		of $x$ and $Q^2$.  Figure taken from \cite{Furnstahl:2013dsa}.}
	\label{fig:uquark_PDF_3D}
	\end{figure}
	%
	This is demonstrated in Fig.~\ref{fig:uquark_PDF_3D}.  The parton
	distributions
	$f_a(x, Q^2)$ and $f_a(x, Q_0^2)$	at two different $Q^2$ are related by
	DGLAP evolution or the Altarelli-Parisi equations \cite{Altarelli:1977zs}.
	Thus the scale dependence of the structure and reaction components is well
	understood in high-energy QCD.

	The situation is far from well settled in low-energy nuclear physics.
	Nuclear
	structure has conventionally been treated largely separate from nuclear
	reactions
	(e.g., the two volumes of Feshbach's \emph{Theoretical Nuclear Physics} are
	divided
	this way).  The nuclear structure community usually dealt with calculating
	time-independent properties such as nuclear binding energies, excitation
	spectra, radii, so on whereas the nuclear reaction experts worked on
	disintegration,	knock-out and
	transfer reactions.  However, both the communities invariably use inputs
	from the other side, and the consistency and universality of different
	components is not
	always guaranteed. 	To go back to the high-energy QCD analogy, the parton
	distribution functions (PDFs) $f_a(x, Q)$ extracted from the DIS are
	universal, in	the sense that they are process-independent.  For instance,
	the PDFs extracted
	from the DIS can be used for making prediction for the Drell-Yan process.
	The analogous process independence in the extracted quantities has not yet
	been achieved in low-energy nuclear physics.  This leads to ambiguous
	uncertainty	quantification when the nuclear properties extracted from one
	process	(cf.~Fig.~\ref{fig:factorization_schematic}) are used as an input to
	predict something	else.

	The factorization in low-energy nuclear physics is illustrated in
	Fig.~\ref{fig:low_energy_factorization}.
	%
	\begin{figure}[thbp]
	\centering
	\includegraphics[width=0.6\textwidth]%
	{Factorization/low_energy_factorization}
	\caption{Schematic illustration of factorization in low-energy nuclear
		physics.}
	\label{fig:low_energy_factorization}
	\end{figure}
	%
	The observable cross section in this case is a convolution of the
	spectroscopic factor
	and single-particle cross section.  However, there are many open questions
	such as	what are the nuclear properties that we can extract and what is the
	scale/scheme dependence of the extracted properties.

	The Similarity Renormalization Group (or the SRG) transformations were
	introduced in	Subsec.~\ref{subsec:SRG_intro}.  We noted that SRG
	transformations are a class of unitary transformations that soften nuclear
	Hamiltonians and lead to accelerated convergence of observables.
	%
	\begin{figure}[thbp]
	\centering
	\includegraphics[width=0.6\textwidth]%
	{Factorization/vsrg_T_3S1_kvnn_06_md_3dplot3}
	\caption{Deuteron momentum distribution at different SRG resolutions
		$\lambda$.  The evolved momentum distribution does not have the short-range
		correlations (SRCs).  Figure from \cite{Furnstahl:2013dsa}.}
	\label{fig:momentum_distribution}
	\end{figure}
	%
	Figure~\ref{fig:momentum_distribution} shows the momentum distribution for
	the deuteron as a function of the SRG scale and momentum.  Note that
	Fig.~\ref{fig:momentum_distribution} is analogous to
	Fig.~\ref{fig:uquark_PDF_3D}.  The SRG evolution gets rid of the
	high-momentum components and therefore the evolved momentum distributions
	don't have the short-range correlations (SRCs).
	Figure~\ref{fig:momentum_distribution} makes it clear that the
	high-momentum tail of the momentum distribution is dramatically resolution
	dependent.  Yet it is common in the literature that
	high-momentum components are treated as measurable, at least
	implicitly~\cite{Frankfurt:2008zv,Arrington:2011xs,Rios:2013zqa,
	Boeglin:2015cha}.
	In fact, what can be extracted is the momentum distribution at some scale,
	and	with the specification of a scheme.  This makes momentum distributions
	model	dependent~\cite{Ford:2014yua, Sammarruca:2015hba}.

	%
	\begin{figure}[thbp]
	\centering
	\includegraphics[width=0.6\textwidth]%
	{Factorization/wave_functions_evolution}
	\caption{$D$-state wave functions for the deuteron for the AV18 potential and
		the AV18 potential evolved to two SRG $\lambda$'s.}
	\label{fig:wavefunction_evolution_deuteron_D_state}
	\end{figure}
	%
	Figure~\ref{fig:wavefunction_evolution_deuteron_D_state} shows the $D$-state
	wave function for deuteron.  We see that just like the momentum distributions,
	SRG transformed wave functions do not have the high-momentum components.
	Therefore,
	if we use the SRG evolved wave function for calculating cross section for a
	process involving high-momentum probe, then the only way we get the same
	answer as with the unevolved wave function is if the relevant operator
	changed as well.  Thus, with SRG evolution, the high-momentum physics is
	shuffled from the wave function (nuclear structure) to the operator
	(nuclear reaction component).  This is reminiscent of chiral EFTs where the
	renormalization replaces the high-momentum modes in intermediate states
	by contact interactions (see Fig.~\ref{fig:replace_loop_contact}).
	%
	\begin{figure}[thbp]
	\centering
	\includegraphics[width=0.55\textwidth]%
	{Factorization/fig_vsrgreplace3}
	\caption{High-momentum modes in intermediate states replaced by contact
		interactions.  Figure from \cite{Furnstahl:2013dsa}.}
	\label{fig:replace_loop_contact}
	\end{figure}
	%
	Discussion so far shows how SRG makes the scale dependence of factorization
	explicit.  SRG transformations come with the momentum scale $\lambda$ and
	%
	\begin{figure}[thbp]
	\centering
	\begin{overpic}
   [width=0.6\textwidth]%
	 {Factorization/factorization_part.pdf}
   \put(20,33){\textcolor{black}{\large $p < \lambda$}}
   \put(67,33){\textcolor{black}{\large $p > \lambda$}}
   \end{overpic}
	\caption{The SRG scale $\lambda$ sets the natural scale for factorization.}
	\label{fig:SRG_factorization}
	\end{figure}
	%
	this sets the scale for factorization.  As seen in the
	Fig.~\ref{fig:SRG_factorization}, we have a natural separation that the piece
	which involves momenta less than $\lambda$ forms the long-distance part and
	the piece that involves momenta greater than $\lambda$ forms the
	short-distance part.

	Consider the differential cross section given by the overlap matrix element
	of initial and final states.
	\beq
	\frac{d \sigma}{d \Omega} \propto \left|
	\mbraket{\psi_f}{\widehat{O}}{\psi_i} \right|^2 \;.
	\eeq
	The SRG evolved wave function is given by $\ket{\psi_i^\lambda} =
	U_{\lambda} \ket{\psi_i}$, where $U_{\lambda}$ is the unitary matrix
	associated with the SRG transformation.  The cross section is an
	experimental observable and should be independent of our choice of
	the SRG scale.  If we evolve all the components
	of the matrix element consistently
	\beq
	\mbraket{\psi_f}{\wh{O}}{\psi_i} = \mbraket{
	\underbrace{\psi_f U_{\lambda}^\dag}_{\psi_f^\lambda}}
	{\underbrace{U_\lambda \wh{O} U_{\lambda}^{\dag}}_{\wh{O}^\lambda}}
	{\underbrace{U_{\lambda} \psi_i}_{\psi_i^{\lambda}}} \;,
	\label{eq:matrix_element_invariance}
	\eeq
	then the evolved matrix element is same as the unevolved one and the
	observable cross section is unchanged.

	In general, to be consistent between structure and reactions one must
	calculate	cross sections or decay rates within a single framework.
	That is, one must	use
	the same Hamiltonian and consistent operators throughout the calculation
	(which means the same scale and scheme).  Such consistent calculations have
	existed	for some time for few-body nuclei (e.g.,
	see~\cite{Epelbaum:2008ga,Hammer:2012id,Carlson:2014vla,Marcucci:2015rca})
	and	are becoming increasingly feasible for heavier nuclei because of advances
	in reaction technology, such as using complex basis states to handle
	continuum	physics.  Recent examples in the literature include
	No Core Shell Model Resonating Group Method
	(NCSM/RGM)~\cite{Quaglioni:2015via}, coupled cluster~\cite{Bacca:2013dma},
	and lattice EFT calculations~\cite{Pine:2013zja}.
	But there are many open	questions about constructing consistent currents and
	how to compare results from two such calculations.
	Some work along this direction which includes the evolution of the operator
	has recently been done.  Anderson \etal looked at the static properties
	of the deuteron such as momentum distributions, radii, and form factors under
	SRG evolution; they found no pathologies in the evolved operators, and
	the evolution effects were small for low-momentum observables
	\cite{Anderson:2010aq}.  Schuster \etal found in their work on radii and
	dipole transition matrix elements in light nuclei that the evolution effects
	are as important as three-body forces
	\cite{Schuster:2014lga,Schuster:2013sda}.
	Neff \etal looked at the SRG transformed density operators and concluded that
	it is essential to use evolved operators for observables sensitive to
	short-range physics \cite{Neff:2015xda}.  But all this work was done for
	expectation values of the operator, i.e, the state on the either side of the
	matrix element was the same.  In particular, there was no work which dealt
	with the issues related to operator evolution when we have a transition to
	continuum.  This is what we sought to address in \cite{More:2015tpa}.

	The electron scattering knock-out process is particularly interesting because
	of the connection to past, present, and planned
	experiments~\cite{Boffi:1996, LENP_white_paper2015}.
	The conditions for clean factorization
	of structure and reactions in this context is closely related to the impact of
	3N forces,
	two-body currents, and final-state interactions, which have not been cleanly
	understood as yet~\cite{Furnstahl:2010wd}.
	All of this becomes particularly relevant for high-momentum-transfer electron
	scattering.%
	\footnote{Note that high-momentum transfers imply high-resolution
	\emph{probes}, which is different from the resolution induced by the SRG
	scale.  How the latter should be chosen to best
	accommodate the former is a key unanswered question.}
	This physics is conventionally explained in terms of short-range correlation
	(SRC) phenomenology~\cite{Frankfurt:2008zv,Atti:2015eda}.  SRCs are two- or
	higher-body components of the nuclear wave function with high relative
	momentum and low center-of-mass momentum.  These explanations would seem to
	present a	puzzle for descriptions of nuclei with low-momentum Hamiltonians,
	for which	SRCs are essentially absent from the wave functions.

	This puzzle is resolved by the unitary transformations that mandate the
	invariance of cross section (cf.~Eq.~\eqref{eq:matrix_element_invariance}).
	The physics that was described by SRCs in the
	wave functions must shift to a different component, such as a two-body
	contribution from the current (cf.~Fig.~\ref{fig:replace_loop_contact}).
	This may appear to complicate the reaction
	problem just as we have simplified the structure part, but past work and
	analogies to other processes suggests that factorization may in fact
	become cleaner~\cite{Anderson:2010aq,Bogner:2012zm}.  One of our goals is to
	elucidate this issue, although we will only started doing so in
	\cite{More:2015tpa}.

	In particular, we take the first steps in exploring the interplay of
	structure and reaction as a function of kinematic variables and SRG decoupling
	scale $\lambda$ in a controlled calculation of a knock-out process.  There are
	various complications for such processes.  With RG evolution, a
	Hamiltonian---even with only a two-body potential initially---will develop
	many-body components as the decoupling scale decreases.  Similarly, a one-body
	current will develop two- and higher-body components.

	Our strategy is to avoid dealing with all of these complications
	simultaneously
	by considering the cleanest knock-out process: deuteron electrodisintegration
	with only an initial one-body current.  With a two-body system, there are no
	three-body forces or three-body currents to contend with.  Yet it still
	includes several key ingredients to investigate: i) the wave function will
	evolve with changes in resolution; ii) at the same time, the one-body current
	develops two-body components, which are simply managed; and iii) there are
	final-state interactions (FSI).  It is these ingredients that will mix under
	the RG evolution.  We can focus on different effects or isolate parts of the
	wave function by choice of kinematics.  For example, we can examine when the
	impulse approximation is best and to what extent that is a
	resolution-dependent assessment.


	\section{Test ground: Deuteron disintegration}

	\subsection{Formalism}

	\subsection{Numerical implementation}

	\subsection{Results}

	\section{Summary and Outlook}
