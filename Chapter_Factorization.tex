% !TEX root = More_PhD_Thesis.tex
\cleardoublepage
\chapter{Factorization}

	\section{Motivation}

	Most of the information we know about nuclear interactions and the properties
	of the nuclei comes from some kind of scattering experiments (either elastic
	or inelastic).  In such experiments we scatter a known probe off a nucleus
	and	extract information about nuclear interactions by looking at the final
	outcome of the scattering experiment.
	%
	\begin{figure}[thbp]
	\centering
	\includegraphics[width=0.5\textwidth]%
	{Factorization/knock_out_schematic.png}
	\caption{Schematic of a nucleon knockout reaction.}
	\label{fig:knock_out_schematic}
	\end{figure}
	%
	\begin{figure}[thbp]
	\centering
	\includegraphics[width=0.6\textwidth]%
	{Factorization/factorization_schematic_annotated}
	\caption{Schematic illustration of factorization between nuclear structure
		and reactions component.}
	\label{fig:factorization_schematic}
	\end{figure}
	%
	Figure~\ref{fig:knock_out_schematic} shows a schematic for a nucleon knockout
	reaction where the probe is electrons which in turn interact with the nucleus
	by emitting virtual photons.

	The process of extracting nuclear properties from such experiments relies
	on the assumption that the effects of the probe are well understood and can
	be separated from the nuclear interactions we are trying to study.  This is
	the	\emph{factorization} between the nuclear structure and the nuclear
	reaction components illustrated schematically in
	Fig.~\ref{fig:factorization_schematic}.  The reaction component describes the
	probe and the structure includes description of the initial and final states.
	This factorization between the structure and reaction components depends
	on the renormalization scale and scheme.  In some physical systems
	(e.g., in cold atoms near unitarity~\cite{Hoinka:2013fsa}), the scale and
	scheme dependence is very weak and can be safely neglected.  In some other
	physical system such as in deep inelastic scattering (DIS) in high-energy
	QCD, the scale and scheme dependence is very manifest.
	%
	\begin{figure}[thbp]
	\centering
	\includegraphics[width=0.55\textwidth]%
	{Factorization/factorization_schematic_highE_QCD}
	\caption{Factorization in high-energy QCD.  $x$ is the Bjorken-$x$ and it
		denotes the fraction of momentum of the nucleon carried by the parton
		under consideration.  $a$ denotes the parton flavor.}
	\label{fig:factorization_schematic_highE_QCD}
	\end{figure}
	%
	Figure~\ref{fig:factorization_schematic_highE_QCD} illustrates the
	factorization in DIS.  The form factor $F_2$ of the nucleon
	(which up to some kinematic factors is the cross section) is given by the
	convolution of the long-distance parton density and the short-distance
	Wilson coefficient.  In this case, the parton density forms the structure
	part which is non-perturbative and the Wilson coefficient form the reactions
	part which can be calculated in perturbative QCD.  These separation
	between long- and short-distance physics is not unique, and is defined by
	the factorization scale $\mu_f$.  To minimize the contribution of logarithms
	that can disturb the perturbative expansion, $\mu_f$ is chosen to be equal to
	the	magnitude of the four-momentum transfer $Q$.
	The form factor $F_2$ (because it is
	related to the observable cross section) is independent of $\mu_f$, but
	the individual components are not.  As a consequence, the parton density
	(or distribution) function $f_a(x, Q^2)$ runs with $Q^2$.
	%
	\begin{figure}[thbp]
	\centering
	\includegraphics[width=0.55\textwidth]%
	{Factorization/uquark_parton_distribution_3d_surf}
	\caption{Parton distribution for the up quarks in the proton as a function
		of $x$ and $Q^2$.  Figure taken from \cite{Furnstahl:2013dsa}.}
	\label{fig:uquark_PDF_3D}
	\end{figure}
	%
	This is demonstrated in Fig.~\ref{fig:uquark_PDF_3D}.  


	\section{Test ground: Deuteron disintegration}

	\section{Summary and Outlook}
