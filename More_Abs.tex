% !TEX root = More_PhD_Thesis.tex
\begin{abstract}

	New insights into the inter-nucleon interactions, developments in
	many-body technology, and the surge in computational capabilities
	has led to phenomenal progress in low-energy nuclear physics
	in the past few years.  Nonetheless, many calculations still lack
	a robust uncertainty quantification which is essential for making
	reliable predictions.  In this work we investigate two distinct sources
	of uncertainty and develop ways to account for them.

	Harmonic oscillator basis expansions are widely used in ab-initio nuclear
	structure calculations.
	Finite computational resources usually require that the basis be truncated
	before observables are fully converged, necessitating
	reliable extrapolation schemes.
	It has been demonstrated recently that errors introduced from basis
	truncation can be taken into account by focusing on the infrared and
	ultraviolet cutoffs induced by a truncated basis.
	We show that a finite oscillator basis effectively imposes a hard-wall
	boundary condition in coordinate space.
	We accurately determine the position of the hard-wall as a function of
	oscillator space parameters, derive infrared extrapolation formulas for
	the energy and other observables, and discuss the extension of this
	approach to higher angular momentum and to other localized bases.
	We exploit the duality of the harmonic oscillator to account for the errors
	introduced by a finite ultraviolet cutoff.

	Nucleon knockout reactions have been widely used to study and understand
	nuclear properties.  Such an analysis implicitly assumes that the
	effects of the probe can be separated from the physics of the target nucleus.
	This factorization between nuclear structure and reaction components
	depends on the renormalization scale and scheme, and has not been
	well understood.  But it is potentially critical for
	interpreting experiments and for extracting process-independent
	nuclear properties.
	We use a class of unitary transformations called the
	similarity renormalization group (SRG) transformations to systematically
	study the scale dependence of factorization for the simplest knockout process
	of deuteron electrodisintegration.
	We find that the extent of
	scale dependence depends strongly on kinematics, but in a \emph{systematic}
	way.
	We find a relatively weak scale dependence at the quasi-free kinematics
	that gets
	progressively stronger as one moves away from the quasi-free region.
	Based on examination of the relevant overlap matrix elements, we are able
	to qualitatively
	explain and even predict the nature of scale dependence based on the
	kinematics
	under consideration.


\end{abstract}
