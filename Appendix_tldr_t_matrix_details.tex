% !TEX root = More_PhD_Thesis.tex
% !TEX encoding = UTF-8 Unicode
% !TEX spellcheck = en_US
\cleardoublepage
\chapter{$t$-matrix details}
\label{Appendix:t_matrix_details}

  \section{Solving the Lippmann-Schwinger equation}

  The Lippmann-Schwinger equation (LSE), which is essentially the
  Schr\"{o}dinger equation for scattering states is given in operator form by
  %
  \beq
  t = V + V \, G \, t \;,
  \label{eq:LSE_operator_form}
  \eeq
  %
  where $t$ is the $t$-matrix, $V$ is the potential, and $G$ is the Green's
  function.  In momentum space, Eq.~\eqref{eq:LSE_operator_form} becomes
  \beq
  t_{\lp l} (E_0 \!=\! p_0^2/M ; \pp, p) = V_{\lp l} + \sum_{\lpp}
  \frac{2}{\pi} M \int \frac{\dd \ppp \, {\ppp}^2 V_{l \lpp}(\pp, \ppp) \,
    t_{\lpp l}(E_0; \ppp, p)}{p_0^2 - {\pp}^2 + i \epsilon} \;.
  \label{eq:t_matrix_mom_space}
  \eeq
  Derivation of Eq.~\eqref{eq:t_matrix_mom_space} makes use of the completeness
  relation in Eq.~\eqref{eq:completeness_partial_wave} and the definition of
  Green's function in Eq.~\eqref{eq:G0_def_pw}.   The indices which are the
  same on both sides of Eq.~\eqref{eq:t_matrix_mom_space} and are not summed
  over are suppressed.  For deuteron disintegration calculations, we need only
  the half on-shell $t$-matrix.  But here we will look at the more general case
  of evaluating the fully off-shell $t$-matrix.

  For the sake of pedagogy, let us consider that we are evaluating the
  $t$-matrix for uncoupled channels.  Suppressing the angular momentum indices
  and putting in the limits of integration,
  the Eq.~\eqref{eq:t_matrix_mom_space} becomes
  \beq
  t(E_0; \pp, p) = V(\pp, p) + \frac{2}{\pi} M \int_0 ^\Lambda
  \frac{\dd \ppp \, {\ppp}^2
  V(\pp, \ppp)\, t(E_0; \ppp, p)}{p_0^2 - {\ppp}^2 + i \epsilon} \;.
  \label{eq:t_uncoupled_channels}
  \eeq
  %
  Next we outline the steps involved in solving
  Eq.~\eqref{eq:t_uncoupled_channels} numerically.  We follow the approach of
  Ref.~\cite{Landau:1989}.

  The integrals are efficiently evaluated numerically using a Gauss-Legendre
  quadrature.  However, the integrand in Eq.~\eqref{eq:t_uncoupled_channels} has
  a pole at $\ppp = p_0$, and that needs to be accounted for properly.
  Let's consider the expression
  \bea
  \label{eq:separating_pole_first_step}
  \int_0 ^\Lambda \dd p \frac{p^2 f(p)}{p_0^2 - p^2 + i \epsilon}
  &=& \int_0 ^\Lambda \dd p \frac{p^2 f(p)}{(p_0 + p)(p_0 - p + i \epsilon)}
  \\ [0.4 em]
  &\equiv& \int_0 ^\Lambda \dd p \frac{\wt{f}(p)}{(p_0 - p + i \epsilon)} \;,
  \label{eq:separating_pole}
  \eea
  where we have defined $\wt{f}(p)$ as
  \beq
  \wt{f}(p) = \frac{p^2 \, f(p)}{p_0 + p} \;.
  \eeq
  %
  In principle, we can work without separating the singular and non-singular
  factors of $p_0^2 - p^2 + i \epsilon$.  However, we find better numerical
  convergence when the pole term is factorized as in
  Eq.~\eqref{eq:separating_pole_first_step}.

  Using Sokhotsky's formula
  \beq
  \frac{1}{x \pm i \epsilon} = \mathcal{P} \left(\frac{1}{x}\right) \mp i \,
  \pi \, \delta(x) \;,
  \eeq
  we have
  \beq
  \int_0 ^\Lambda \dd p \frac{\wt{f}(p)}{(p_0 - p + i \epsilon)} =
  \mathcal{P} \int_0 ^\Lambda \dd p \frac{\wt{f}(p)}{p_0 - p}  - i \, \pi \,
  \wt{f}(p_0)\;.
  \label{eq:f_tilde_simple_pole}
  \eeq
  %
  Let's first evaluate the principal value integration in the
  Eq.~\eqref{eq:f_tilde_simple_pole}.
  \bea
  \mathcal{P} \int_0 ^\Lambda \dd p \frac{\wt{f}(p)}{p_0 - p}
  & = &  \mathcal{P} \int_0 ^\Lambda \dd p \frac{\wt{f}(p) - \wt{f}(p_0) +
  \wt{f}(p_0)}{p_0 - p} \nonumber \\ [0.4 em]
  & = & \int_0 ^\Lambda \dd p \frac{\wt{f}(p) - \wt{f}(p_0)}{p_0 - p} +
  \wt{f}(p_0) \, \mathcal{P} \int_0 ^\Lambda \dd p \frac{1}{p_0 - p} \;.
  \label{eq:add_subtract_pole}
  \eea
  %
  The integrand of the first term on the right side of
  Eq.~\eqref{eq:add_subtract_pole} is zero at the pole $p = p_0$ and therefore
  non-singular.  We can therefore drop the principal value for that term and
  evaluate it as a normal integral.  The second term on the right side of
  Eq.~\eqref{eq:add_subtract_pole} can be evaluated analytically.
  \bea
  \mathcal{P} \int_0 ^\Lambda \dd p \frac{1}{p_0 - p}
  & = & \int_0 ^{p_0 - \epsilon} \dd p \frac{1}{p_0 - p} +
  \int_{p_0 + \epsilon} ^\Lambda \dd p \frac{1}{p_0 - p} \nonumber \\ [0.5 em]
  & = &  \null - {\rm{ln}}(p_0 - p) \Big\vert_0 ^{p_0 - \epsilon} +
         \null - {\rm{ln}}(p_0 - p) \Big\vert_{p_0 + \epsilon} ^\Lambda
         \nonumber \\ [0.5 em]
  & = & - {\rm{ln}}\left(\frac{\Lambda - p_0}{p_0}\right)\;.
  \label{eq:PV_1_over_x}
  \eea
  %
  From Eqs.~\eqref{eq:PV_1_over_x}, \eqref{eq:add_subtract_pole}, and
  \eqref{eq:f_tilde_simple_pole}, we have
  \begin{equation}
  \int_0 ^\Lambda \dd p \frac{\wt{f}(p)}{p_0 - p + i \epsilon}
  = \int_0 ^\Lambda \dd p \frac{\wt{f}(p) - \wt{f}(p_0)}{p_0 - p} -
  \wt{f}(p_0) \, {\rm{ln}}\left(\frac{\Lambda - p_0}{p_0}\right)
  \null - i \, \pi \, \wt{f}(p_0)\;.
  \end{equation}
  %
  Discretizing this on the Gauss-Legendre mesh we have
  \beq
  \int_0 ^\Lambda \dd p \frac{\wt{f}(p)}{p_0 - p + i \epsilon}
  = \sum_{j = 1}^N \frac{\wt{f}(p_j)}{p_0 - p_j} w_j - \wt{f}(p_0)
  \left[ i \pi + {\rm{ln}}\big(\frac{\Lambda - p_0}{p_0}\big) + \sum_{j=1}^N
  \frac{w_j}{p_0 - p_j}\right] \;.
  \label{eq:pole_GL}
  \eeq
  $p_j$'s are the momentum mesh points, $N$ is the number of mesh points,
  and $w_j$'s are the associated weights.

  Comparing Eqs.~\eqref{eq:t_uncoupled_channels} and
  Eq.~\eqref{eq:separating_pole}, the corresponding $\wt{f}$ function for
  the LSE is
  \beq
  \wt{f}(\ppp) = \frac{2}{\pi} M \frac{{\ppp}^2 V(\pp, \ppp) \, t(E_0; \ppp, p)}
  {(p_0 + \ppp)} \;.
  \eeq
  %
  Using the result of Eq.~\eqref{eq:pole_GL}, the LSE from
  Eq.~\eqref{eq:t_uncoupled_channels} on the Gauss-Legendre mesh becomes
  %
  \begin{multline}
  t(E_0; \pp, p) = V(\pp, p) + \frac{2}{\pi} M \sum_{j = 1}^N
  \frac{k_j^2 \, V(\pp, k_j) \, t(E_0; k_j, p)}{p_0^2 - k_j^2} w_j \\
  \null - \frac{2}{\pi} M \frac{p_0^2 \, V(\pp, p_0) \, t(E_0; p_0, p)}{2 p_0}
  \left[ i \pi + {\rm{ln}}\big(\frac{\Lambda - p_0}{p_0}\big)
  + \sum_{j = 1}^N \frac{w_j}{p_0 - k_j} \right] \;.
  \label{eq:LSE_GL}
  \end{multline}
  %
  Note that $k_j$ are Gauss-Legendre momentum mesh points and $w_j$ are the
  corresponding weights.

  Let's define an array $D$ such that
  \beq
  D_j = \left\{ \begin{array}{ll}
          \dfrac{2}{\pi} M \dfrac{k_j^2 \, w_j}{p_0^2 - k_j^2}
          \mbox{~~~~for $j = 1, \cdots, N$} \\ [0.7em]
          -\dfrac{2}{\pi} M \dfrac{p_0^2}{2 p_0} \Big( i \pi + {\rm{ln}}\big(
          \dfrac{\Lambda - p_0}{p_0}\big) + \sum_{j = 1}^N \dfrac{w_j}{p_0 - k_j}
          \Big) \mbox{~~for $j = N+ 1$}.
        \end{array} \right.
  \label{eq:D_def}
  \eeq
  %
  Using this definition of $D$ and with the identification that
  $k_{j = N+1} = p_0$, Eq.~\eqref{eq:LSE_GL} can be written as
  \beq
  t(E_0; \pp, p) - \sum_{j = 1}^{N+1} V(\pp, k_j) \, D_j \, t(E_0; k_j, p)
  = V(\pp, p) \;.
  \label{eq:t_D}
  \eeq

  To solve Eq.~\eqref{eq:t_D} in matrix form, we let $\pp \to \{p_i\}$,
  where $i = 1, \cdots, N$ are the Gauss-Legendre mesh points and $i = N+1$
  is the on-shell point $p = p_0$ \footnote{$k_i$'s and $p_i$'s are actually
  the same set of momentum points.  To avoid confusion, we keep the notation
  separate.}.  Eq.~\eqref{eq:t_D} can then be written as
  a matrix multiplication equation.
  \beq
  t(E_0; \pp_i, p) - \sum_{j = 1}^{N+1} V(\pp_i, k_j) \, D_j \, t(E_0; k_j, p)
  = V(\pp_i, p)
  \label{eq:t_D_p_i}
  \eeq
  \beq
  \sum_{j = 1}^{N+1} \underbrace{\big(\delta_{i j} - V(p_i, k_j) \, D_j\big)}
  _{\equiv \, F_{ij}} \, t(E_0, k_j, p)  =  V(p_i, p)
  \label{eq:F_def}
  \eeq
  \beq
  [F]_{(N+1)\times (N+1)} [t]_{N+1} = [V]_{N+1}
  \label{eq:F_t_matrix}
  \eeq
  %
  Note that $[V]$ in Eq.~\eqref{eq:F_t_matrix} is an array whose $N+1^{\rm{th}}$
  element is $V(p_0, p)$ and the first $N$ elements are $V(p_i, p)$, where as
  mentioned before $p_i$'s are the Gauss-Legendre mesh points.  The same
  indexing holds for $[t]$.

  $[F]$ and $[V]$ in Eq.~\eqref{eq:F_t_matrix} are
  known.  Eq.~\eqref{eq:F_t_matrix} can be solved using standard matrix equation
  solving subroutines to get the $t$-matrix array $[t]$.  Recall that
  $\displaystyle {[t] = \big( t(E_0; p_{j = 1,\cdots, N}, p), \, t(E_0; p_0, p)
  \big)}$.  Thus, for a given $p$ and $E_0$, we have the $t$-matrix
  $t(E_0; p_j, p)$ for any point $p_j$ on the Gauss-Legendre mesh, and also
  have it at the half on-shell point $t(E_0; p_0, p)$.  In principle, we can
  use any standard interpolation routine to get the $t$-matrix at a point
  not on the Gauss-Legendre mesh.  However, it turns out that we can use the
  LSE itself for interpolation.  Consider a point $\wt{p}$ not on the momentum
  mesh.  From Eq.~\eqref{eq:t_D}, we have
  \beq
  t(E_0; \wt{p}, p) = V(\wt{p}, p) + \sum_{j = 1}^{N+1}
  V(\wt{p}, k_j) \, D_j \, t(E_0, k_j, p) \;.
  \label{eq:t_interpolation}
  \eeq
  %
  As mentioned before $\{k_j\} = \{p_j\}$, and therefore all the terms on the
  right side of Eq.~\eqref{eq:t_interpolation} are known allowing us to
  evaluate $t(E_0; \wt{p}, p)$.  To interpolate the potential, which is stored
  on a momentum-space grid, we use
	the two-dimensional cubic spline algorithm from ALGLIB~\cite{ALGLIB:0915}.

  \medskip
  \subsubsection{Coupled channels}

  The neutron-proton system has both spin $S = 0$ and $S = 1$ channels.  The
  uncoupled channels have $S = 0$, whereas coupled channels have $S = 1$.
  The angular momentum numbers of the coupled channel pairs differ by $2$.
  For example, some of the coupled channel pairs are $^3 S_1$ - $^3 D_1$,
  $^3 P_2$ - $^3 F_2$, $^3 D_3$ - $^3 G_3$, so on.

  For coupled channels Eq.~\eqref{eq:LSE_operator_form} becomes
  \beq
  \left( \begin{array}{cc}
    t_{00} & t_{02} \\
    t_{20} & t_{22}
  \end{array} \right) =  \left( \begin{array}{cc}
    V_{00} & V_{02} \\
    V_{20} & V_{22}
  \end{array} \right) + \left( \begin{array}{cc}
    V_{00} & V_{02} \\
    V_{20} & V_{22}
  \end{array} \right)  \left( \begin{array}{cc}
    G_{0} & 0 \\
    0 & G_{0}
  \end{array} \right)  \left( \begin{array}{cc}
    t_{00} & t_{02} \\
    t_{20} & t_{22}
  \end{array} \right) \,
  \eeq
  where the subscripts $00$, $02$ etc.\ indicate the coupled channels.
  Following steps similar to the uncoupled case, the analog of
  Eq.~\eqref{eq:t_D_p_i} is
  \begin{multline}
    \left( \begin{array}{cc}
      t_{00}(\pp_i, p) & t_{02}(\pp_i, p) \\
      t_{20}(\pp_i, p) & t_{22}(\pp_i, p)
    \end{array} \right) - \sum_{j = 1}^{N+1} \left( \begin{array}{cc}
      V_{00}(\pp_i, k_j) & V_{02}(\pp_i, k_j) \\
      V_{20}(\pp_i, k_j) & V_{22}(\pp_i, k_j)
    \end{array} \right) \left( \begin{array}{cc}
      \mathcal{D}(k_j) & 0 \\
      0 & \mathcal{D}(k_j)
    \end{array} \right) \\  \left( \begin{array}{cc}
      t_{00}(k_j, p) & t_{02}(k_j, p) \\
      t_{20}(k_j, p) & t_{22}(k_j, p)
    \end{array} \right) =  \left( \begin{array}{cc}
      V_{00}(\pp_i, p) & V_{02}(\pp_i, p) \\
      V_{20}(\pp_i, p) & V_{22}(\pp_i, p)
    \end{array} \right) \;.
  \end{multline}
  %
  $\mathcal{D}$ is a $(N+1) \times (N+1)$ diagonal matrix element with the
  diagonal matrix elements given by $D$ from Eq.~\eqref{eq:D_def}.
  Each of $V_{0,2 \, 0,2}(\pp_i, k_j)$ are a $(N+1) \times (N+1)$ matrix,
  whereas $t_{0,2 \, 0,2}(\pp_i, p)$ and $V_{0,2 \, 0,2}(\pp_i, p)$ have
  dimensions of $(N+1) \times 1$.

  Analogous to Eqs.~\eqref{eq:F_def} and \eqref{eq:F_t_matrix}, we now have
  \begin{multline}
  \sum_{j = 1}^{N+1} \bigg[\delta_{i j} \left( \begin{array}{cc}
  \mathds{1}_{(N+1) \times (N+1)} & 0 \\
  0 & \mathds{1}_{(N+1) \times (N+1)}
  \end{array} \right) -
  \left( \begin{array}{cc}
  V_{00}(\pp_i, k_j) D_j & V_{02}(\pp_i, k_j) D_j \\
  V_{20}(\pp_i, k_j) D_j & V_{22}(\pp_i, k_j) D_j
  \end{array} \right)
  \bigg] \\
  \left( \begin{array}{cc}
  t_{00}(k_j, p) & t_{02}(k_j, p) \\
  t_{20}(k_j, p) & t_{22}(k_j, p)
  \end{array} \right)
  = \left( \begin{array}{cc}
    V_{00}(\pp_i, p) & V_{02}(\pp_i, p) \\
    V_{20}(\pp_i, p) & V_{22}(\pp_i, p)
    \end{array} \right) \;,
  \end{multline}
  and
  \beq
  [F]_{(2 N+2)\times (2N+2)} [t]_{(2 N+ 2) \times 2 } =
  [V]_{(2 N+2) \times 2} \;.
  \label{eq:F_t_coupled}
  \eeq
  Solving Eq.~\eqref{eq:F_t_coupled} for a given $E_0$ and $p$ yields
  $t_{0,2 \, 0,2}(E_0; k_j, p)$ for points $k_j$ on the mesh.  Analogous to
  Eq.~\eqref{eq:t_interpolation}, we can again use the LSE to interpolate
  the $t$-matrix in coupled channel.  Below, we note interpolation for one
  of the components for the point $p = \wt{p}$ not on the mesh.
  \beq
  t_{02}(E_0; \wt{p}, p) = V_{02}(\wt{p}, p) + \sum_{j = 1}^{N+1}
  V_{00}(\wt{p}, k_j) \, D_j \, t_{02}(E_0, k_j, p) + \sum_{j = 1}^{N+1}
  V_{02}(\wt{p}, k_j) \, D_j \, t_{22}(E_0, k_j, p)
  \label{eq:t_interpolation_coupled}
  \eeq
  Note that all the quantities on the right side of
  Eq.~\eqref{eq:t_interpolation_coupled} are known allowing us to evaluate
  $t_{02}(E_0; \wt{p}, p)$.  Similarly, we can write down equations for
  interpolation of other components of the $t$-matrix.

  \section{$t$-matrix checks}

  We checked the accuracy of our $t$-matrix by calculating the phase shifts and
  verifying them against standard values (such as from NN-online).
  For uncoupled channels, the on-shell part of the $t$-matrix is related to
  the phase shift as follows \footnote{Be aware that the factors of $M$
  (nucleon mass) and $\hbar$ might differ based on the conventions and units
  used.}
  \beq
  t_{l} (E_k; k, k) = \frac{\ee^{i \delta_l} \, \sin \delta_l}{- M \, k} \;.
  \label{eq:t_phase_shift}
  \eeq
  Thus, argument of the $t$-matrix gives the phase shift.
  \beq
  \delta_l(k) = {\rm{arg}} (t_{l} (E_k; k, k))
  \label{eq:delta_l_uncoupled}
  \eeq

  The coupled channel calculation involves an additional parameter called
  mixing angle denoted by $\bar{\epsilon}$.  In the ``Stapp'' or the ``bar''
  phase shift parametrization \cite{Stapp:1956mz}, the $S$-matrix is written
  as
  \beq
  S = \left( \begin{array}{cc}
  \cos 2 \bar{\epsilon} \, \ee^{2 i \bar{\delta}_1} &
  i \, \sin 2 \bar{\epsilon} \, \ee^{i (\bar{\delta}_1 + \bar{\delta}_2)} \\
  i \, \sin 2 \bar{\epsilon} \, \ee^{i (\bar{\delta}_1 + \bar{\delta}_2)} &
  \cos 2 \bar{\epsilon} \, \ee^{2 i \bar{\delta}_2}
  \end{array} \right) \;.
  \label{eq:S_stapp}
  \eeq
  %
  We calculate $S$ in terms $t$-matrix using (see for instance Eq.~(8.70) in
  \cite{Landau:1989})
  \beq
  S = \left( \begin{array}{cc}
  1 - 2 \, i \, k \, t_{00}(E_k; k, k )  &
  - 2 \, i \, k \, t_{02}(E_k; k, k ) \\
  - 2 \, i \, k \, t_{20}(E_k; k, k ) &
  1 - 2 \, i \, k \, t_{22}(E_k; k, k )
  \end{array} \right) \;.
  \eeq
  %
  From Eq.~\eqref{eq:S_stapp}, we can work out that the phase shifts and the
  mixing angles are given as follows.
  \bea
  \bar{\delta}_1 = \dfrac{1}{2} \tan^{-1} \left( \dfrac{
  {\rm{Im}}\big[S[1,1]\big]}
  {{\rm{Re}}\big[S[1,1]\big]} \right) \\ [0.5 em]
  \bar{\delta}_2 = \dfrac{1}{2} \tan^{-1} \left( \dfrac{
  {\rm{Im}}\big[S[2,2]\big]}
  {{\rm{Re}}\big[S[2,2]\big]} \right) \\ [0.5 em]
  \bar{\epsilon} = \dfrac{1}{2} \sin^{-1} \left( \dfrac{
  {\rm{Im}}\big[S[1,2]\big]}
  {{\rm{Re}}\big[\sqrt{{\rm{det}}(S)}\big]} \right)
  \eea
  %
  The phase shifts and the mixing angles calculated using these formulas match
  the results on NN-online.  This indicates that $t$-matrix we have is correct.

  \subsubsection{Checking the imaginary part of the $t$-matrix}

  The formulas for phase shifts check the ratio of real and imaginary parts of
  $t$-matrix.  This is particularly evident in Eq.~\eqref{eq:delta_l_uncoupled}.
  But it is also possible to check the imaginary part of the $t$-matrix.
  From Eq.~\eqref{eq:t_phase_shift}, we have (suppressing the arguments of the
  $t$-matrix)
  \bea
  t & = &  \frac{\sin \delta}{-M \, k \, (\cos \delta - i \, \sin \delta)}
  \;. \\
  \Rightarrow \dfrac{1}{-M \, t} & = & k \, \cot \delta - i \, k \;. \\
  \Rightarrow {\rm{Im}}[1/t] & \propto & k \;.
  \label{eq:Im_t_behavior}
  \eea
  Thus, the imaginary part of $1/t_l(E_k; k, k)$ when plotted as a function of
  $k$ should be a straight line.  Our $t$-matrix satisfies this condition.
  The slope of the line in this case is $M$, but in general depends on the
  units chosen.

  \subsubsection{Symmetric property of $t$-matrix}

  The phase shifts, mixing angles, and the behavior of the imaginary part of
  $t$-matrix described in Eq.~\eqref{eq:Im_t_behavior}, all check the
  on-shell part of the $t$-matrix.  To get some confidence about the off-shell
  part of $t$-matrix, we can check if it has the right symmetries.
  The $t$-matrix is symmetric under the angular momentum and momentum
  interchange, i.e.,
  \beq
  t(E_0; k, k^\prime, L, L^\prime, J, S , T) = t(E_0; k^\prime, k, L^\prime, L,
  J, S, T) \;.
  \eeq

  Note that the $t$-matrix is not Hermitian as one would naively expect.


  \section{Using the LSE for the wave function interpolation}

  Solving Schr\"{o}dinger's equation we obtain the deuteron wave function
  on the momentum mesh on which our potential is stored.  We can use the LSE
  to obtain the wave function at any intermediate momentum point.
  We use the property that near the bound state pole, the $t$-matrix
  factorizes as (see Appendix of Ref.~\cite{Koenig:2013} and references therein)
  \beq
  \lim_{E \to -E_B} (E + E_B) \, t(E; k, k^\prime) = \mathcal{B}^\ast (k) \,
  \mathcal{B} (k^\prime) \;,
  \label{eq:t_near_pole}
  \eeq
  where $E_B$ is the bound state energy, and the factor $\mathcal{B}$ is
  the wave function apart from a factor of propagator
  \beq
  \mathcal{B}(q) = \dfrac{-\pi (k_B^2 + q^2)}{2 \, M} \psi(q) \;,
  \label{eq:B_psi_relation}
  \eeq
  where $k_B$ is the bound state momentum.

  Multiplying the LSE equation for the $t_{22}$ channel by $(E + E_B)$ and
  taking the limit $E \to -E_B$, we have
  \begin{multline}
  \lim_{E \to -E_B} (E+E_B) \, t_{22} (E; k, k^\prime)
  = \lim_{E \to -E_B} (E+E_B) \, V_{22} (k, k^\prime) \\
  \null + \frac{2}{\pi} M
  \lim_{E \to -E_B}  \int \dd p \, p^2 \, V_{20} (k, p)
  \frac{t_{02}(E; p, k^\prime)}{k_E^2 - p^2 + i \epsilon} (E+E_B) \\
  \null + \frac{2}{\pi} M \lim_{E \to -E_B}  \int \dd p \, p^2 \, V_{22} (k, p)
  \frac{t_{22}(E; p, k^\prime)}{k_E^2 - p^2 + i \epsilon} (E+E_B)
  \label{eq:lim_LSE} \;.
  \end{multline}
  %
  The first term on the right side of Eq.~\eqref{eq:lim_LSE} vanishes as the
  potential does not have a singular part.
  Using Eq.~\eqref{eq:t_near_pole}, we get
  \begin{equation}
  \mathcal{B}_2^\ast(k) = \frac{2}{\pi} M \int \dd p \, p^2
  \frac{V_{20}(k, p) \, \mathcal{B}_0^\ast (p)}{-k_B^2 - p^2} +
  \frac{2}{\pi} M \int \dd p \, p^2
  \frac{V_{22}(k, p) \, \mathcal{B}_2^\ast (p)}{-k_B^2 - p^2} \;.
  \label{eq:eq_for_B2}
  \end{equation}
  %
  In deriving Eq.~\eqref{eq:eq_for_B2}, we have dropped the factor of
  $\mathcal{B}_2(k^\prime)$ which is common on both sides.
  Substituting Eq.~\eqref{eq:B_psi_relation} gives
  \beq
  \frac{-\pi}{2 \, M} \, (k_B^2 + k^2) \, \psi_2^\ast(k) =
  \int \dd p \, p^2 \, V_{20}(k, p) \psi_0^\ast(p) +
  \int \dd p \, p^2 \, V_{22}(k, p) \psi_2^\ast(p) \;.
  \eeq
  The wave function in our case is real and therefore the complex conjugation
  can be dropped.  Writing the integral in terms of sum, we have
  \beq
  \psi_2(k) = \frac{-2 \, M}{\pi \, (k_B^2 + k^2)} \left[
  \sum_{j = 1}^N w_j \, p_j^2 \, V_{20}(k, p_j) \, \psi_0(p_j)
  + \sum_{j = 1}^N w_j \, p_j^2 \, V_{22}(k, p_j) \, \psi_2(p_j) \right] \;.
  \label{eq:psi_2_interpolation}
  \eeq
  The wave functions on the mesh---$\psi_0(p_j)$ and $\psi_2(p_j)$---are
  already known from solving the Schr\"{o}dinger equation and therefore
  Eq.~\eqref{eq:psi_2_interpolation} allows us to obtain $\psi_2(k)$ for any
  desired momentum $k$.  We also checked that if we choose $k$ in
  Eq.~\eqref{eq:psi_2_interpolation} to be one of the on mesh points, then we
  get back the expected answer.

  Similarly for the $S$-state wave function, we have
  \beq
  \psi_0(k) = \frac{-2 \, M}{\pi \, (k_B^2 + k^2)} \left[
  \sum_{j = 1}^N w_j \, p_j^2 \, V_{00}(k, p_j) \, \psi_0(p_j)
  + \sum_{j = 1}^N w_j \, p_j^2 \, V_{02}(k, p_j) \, \psi_2(p_j) \right] \;.
  \label{eq:psi_0_interpolation}
  \eeq
  A word of caution---the $M$ in Eqs.~\eqref{eq:psi_2_interpolation} and
  \eqref{eq:psi_0_interpolation} depends on conventions.  In some cases, the
  mass factor is absorbed in the potential.  Same goes for the factor of
  $\hbar$'s.  Therefore, it is important to check that the units are consistent.

  The interpolation techniques for the wave functions and for the $t$-matrix
  (Eqs.~\eqref{eq:t_interpolation} and \eqref{eq:t_interpolation_coupled})
  keep the numerical errors minimal.  In our case, the only source of error is
  from interpolation of the potential.
