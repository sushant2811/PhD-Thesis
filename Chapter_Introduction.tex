% !TEX root = More_PhD_Thesis.tex
\cleardoublepage
\chapter{Introduction}

	\section{Overview of Nuclear Physics}

	Nuclear physics deals with studying the properties of nuclei.  Nuclear
	landscape shown in Fig.~\ref{fig:Nuclear_landscape} has been the
	traditional playground for nuclear physics.  The questions historically
	driving nuclear physics have been: how do protons and neutrons
	make stable nuclei and rare isotopes?  Where are the drip-line limits?
	What are the nuclear binding energies, excitation spectra, radii, so on?
	We would also like to describe nuclear reactions, make predictions about
	the shape of the nuclei and understand how the shape dictates the nuclear
	properties.
	%
	\begin{figure}[htbp]
	 \centering
	 \includegraphics[width=0.9\textwidth]%
	 {Introduction/Nuclear_landscape}
	 \caption{Nuclear landscape.  The uncertainties around drip lines (in red)
	  	were obtained by averaging the results of different theoretical models.
			Figure from \cite{Long_range_plan}.}
	 \label{fig:Nuclear_landscape}
	\end{figure}

	Around 1970, it became well-accepted that the nucleons (proton and neutrons)
	and other hadrons are composed of quarks, and that the quarks are held
	together through the exchange of gluons.  Following decades witnessed rapid
	development in the theory of strong interactions, the fundamental force
	describing the interactions between quarks and gluons.  This theory goes
	by the name of Quantum Chromodynamics (QCD).  One of the active areas of
	investigation is obtaining the hadron structure from QCD.  This includes,
	for example, understanding the origin of proton spin.  A related focus area
	is understanding the nature of the quark-gluon plasma (QGP)---the phase
	in which universe is believed to exist for up to a few milliseconds after
	the Big Bang.  The energies involved in this subfield are higher (few GeVs)
	than the energies in `traditional' nuclear physics (few MeVs) introduced in
	the opening paragraph.  Therefore it is conventional to refer to the two
	subfields as high-energy nuclear physics and low-energy nuclear physics.
	The work in this thesis will mainly focus on questions in low-energy nuclear
	physics (LENP).

	Apart from the questions at the core of LENP, mentioned in connection to
	Fig.~\ref{fig:Nuclear_landscape}, inputs from LENP are immensely important
	in other areas as well.  One such broad area is that of nuclear astrophysics.
	Majority of the stable
	and known nuclei shown in Fig.~\ref{fig:Nuclear_landscape} were formed in
	big bang, stellar, or supernova nucleosynthesis.  Inputs from LENP are
	critical in understanding the processes involved in nucleosynthesis and
	predicting the observed abundances of isotopes.
	Neutron stars are another fascinating astrophysical objects for low-energy
	nuclear physicists.  We would like to determine the equation of state for
	neutron star and understand how and why the stars explode.

	Finally, there are questions about the fundamental symmetries of the universe
	where
	nuclear physics hopes to make significant contributions.  For instance,
	why is there more matter than antimatter in the universe?  What is the nature
	of dark matter?  What is the nature of the neutrinos (Majorana or Dirac
	fermions) and how have they shaped the evolution of the universe?
	In fact, as we will see later accurate calculation of nuclear matrix elements
	is critical for the experiments undertaken to understand the nature of the
	neutrinos.

	In addition to the broad scientific impact of nuclear physics that we have
	already mentioned, it also has many real-life applications.  Our knowledge of
	nuclei and ability to produce them has led to increase in quality of life
	for the humankind.  Applications of nuclear physics encompass a diverse
	domain including but not limited to energy, security, medicine,
	radioisotope dating, and material sciences.


	\section{Checkered past; promising future}

	In 1935, Hideki Yukawa proposed the seminal idea of nuclear interaction
	being mediated by a massive boson \cite{Yukawa:1935xg}.   This could explain
	how protons and neutrons would stay bound in a nucleus overcoming the Coulomb
	repulsion between protons.  The fact that such a model described scattering
	data well at low energies (few MeVs) and eventual discovery of pions in 1947
	led to a wide acceptance of this model.  Very soon other heavy mesons
	($\rho$, $\omega$, $\sigma$) were discovered as well.  Scattering experiments
	also indicated that the strength on the nuclear potential depended on
	distance and at short distances the potential was repulsive.

	By 1950, there emerged an industry for coming up with better nuclear
	potentials.  These boson-exchange models shared some common features.
	The long-range part of the nucleonic interaction was given by pion exchange,
	the intermediate range was governed by multiple (mostly two) pion exchange,
	and short-range repulsion was thought to be because of overlap of nucleons.
	When heavy mesons were discovered, they were added to the intermediate range
	sector.  (Pion being the lightest meson has the longest range.)
	These general considerations form the basis for phenomenological potentials
	used even today as seen in Fig.~\ref{fig:Nuclear_landscape}.
	%
	\begin{figure}[htbp]
	 \centering
	 \includegraphics[width=0.6\textwidth]%
	 {Introduction/nuclear_potentials}
	 \caption{AV18 \cite{Wiringa:1994wb}, Reid93 \cite{Stoks:1994wp}, and
	  	Bonn \cite{Machleidt:1989tm} potentials for $^1S_0$ channel as
			functions of internucleonic distance.  These potentials accurately
			describe neutron-proton scattering up to laboratory energies of
			300 MeV.  Regions I, II, and III correspond to long-range,
			intermediate-range, and short-range parts discussed in the text.
			Figure from \cite{Aoki:2008hh}. }
	 \label{fig:Nuclear_landscape}
	\end{figure}

	This intense effort is well summarized by Hans Bethe's quote in his essay
	`What Holds the Nucleus Together?' in Scientific American (1953):
	``In past quarter century physicists have devoted a huge amount of
	experimentation and mental labor to this problem -- probably more man-hours
	than have been given to any other scientific question in the history of
	mankind.''
	The boson models did not have a smooth sailing though.  In particular
	the intermediate range multi-pion sector was beset with problems.  The
	pessimism this resulted in is palpable in Marvin Goldberger's comment
	in 1960: ``There are few problems in nuclear theoretical physics which
	have attracted more attention that that of trying to determine the
	fundamental interaction between two nucleons.  It is also true that
	scarcely ever has the world of physics owed so little to so many...It
	is hard to believe that many of the authors are talking about the same
	problem or, in fact, that they know what the problem is.'' A running joke
	was that nuclear physics is really `unclear' physics!

	There was relatively slow progress with regards to the development of
	inter-nucleonic potential in 1970's and 80's.  However, this period saw
	a rapid development of QCD.  It was realized that the nucleons and pions
	are composed of quarks which are held together by exchange of gluons
	(cf.~Fig.~\ref{fig:nucleon_interaction}).
	This pushed the effort to derive the nuclear
	potential from the `fundamental' theory of QCD.

	%
	\begin{figure}[htbp]
		\centering
		\begin{subfigure}[c]{0.43\textwidth}
			\centering
			\includegraphics[width=\textwidth]
			{Introduction/nucleon_pion_exchange}
			\caption{Inter-nucleon interaction in 1940}
			\label{fig:nucleon_pion_exchange}
		\end{subfigure}
		\hspace{0.07\textwidth}
		\begin{subfigure}[c]{0.42\textwidth}
			\centering
			\includegraphics[width=\textwidth]
			{Introduction/nucleon_gluon_exchange}
			\caption{Inter-nucleon interaction in 1980}
			\label{fig:nucleon_gluon_exchange}
		\end{subfigure}
		\caption{Evolution of the inter-nucleon interaction picture over time.}
		\label{fig:nucleon_interaction}
	\end{figure}
	%
	However, the effort to replace the hadronic descriptions at ordinary nuclear
	densities with quark description as in Fig.~\ref{fig:nucleon_gluon_exchange}
	was not very fruitful.
	%
	\begin{figure}[htbp]
	 \centering
	 \includegraphics[width=0.6\textwidth]%
	 {Introduction/alpha_QCD_running}
	 \caption{Summary of measurements of the QCD coupling $\alpha_s$ as a
	 function of energy scale $Q$. }
	 \label{fig:alpha_QCD_running}
	\end{figure}
	%
	As seen in Fig.~\ref{fig:alpha_QCD_running}, the strength of the QCD
	coupling $\alpha_s$ increases with decreasing energies.  This makes
	QCD non-perturbative in the low-energy regime of nuclear physics limiting
	the success of analytical calculations.

	Another aspect that make low-energy nuclear physics difficult is that it is
	a many-body problem.  It exhibits some emergent phenomena that cannot be
	explained in reductionist approach.  This issue has been well-summarized by
	the famous article `More is different' by Phillip Anderson
	\cite{Anderson:1972pca} (albeit with a focus on many-body problem in
	condensed matter).

	Despite all these challenges, great strides have been made in LENP
	in the last few decades.  As indicated in Fig.~\ref{fig:Moores_law_LENP}
	%
	\begin{figure}[htbp]
	 \centering
	 \includegraphics[width=0.7\textwidth]%
	 {Introduction/Moores_law_LENP}
	 \caption{LENP version of Moore's law and its violation.  $Y$ axis is the mass
	  	number of nuclei that can be calculated from ab-initio calculations.
			In past few years, it has been possible to push the ab-initio frontier to
			heavier nuclei.  Figure courtesy of Gaute Hagen. }
	 \label{fig:Moores_law_LENP}
	\end{figure}
	%
	particularly the last few years have seen an explosion in capabilities of
	low-energy nuclear theory.  This phenomenal progress has been possible due
	to combination of a few factors --- new insights about the nuclear force,
	developments in the many-body technology, and the surge in computational
	prowess.  In the following sections, we look briefly at each of these
	developments which will lead us to how author's PhD work fits into the
	bigger picture.


	\section{Understanding the Force}
	\label{sec:recent_advances}

	\subsection{EFT and RG techniques}

	\begin{figure}[htbp]
	 \centering
	 \includegraphics[width=0.6\textwidth]%
	 {Introduction/degrees_of_freedom2}
	 \caption{Hierarchy of degrees of freedom and associated energy scales in
	  	nuclear physics \cite{LRP:2007}.}
	 \label{fig:degrees_of_freedom}
	\end{figure}
	%
	An intriguing aspect of the world we live in, is that there are interesting
	phenomena at virtually all energy and length scale we can probe.  From TeV
	energies at the Large Hadron Collider (LHC) to the life-defining process of
	respiration which has the energy scale of only few meV, there are physical
	processes of interest at each step.  Nuclear physics spans a wide range of
	energy and length scales (cf.~Fig.~\ref{fig:degrees_of_freedom}); probably
	a wider range compared to most subfields.  This hierarchy provides both
	challenges and opportunities.

	Figure~\ref{fig:degrees_of_freedom} indicates the relevant degrees of freedom
	for the given energy scales.  Even though degrees of freedom are a matter of
	choice, in practice, appropriate degrees of freedom often dictate the
	success of a theory.  To quote Steven Weinberg \cite{Guth:1984rq}:
	``You can use any degrees of freedom you want, but if you use the wrong ones,
	you'll be sorry.''
	Weinberg in his seminal paper \cite{Weinberg:1978kz} applied the concept
	of effective field theory (EFT) to low-energy QCD.  This effort
	proceeds by writing down the most general Lagrangian consistent with the
	(approximate) symmetries.

	QCD has the expected symmetries of translational, Galilean, and rotational
	invariance, and spatial reflection and time reversal.  Along with that,
	in the limit of vanishing quark masses, QCD Lagrangian also possess an exact
	chiral symmetry \cite{Peskin1995a}.  If the chiral symmetry holds,
	``left''- and
	``right''-handed fields do not mix.  Like with any other continuous symmetry,
	spontaneous breaking of chiral symmetry leads to massless Goldstone boson(s).
	The masses of the up and down quark (quarks relevant in LENP) are both small
	($\sim 2 - 6$ MeV \cite{Agashe:2014kda}), but non-zero.  Therefore the
	chiral symmetry of QCD Lagrangian is only approximate and the resulting
	Goldstone boson --- pion --- is light (compared to mass of nucleon), but not
	massless.

	Chiral EFT ($\chi$-EFT) uses nucleons and pions as degrees of freedom.
	Heavy mesons are
	integrated out.  The crucial difference that distinguishes $\chi$-EFTs from
	meson theories of 1950s is that they are constrained by the chiral symmetry.
	Broken chiral symmetry serves as a connection with the underlying theory of
	QCD.
	A major advantage of $\chi$-EFT is that it permits systematic improvements
	and allows the possibility of having reliable uncertainty quantification.


	\begin{figure}[htbp]
	 \centering
	 \includegraphics[width=0.8\textwidth]%
	 {Introduction/ChEFTTerms}
	 \caption{Diagrams in a chiral Lagrangian at each order.  Solid lines are the
	 nucleons and dashes lines pions.  Figure from \cite{MAchleidt:2011zz}.}
	 \label{fig:ch_EFT_diagrams}
	\end{figure}
	%
	Diagrams in a $\chi$-EFT are shown in Fig.~\ref{fig:ch_EFT_diagrams}.  The
	three- and higher-body forces appear naturally in $\chi$-EFT.  The coupling
	constants at the vertices in Fig.~\ref{fig:ch_EFT_diagrams} are called
	low-energy coupling constants (LECs); they encode the QCD physics and
	cannot be calculated in $\chi$-EFT.  LECs are fitted to experimental data.
	Lattice QCD provides a promising method for extracting them in near
	future (cf.~Sec.~\ref{sec:lattice_QCD}).  Despite some concerns over
	power-counting (\emph{cite Bira here?}), chi-EFT good tool (systematics).
	Cite Machledit's
	two references.

	CD-Bonn

	AV18

	RG



	\subsection{SRG}
	\label{subsec:SRG_intro}

	\section{Many-body methods}

	Developments in many-body methods.

	Add stuff about lattice.

	Nuclear landscape chart with different methods.

	\section{Getting it from the bit --- lattice theories}
	\label{sec:lattice_QCD}

	``quarks and gluons have so many degrees of freedom, in terms of spin,
	color charge, and flavors, that non-perturbative numerical calculations of
	QCD are still just outside the realm of the physics of nucleons. For the
	lightest nuclei, this situation should be resolved within the next few years
	cite[Detmold2012], though direct applicability to many-nucleon system will
	still be inaccessible.""

	\section{Path forward for LENP}

	\section{Thesis organization}
