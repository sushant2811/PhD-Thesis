% !TEX root = More_PhD_Thesis.tex
\cleardoublepage
\chapter{Introduction}

	\section{Overview of Nuclear Physics}

	Nuclear physics deals with studying the properties of nuclei.  Nuclear
	landscape shown in Fig.~\ref{fig:Nuclear_landscape} has been the
	traditional playground for nuclear physics.  The questions historically
	driving nuclear physics have been: how do protons and neutrons
	make stable nuclei and rare isotopes?  Where are the drip-line limits?
	What are the nuclear binding energies, excitation spectra, radii, so on?
	We would also like to describe nuclear reactions, make predictions about
	the shape of the nuclei and understand how the shape dictates the nuclear
	properties.
	%
	\begin{figure}[htbp]
	 \centering
	 \includegraphics[width=0.9\textwidth]%
	 {Introduction/Nuclear_landscape}
	 \caption{Nuclear landscape.  The uncertainties around drip lines (in red)
	  	were obtained by averaging the results of different theoretical models.
			Figure from \cite{Long_range_plan}.}
	 \label{fig:Nuclear_landscape}
	\end{figure}

	Around 1970, it became well-accepted that the nucleons (proton and neutrons)
	and other hadrons are composed of quarks, and that the quarks are held
	together through the exchange of gluons.  Following decades witnessed rapid
	development in the theory of strong interactions, the fundamental force
	describing the interactions between quarks and gluons.  This theory goes
	by the name of Quantum Chromodynamics (QCD).  One of the active areas of
	investigation is obtaining the hadron structure from QCD.  This includes,
	for example, understanding the origin of proton spin.  A related focus area
	is understanding the nature of the quark-gluon plasma (QGP)---the phase
	in which universe is believed to exist for up to a few milliseconds after
	the Big Bang.  The energies involved in this subfield are higher (few GeVs)
	than the energies in `traditional' nuclear physics (few MeVs) introduced in
	the opening paragraph.  Therefore it is conventional to refer to the two
	subfields as high-energy nuclear physics and low-energy nuclear physics.
	The work in this thesis will mainly focus on questions in low-energy nuclear
	physics (LENP).

	Apart from the questions at the core of LENP, mentioned in connection to
	Fig.~\ref{fig:Nuclear_landscape}, inputs from LENP are immensely important
	in other areas as well.  One such broad area is that of nuclear astrophysics.
	Majority of the stable
	and known nuclei shown in Fig.~\ref{fig:Nuclear_landscape} were formed in
	big bang, stellar, or supernova nucleosynthesis.  Inputs from LENP are
	critical in understanding the processes involved in nucleosynthesis and
	predicting the observed abundances of isotopes.
	Neutron stars are another fascinating astrophysical objects for low-energy
	nuclear physicists.  We would like to determine the equation of state for
	neutron star and understand how and why the stars explode.

	Finally, there are questions about the fundamental symmetries of the universe
	where
	nuclear physics hopes to make significant contributions.  For instance,
	why is there more matter than antimatter in the universe?  What is the nature
	of dark matter?  What is the nature of the neutrinos (Majorana or Dirac
	fermions) and how have they shaped the evolution of the universe?
	In fact, as we will see later accurate calculation of nuclear matrix elements
	is critical for the experiments undertaken to understand the nature of the
	neutrinos.

	In addition to the broad scientific impact of nuclear physics that we have
	already mentioned, it also has many real-life applications.  Our knowledge of
	nuclei and ability to produce them has led to increase in quality of life
	for the humankind.  Applications of nuclear physics encompass a diverse
	domain including but not limited to energy, security, medicine,
	radioisotope dating, and material sciences.


	\section{Chequered past; promising future}

	Yukawa

	Hanse Bethe

	Goldberger

	Unclear physics

	alpa running

	Many-body problem



	Add jokes about `Unclear' physics, `so little owed to so many',
	`More is different'




	Exceeding More's law plot.

	\section{Recent Advances}
	\label{sec:recent_advances}

	\subsection{EFT and RG techniques}

	\begin{figure}[htbp]
	 \centering
	 \includegraphics[width=0.6\textwidth]%
	 {Introduction/degrees_of_freedom2}
	 \caption{Hierarchy of degrees of freedom and associated energy scales in
	  	nuclear physics \cite{LRP:2007}.}
	 \label{fig:degrees_of_freedom}
	\end{figure}
	%
	An intriguing aspect of the world we live in, is that there are interesting
	phenomena at virtually all energy and length scale we can probe.  From TeV
	energies at the Large Hadron Collider (LHC) to the life-defining process of
	respiration which has the energy scale of only few meV, there are physical
	processes of interest at each step.  Nuclear physics spans a wide range of
	energy and length scales (cf.~Fig.~\ref{fig:degrees_of_freedom}); probably
	a wider range compared to most subfields.  This hierarchy provides both
	challenges and opportunities.

	\subsection{SRG}
	\label{subsec:SRG_intro}

	\subsection{Many-body methods}

	Developments in many-body methods.

	Add stuff about lattice.

	Nuclear landscape chart with different methods.

	\section{Path forward for LENP}

	\section{Thesis organization}
